\documentclass{beamer}

\usetheme{Madrid} 
\setbeamertemplate{section in toc}{\inserttocsectionnumber.~\inserttocsection}
\setbeamertemplate{subsection in toc}{
  \hspace{5mm}\inserttocsectionnumber.\inserttocsubsectionnumber.~\inserttocsubsection
  \breakhere}
\setbeamertemplate{subsubsection in toc}{}
\setbeamertemplate{navigation symbols}{} % To remove the navigation symbols from the bottom of all slides uncomment this line


\AtBeginSubsubsection[]{
  \begin{frame}
   \frametitle{Parte \insertsectionnumber: \insertsectionhead}
    
    \begin{center}
      \huge{\insertsubsectionhead}
    \end{center}
    \vspace{5mm}
    \begin{center}
      \begin{beamercolorbox}[sep=8pt,center,shadow=true]{title}
        \usebeamerfont{title}
        \insertsubsubsectionhead\par%\\
      \end{beamercolorbox}
    \end{center}
  \end{frame}
}

\newcommand{\ib}[1]{\item \textbf{#1} }

\usepackage{graphicx} % Allows including images
\usepackage{booktabs} % Allows the use of \toprule, \midrule and \bottomrule in tables
\usepackage{import}
\usepackage{eurosym}
\usepackage[utf8]{inputenc}
\usepac kage[spanish]{babel}
\usepackage{media9}    
\usepackage{etoolbox} 
\usepackage{tikz}
\usepackage{subcaption}

\usepackage[sort&compress,sectionbib]{natbib}
\bibliographystyle{IEEEtranN}

\newcommand{\framei}[4]{
  \begin{frame}{#1}
    \begin{center}
      \includegraphics[width=0.9\linewidth,height=0.7
      \textheight,keepaspectratio]{img/#2}\\
      #3~\cite{#4}.
    \end{center}
  \end{frame}
}

\newcommand{\framein}[3]{
  \begin{frame}{#1}
    \begin{center}
      \includegraphics[width=0.9\linewidth,height=0.7
      \textheight,keepaspectratio]{img/#2}\\
      #3.
    \end{center}
  \end{frame}
}

\newcommand{\framet}[3]{
  \begin{frame}{#1}
    #2
  \end{frame}
}
\title[Descentralizando la ciencia]{\huge{Descentralizando la ciencia:}\\
  Plataforma distribuida para publicación de artículos académicos}

\author[V. Jacynycz]{Viktor Jacynycz Garc\'ia}

\institute[UCM] % Your institution as it will appear on the bottom of every slide, may be shorthand to save space
{
  Grupo Grasia\\
  Facultad de informática\\
  Universidad Complutense de Madrid \\ % Your institution for the title page
  \medskip
  \textit{vsjg@ucm.es} % Your email address
}
\date{} % Date, can be changed to a custom date

\setcitestyle{numbers,square}
% ~\citestyle{nature}
\bibliographystyle{unsrtnat}

\begin{document}
\begin{frame}
\titlepage % Print the title page as the first slide
\end{frame}

\begin{frame}
\frametitle{\'Indice}
\tableofcontents 
\end{frame}

\section{Contexto}
\subsection{Contexto Socio-cultural}
\subsubsection{Sistemas de publicación}
\framei{Sistemas de publicación}{journal.png}{Los primeros \emph{journals} comenzaron a
  surgir a partir de 1665}{kronick1976history}

\framei{Método de publicación}{publishing.png}{Antiguamente las ediciones de los
  \emph{journals} se imprimían en papel a través de las imprentas}{spier2002history}

\framei{El proceso de revisión por pares (\emph{1972})}{peerreview.png}{El proceso de revisión
  por pares determina la eligibilidad de un artículo para ser publicado en un \emph{journal}}{spier2002history}

\framei{Papel de los editores hoy en día}{publishers2.png}{Con la llegada de
  internet, el papel de los editores se cuestiona, ya que no es necesario
  imprimir los artículos para publicarlos}{lariviere2015oligopoly}

\subsubsection{Sistemas de reputación}
\framei{Sistemas de reputación}{reputation.png}{Los sistemas de reputación nos permiten
  confiar en terceros dentro de un sistema, sin tener que conocerlos
  previamente}{resnick2000reputation}

\framein{Sistemas de reputación}{reputation2.png}{Hoy en día muchos servicios
  utilizan estos sistemas}

\subsection{Contexto Técnico}
\subsubsection{Arquitecturas distribuidas}
\framei{Arquitecturas de las redes}{architectures.png}{Diferentes arquitecturas de una red}{baran1964distributed}

\framei{Blockchain}{blockchain.png}{Funcionamiento de la primera cadena de bloques}{nakamoto2008bitcoin}

\framei{Sistema basado en transacciones}{transaction.png}{Las transacciones generan un cambio de estado}{antonopoulos2014mastering}

\framei{Smart contracts}{sc.png}{Los contratos inteligentes son
  pequeños fragmentos de código que cambian el estado de la cadena de bloques}{buterin2014ethereum}

\subsection{Estado del arte}
\subsubsection{Métodos alternativos de publicación}
\framet{Métodos alternativos de publicación}{
  \begin{itemize}
    \ib{Open Access Journals:} Journals que ofrecen sus papers de manera
    gratuita, pero normalmente el precio de publicación recae sobre los autores~\cite{solomon2012study}.
    \ib{\emph{Open Journal Systems}:} Programas diseñados para facilitar el
    proceso de publicación para \textit{journals} open access~\cite{willinsky2005open}.
    \ib{Preprints:} Son artículos pendientes de publicación accesibles
    para cualquier persona~\cite{shuai2012scientific}.
    \ib{Mega-journals:} Combinación de varios \emph{journals} en uno solo para
    fomentar la idea del open access~\cite{binfield2013open}
    \ib{Publicación continua:} Metodología que utilizan algunos
      \emph{journals} online. Consiste en publicar directamente los artículos
      aceptados~\cite{anderton2013continuous}.
  \end{itemize}
}

\subsubsection{Sistemas actuales de reputación}
\framein{Sistema de votación}{se.png}{Stack exchange es una de las comunidades más
  conocidas que implementa un sistema de reputación}

\framein{Sistema de rating}{amazon.jpg}{Amazon, Google y muchas otras empresas  tienen un sistema de
  puntuación basado en 5 estrellas}
\subsection{Objetivos}
\framet{Objetivos de la plataforma}{
\begin{itemize}
  \item \textbf{Crear} un sistema de publicación de ciencia abierto y descentralizado.
  \item \textbf{Reducir} el tiempo de revisión de los artículos académicos.
  \item Explorar \textbf{métricas alternativas} para los artículos, journals y revisores.
  \item \textbf{Prevenir} posibles problemas de nepotismo o sexismo en la revisión por pares.
\end{itemize}
}

\section{Plataforma}
\subsection{Descripción general}
\subsubsection{Idea principal}

\framein{Idea principal}{eipfs.png}{La plataforma utiliza \textbf{IPFS} como sistema de archivos y \textbf{Ethereum} como sistema de gestión para facilitar el proceso de publicación científica}

\subsection{Pilares fundamentales}
\subsubsection{Red P2P distribuida}

\framet{Red P2P distribuida}{Una red P2P ofrece tanto una gran persistencia de
  datos como robustez frente a la censura o la prohibición
  \vfill
  La plataforma ofrece una descentralizacion de todo el proceso, incluyendo:
  envío de papers, asiganción de revisores, envío de revisiones y publicación.
}

\subsubsection{Sistema de reputación de revisores}
\framet{Sistema de reputación de revisores}{El sistema de reputación de revisores tiene las siguientes características
\begin{itemize}
    \item Revisiones totalmente públicas y accesibles.
    \item Cada revisión es puntuable por los autores y los otros revisores.
    \item Revisores obtienen reputación por \emph{hacer bien} su trabajo.
\end{itemize}
}

\framet{Sistema de reputación de revisores}{
Las revisiones tienen un tiempo límite para ser realizadas.\\
 \vfill
Al realizar la entrega, los autores y los otros revisores podrán votar la
revisión.\\
\vfill
De esta manera se evitan posibles revisiones injustas o incompletas.
}

\subsubsection{Open Science por diseño}


\framet{Open Science por diseño}{Al estar en \textbf{IPFS} y \textbf{Ethereum}
  implica que:
  \begin{itemize}
  \item Todos los contenidos son públicos y gratuitos, tanto para los autores como
    para los revisores.
  \item Todo el proceso es auditable, por lo que se puede saber de antemano toda
    la información sobre revisores, autores y journals.
  \end{itemize}}



\subsection{Arquitectura}

\subsubsection{Frontend y Backend}

\framein{Estructura del Frontend}{homepage.png}{HTML + JavaScript + Metamask}
\framein{Estructura del Frontend}{topic.png}{HTML + JavaScript + Metamask}
\framein{Estructura del Frontend}{rating.png}{HTML + JavaScript + Metamask}

\framein{Estructura del Backend}{architecture.png}{El diseño une las dos
  tecnologías bajo la misma plataforma}

\section{Conclusiones y trabajo futuro}
\framet{Conclusiones}{
  \begin{itemize}
    \item Descentralizar los servicios de publicación abre muchísimas
      posibilidades a favor del Open Access.
    \item La transparencia que ofrece esta plataforma genera debates sobre la
      privacidad de las revisiones y los revisores.
    \item La eliminación de intermediarios que se lucran de este sistema
      beneficiará a la comunidad cientifica.
  \end{itemize}
}

\framet{Trabajo futuro}{
  \begin{itemize}
  \item Buscar journals independientes que quieran cambiar el proceso de publicación científica.
  \item Finalizar el desarrollo de la plataforma.
  \item Comenzar una etapa de pruebas.
  \item Desplegar el contrato en blockchain.
  \item Legitimizar esta plataforma.
  \end{itemize}
}

\section{Referencias}
\begin{frame}[allowframebreaks]{Referencias}
  \bibliography{references}

  \printbibliography
\end{frame}

\end{document} 