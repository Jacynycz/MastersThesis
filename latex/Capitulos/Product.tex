\chapter{Proof of concept}

\begin{FraseCelebre}
  \begin{Frase}
    Sooner or later you're going to realize just as I did that there's a
    difference between knowing the path and walking the path.
  \end{Frase}
  \begin{Fuente}
    Morpheus - The Matrix
  \end{Fuente}
\end{FraseCelebre}

As proof of concept, the final platform was uploaded to a server with an IPFS
node. Users can interact with this platform and connec to to a blockchain test
net in which all the accounts have unlimited money to make all the transactions
to interact with Ethereum.

As part of a decentralized system, users may run a local node instead of
connecting to the gateway provided. The source code of the platform is open
source, and can be obtained and deployed in an Ethereum and IPFS node.

\sst{Frontpage}

As showed in the figure~\ref{poc:fp}, the front page of the platform shows three
main sections:

\begin{itemize}
  \itbf{A showcase showing the journals registered:} All journals registered in
  the platform are automatically showed in the front page in order of
  popularity. Showing the most popular journals first (i.e. journals with the
  most publications).

  \itbf{A list of filters:} A list showing the most relevant topics containing a
  link to the topic search page.

  \itbf{Register jornal link:} A link to the journal registration page.

\end{itemize}

The platform also has a searchbox to search queries within the platform.

\figuraHere{homepage.png} {width=0.9\linewidth}{The front page of the
  platform}{poc:fp}

\sst{Register Journal}

Decentralized Science offers the possibility to register journals within the
platform. Journal editors who want to adopt this system, can implement a smart
contract template hosted in github, upload it to Ethereum and provide the
address the registration page. This will create a transaction in Ehtereum and
will link the new registered journal in the platform. \figuraHere{register.png}%
{width=0.9\linewidth}%
{Register journal page}%
{poc:rj}

The figure~\ref{poc:rj} shows the registration page for journals. The HTML form
creates an Ethereum transaction to Decentralized Science's address. The address
provided must be an extension of the original contract for a decentralized
journal, otherwise the transaction will be unsuccesful. \sst{Journal Page}

The journal page contains all the infomation about the journal:
\begin{itemize}
\item Title of the journal.
\item A short description about the journal.
\item An image of the journals cover.
\item Tags of the journal with links to a search of each one.
\item Lastest papers published and a link to a list to all the papers of the
  journal.
\item A link to the paper subsission page.
\item The average review time of the papers published in the journal.
\end{itemize}

\figuraHere{journal.png}%
{width=0.9\linewidth}%
{Journal page}%
{poc:j}

The figure~\ref{poc:j} shows the journal's page with all the information
mentioned above. The HTML connects to the journal's Ethereum address to get the
IPFS address of the journal. This last address references a \emph{JSON} file
containing all the information about the journal.

\vfill{1em}

\sst{Paper page}

A paper page contains useful information about the paper:
\begin{itemize}
\item Title.
\item Authors, with links to each author page.
\item Abstact of the paper.
\item Topics with it's corresponding links.
\item A download link for the paper and links to download each draft, offering
  the possibillity to view the previous work of the researchers.
\item The reviews from the reviewers showing the author, the text of the review
  and the acceptance level. Each review is also a link to the reviewer's page.
\end{itemize}
\figuraHere{paper.png}%
{width=0.9\linewidth}%
{Paper page}%
{poc:p}

The figure~\ref{poc:p} shows the paper's page with all the information mentioned
above.

\sst{Author's page}

The authors page shows all the information about a researcher, the main features
about this page is:

\begin{itemize}
  \itbf{Reputation:} The reputation of a researcher is represented with a 5-star
  rating, depending on how good or bad is as a reviewer. It also contains the
  percent of the reviews delivered in time. \itbf{Ethreum's address:} The
  address of the researcher can be used to verify all the transactions done with
  that address. This offers the possibility to check a researcher career with
  only it's address. \itbf{Papers published:} All the papers published as a
  researcher with the mentioned address. \itbf{Papers reviewed:} All the reviews
  the researcher have done. \itbf{Tags:} The tags the reviewer have participated
  in.
\end{itemize}

\figuraHere{rating.png}%
{width=0.97\linewidth}%
{Author's page}%
{poc:a}

The figure~\ref{poc:a} shows the author's page with all the information
mentioned above.

\sst{Topic Page}

The topic page contains information inside the platform about a specific topic.
This page searches papers published under that topic. It also searchers
reviewers (ordered by reviews, rating or time score) and journals (ordered by
number of papers or average review time) in that topic.

\figuraHere{topic.png}%
{width=0.97\linewidth}%
{Topic's page}%
{poc:t}

\sst{Journal's management page}

The platforms allows addresses with editor privileges to assign reviewers to
pending papers. The page shows the pending, accepted an rejected papers as shown
in the figure~\ref{poc:jmp}.

For each pending papers there is a modal to assign the addresses of the
reviewers as shown in figure~\ref{poc:am}. This will create an Ethereum
transaction to make the assignment public, and allows the reviewers assigned to
accept the review.

\figuraHere{manage.png}%
{width=0.97\linewidth}%
{Journal's management page}%
{poc:jmp}

\figuraHere{assign.png}%
{width=0.97\linewidth}%
{Assign modal from journal's management page}%
{poc:am}


