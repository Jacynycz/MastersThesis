
\section{Conclusiones finales}

Este trabajo propone la descentralización de tres de las funciones del proceso
de revisión por pares: 1) la comunicación del proceso de revisión por pares, 2)
la reputación de los revisores, y 3) la distribución de los artículos y las
revisiones de estos. Ofrece un primer acercamiento a la implementación de la
plataforma con un prototipo funcional.

Las tecnologías distribuidas como Blockchain e IPFS pueden cumplir las promesas
del \ii{Open Access} a la vez que abren nuevos modelos para la distribución de
ciencia.

La descentralización de la estructura mejora la trasparencia y la honestidad del
sistema, y puede proporcionar nuevos escenarios para fomentar la innovación.
También se ha de tener en cuenta que el sistema y los prototipos propuestos no
se basan en el uso de criptomonedas, ya que se centran en un enfoque sin fines
lucrativos, lejos de los enfoques comerciales impulsados ​​por las startups, cada
vez más comunes dentro en el espacio blockchain.

La transparencia proporcionada al abrir el proceso de revisión por pares permite
la construcción de un sistema de reputación de revisores, pero también plantea
problemas sobre la privacidad y la equidad. Además, la introducción de una nueva
métrica pública (la reputación de los revisores) también puede afectar las
carreras de los investigadores, lo que puede suponer un aumento en la presión a
los procesos ya de por sí agotadores para la supervivencia
académica~\cite{de2005publish}.

Las tecnologías de Blockchain se pueden usar para replicar la configuración de
privacidad utilizada actualmente en los procesos de revisión por pares. Sin
embargo, Blockchain también se puede utilizar para presentar un nuevo modelo de
revisión que respalde la responsabilidad de la revisión por pares y al mismo
tiempo mantener el anonimato de las revisiones a ciegas y doble ciego para
mejorar la equidad. Las implicaciones de estos posibles modelos de revisión
responsables, abiertas y anónimas aún están por revelarse.


Además, la infraestructura del sistema se basa en nuevas tecnologías con sus
propios desafíos. Las tecnologías de Blockchain se enfrentan a problemas de
escalabilidad, costos de transacción, inclusión y usabilidad. Por otro lado, los
sistemas de archivos distribuidos como IPFS pueden ser más resistentes, pero aún
necesitan a alguien a cargo de preservar y proporcionar los datos, ya que sin
esa persona responsable, puede provocar la pérdida imprevista de contenido
almacenado por las plataformas que utilicen dicho sistema de archivos.

Otros temas abiertos que pueden explorarse en el trabajo futuro son la
exploración de diferentes modelos de derechos de autor, el desafío de las
métricas tradicionales centradas en revistas para calificar la calidad de la
publicación, diferentes algoritmos de reputación, diferentes niveles de apertura
y la exploración de revistas autónomas descentralizadas, capaces de operar de
manera automática sin interacción de los usuarios.

A pesar de los desafíos existentes, la descentralización de los procesos en los
que se basa la ciencia podría abrir un nuevo campo de exploración, con
implicaciones que posiblemente no podamos prever ahora. ¿Podrán los beneficios
superar a los riesgos?


\section{Trabajo futuro}

Uno de los primeros problemas a atacar sería el anonimato en las revisiones por
pares, porque no importa lo utópico que parezca, un sistema completamente
público en el que no hay anonimato de los revisores y los autores es difícil de
implementar. Este problema podría mitigarse con algunas de las soluciones
propuestas en la sección~\ref{sec:privacyRating}. Sin embargo, el código fuente
de este trabajo se puede migrar a otras tecnologías de blockchain. Por lo tanto,
si en el futuro, existe una tecnología similar a Ethereum, donde los protocolos
de anonimato ya están implementados, una versión de esta plataforma podría
implementarse fácilmente.


Como posibles extensiones de este proyecto se proponen las siguientes ideas:

\begin{itemize}
\item La incorporación de un sistema para enviar documentos ya aceptados,
  tratando de ampliar la investigación que proponen, incluyendo la posibilidad
  de seguir las líneas de investigación de otros autores, completando sus
  trabajos para construir una comunidad científica colaborativa, basada en el
  apoyo mutuo.
\item Una metodología de citación estándar en la que tanto documentos como
  autores son direcciones de Ethereum que reemplazan identificadores como ISBN o
  DOI. Una dirección de autor podría proporcionar información sobre el trabajo
  del autor, ofreciendo la posibilidad de ver los trabajos que envió, los que
  revisó y las colaboraciones realizadas con otros autores. La dirección de un
  artículos podría proporcionar información sobre el factor de impacto que
  tiene, mostrando todos los demás artículos que lo citan.
\item La automatización de la asignación de revisores para un artículo recién
  enviado, de modo que el papel formal del editor desaparezca, ya que la
  elección dependerá completamente del contrato inteligente de la revista a la
  que corresponda. Esta elección podría basarse en la confianza que la comunidad
  tiene en la red de reputación de los revisores, pudiendo eliminar a uno de los
  intermediarios del sistema.
\item Finalmente, el intermediario de las revistas científicas podría
  eliminarse. Si la asignación de revisores y la publicación de los artículos es
  automática y está en el blockchain, la comunidad científica cuestionaría la
  existencia de estas revistas. Con el apoyo de un sistema de reputación
  totalmente funcional y ampliamente utilizado para los revisores, una forma
  ideal de publicación científica podría ser una gran biblioteca en la que los
  autores presenten su trabajo, los revisores se elijan al azar según su
  reputación y los trabajos se publiquen automáticamente.
\end{itemize}

Para concluir este trabajo, propongo un futuro doctorado con nuevas líneas de
desarrollo con mayor impacto en la comunidad. Esta propuesta incluiría: una
exploración exhaustiva de todas las tecnologías de blockchain, entrevistas con
usuarios reales para validar la plataforma, simulación social usando sistemas
multi-agente para analizar el comportamiento de la plataforma y contactar con
usuarios potenciales para implementar demos en entornos reales.

Puede de dentro de unos años los métodos de publicación científica se basen en
proyectos como el que presenta este trabajo, cambiando el paradigma que lleva
tantos años impuesto en el proceso de investigación científica.
