\chapter{Methodology \& Technology}

\begin{FraseCelebre}
  \begin{Frase}
    The needs of the many outweigh the needs of the few
  \end{Frase}
  \begin{Fuente}
    Spock - The Wrath of Khan
  \end{Fuente}
\end{FraseCelebre}

\section{Methodology}
The idea of this project came up in a Hackathon in September 2017. We were a
group of 4 developers with one month to create an idea and a small prototype to
implement using blockchain technologies.

To give birth to the idea of a decentralized publication system for open science
we used agile methodologies.


\subsection{Brainstorming}

Brainstorming was born as a method to increase creativity in groups and
organizations. There are only few rules on this method: do not criticize any of
the given ideas, quantity is desired over quality, try to combine suggested
ideas and give all the ideas that come to mind, no matter if they are possible
or not~\cite{osborn1953applied}.

This method is used nowadays in companies and work groups as part of the process
of the creation of a product, although there are some critics about
brainstorming and sometimes instead of encouragind creativity, inhibits
it~\cite{sutton1996brainstorming,mullen1991productivity}

Leaving apart these problems, we decided to make a brainstorming session to
define what we were going to do. Many ideas emerged and were capture into a
white board without discrimination, no matter how hard or easy to implement they
were.

After saying enough ideas to fill the board we filtered the ones that were
impossible to achieve. Then, each one voted the best three, making a ranking of
the 3-4 best projects to start working on. Some of the ideas were creating a
distributed \emp{wikipedia} with governance models, an application to contact
people from minority groups in countries where they are persecuted collectives,
a distributed and community driven NGO, and a crowdfunding platform for
\emph{whisteblowers}.

Finally we decided to create an approach to a distributed platform for open
science.

\subsection{Value proposition canvas}

\figura{vpc.jpg}{width=0.9\linewidth}{vpcimage}{Image of the value proposition
  canvas after the session}

A value proposition canvas is a tool to create, design and implement a product
idea. Is commonly used by businesses and entrepreneurs to find the balance
between customer profile and product design, but there are other cases of use
for this tool outside business
scope~\cite{pokorna2015value,meertens2012mapping}.

The process is divided in two parts, customer profile and value map, each of
these divided in other three parts:~\cite{osterwalder2014value}:

\begin{itemize}
\item \textbf{Customer profile:} This step is to identify the profile of the
  final user of the platform. This section is divided in three parts: 1)
  \emph{Customer jobs:} things the customer are trying to get done, 2)
  \emph{Customer pains:} undesired costs and situations, 3) \emph{Customer
    gains:} benefits, social gains and cost savings expected.
\item \textbf{Value Map:} This section is about what the final product has to
  have and what does not, and its also divided in: 1) \emph{Product and
    services:} which products and services are offered that help the customer
  get a job done, 2) \emph{Pain relievers:} how the customer pains are going to
  be alleviated, 3) \emph{Gain creators:} how the products and services create
  customer gains
\end{itemize}

We decided to use this methodology for the definition of the final platform,
since it established the general development framework of the application.

\subsection{Agile methodologies} \TODO{REPASAR}

El gran crecimiento de internet y la economía digital ha alterado el concepto de
ingeniería del software. Las metodologías de desarrollo de software (\emph{MDS})
tradicionales están siendo eclipsadas por nuevas \emph{MDS} ligeras o \emph{MDS}
ágiles. Estas metodologías (en adelante \emph{MDSAs}) están caracterizadas por
la integración continua, desarrollo iterativo y la capacidad de asumir cambios
en los requerimientos de negocio.\cite{boehm2005management,livermore2008factors}

La \emph{MDSAs} más popular es conocida como programación extrema o
\emph{Extreme Programming (XP)}\cite{lindstrom2004extreme} basada en una serie
de conceptos básicos a la hora de realizar el desarrollo de un programa:
simplicidad del código y prototipado rápido, comunicación continua del cliente
con el equipo de desarrollo, responsabilidad del código de todos los integrantes
del grupo, reuniones cortas y rápidas, refactorización e integracción
continua.\cite{theunissen2005search,livermore2008factors}

Otros métodos dentro de las \emph{MDSAs} son scrum\cite{rising2000scrum},
métodos cristalinos\cite{cockburn2004crystal} y desarrollo basado en
funcionalidades (en inglés \emph{FDD})\cite{coad1999java}. El uso de estas
\emph{MDSAs} permite a los desarrolladores crear software de mejor calidad en
periodos de tiempo más cortos. Estas metodologías han sido desarrolladas para
mejorar el proceso de desarrollo, quitando las barreras a la aceptación de
cambios en los requerimientos del cliente, un hecho que se da con bastante
frecuencia\cite{lindstrom2004extreme}.

\section{Technology}

To face the challenges proposed by this project, there are many possibilities to
distribute both the data and the information about the reputation network. To
build a robust and decentralized system i

% -------------------------------------------------------------------
\subsection{IPFS}
% -------------------------------------------------------------------
\label{tech:sec:ipfs}
IPFS stands for Interplanetary File System. It is a peer-to-peer file-sharing
protocol that uses a cryptographic hashes to store files in a distributed
network. IPFS works very similar to HTTP protocol but in a BitTorrent way. It
can be seen as a giant git repository where everyone can store, share and
exchange files\cite{benet2014ipfs}.

IPFS merges three main ideas: Distributed Hash Tables, BitTorrent, Git and
Self-Certified Systems.

\subsubsection{Distributed Hash Tables}
A distributed hash table(\emph{DHT}) is a decentralized structure that works
very similar to a hash table. Hash tables are used to identify items in a
database. The table performs simple mathematical operations generating a random
string called hash. The hash acts as a pointer that directs to the data, this
allows the user to find data directly instead of looking through the entire
database\cite{kaluszka2010distributed}.

In a distributed hash table, any node can use a hash as a key to retrieve data.
This system includes a data structure called ``keyspace'' that is a set of all
possible keys, which is split up across the nodes in the system. The mapping of
the keys is made by another function that describes the distance from one key to
another. All the nodes have and identifier and a set of identifiers pointing to
all its neighbors nodes. If a node is removed from the network, only a small
portion of the data must be recovered by other
nodes\cite{kaluszka2010distributed}.

This system makes \emph{DHTs} scalable, fast and robust. It is used by
frameworks such as Tapestry \cite{zhao2004tapestry}, Chord
\cite{stoica2001chord}, Kelips \cite{gupta2003kelips}, Kademlia
\cite{maymounkov2002kademlia} and IPFS \cite{benet2014ipfs}. These platforms are
similar in cost and performance if they are tested in a large enough network.
They behave very fast when it comes to searching for a key through massive
networks of nodes\cite{li2004comparing}, that's why it is used by IPFS to create
its distributed file system.

\subsubsection{BitTorrent - File sharing}
BitTorrent \cite{cohen2003incentives} is a P2P file sharing system used
worldwide. In this system, files are divides into very small chucks of data, and
are shared in a peer-to-peer network. Each peer aims to maximize its download
rate by connecting to low latency peers. In BitTorrent's network, peers with
high upload rate will get higher download rate, so the key is balancing the
network bandwidth between downloading and uploading
files\cite{pouwelse2005bittorrent}.

IPFS uses three main features from BitTorrent's protocol\cite{benet2014ipfs}:
\begin{itemize}
\item BitTorrent's data exchange protocol rewards nodes who contribute to the
  network, and punishes the ones who don't.
\item BitTorrent tracks the availability of file chunks, sending the rarest
  first rather than sending the most common ones.
\item IPFS uses PropShare\cite{levin2008bittorrent} bandwidth allocation
  strategy to improve BitTorrent's behavior facing exploitable scenarios.
\end{itemize}

\subsubsection{Git - Version control system}
Git is a distributed version control system (\emph{DVCS})\cite{torvalds2010git}.
Git was born in 2005 when the development process of the Linux kernel lost its
version control system. The Linux kernel is one of the biggest free software
projects nowadays, it has a great team of developers behind and the code usually
changes very frequently. In 2002 the team used BitKeeper as VCS since they had a
free license. But in 2005 when this license was over, Linus Torvalds decided to
develop his own VCS\cite{spinellis2005version}.

Git was designed to be scalable and distributed, and the most important factors
that IPFS inherits from Git are: \cite{benet2014ipfs}:

\begin{itemize}
\item Git implements a Merkle Directed Acyclic Graph
  \cite{bleichenbacher1994directed}, an object that reflects changes in a file
  system in a distributed way.
\item Objects are identified by the cryptographic hash of their contents.
\item Version changes only update preferences and add objects. To broadcast
  version changes, git only needs to transfer the new objects and update the
  remote references.
\end{itemize}

\subsubsection{Self-Certified File Systems}


\section{Ethereum}

\figura{cryptomap.png}{width=0.9\linewidth}{criptomap}{Mapa de las criptos}

Ethereum~\cite{buterin2014ethereum} is a very novel technology that allows the
creation of distributed applications that run in an arbitrary large and
trust-less network of nodes. Ethereum is based on
Blockchain, the technology behind Bitcoin~\cite{nakamoto2008bitcoin}, and
facilitates the creation of Bitcoin 2.0 applications, solving most of the
technical issues inherent to decentralized platforms, like the one we introduce
in this work. The use of this technology provides significant advantages for our
application, as it does not require a central organization to manage the
projects and bets. In particular, users don’t need to trust in a third party
with their funds, and the network is resistant to node failures and attacks.

%%% Local Variables:
%%% mode: latex
%%% TeX-master: "../Tesis.tex"
%%% End:
