\chapter{Platform description}
\label{cha:platform-description}
\begin{FraseCelebre}
  \begin{Frase}
    Try not. Do, or do not. There is no try.
  \end{Frase}
  \begin{Fuente}
    Yoda - The Empire Strikes Back
  \end{Fuente}
\end{FraseCelebre}
\section{Design using a Decentralized Infrastructure}
\label{archAndDes}
Decentralized Science is a blockchain-enabled publication system for open
science. The system provides a platform for the peer review process
communications, from paper submission to paper acceptance/rejection, and
supports the rating of peer reviews to build a reviewer reputation network.

The proposed system relies on the technologies mentioned in section~\ref{tech}.
The \emph{Ethereum blockchain} provides a public decentralized ledger to record
the system's interactions. Smart contracts are used to enforce the rules of the
system, such as just accepting reviews of invited reviewers. On the other hand,
\emph{IPFS} provides a distributed content-addressable file system to store the
content of the peer review process, from the first submitted paper to each of
the reviews. This ensures that the information registered in the platform will
be persistent, free and accessible, and will not rely on a centralized server.

\figura{Uml.png}{width=0.9\linewidth}{InteractionDiagram}{Sequence diagram of
  platform interaction}

The sequence diagram of the system (Figure \ref{InteractionDiagram}) describes
the main interactions of the platform featuring the publication process of a
paper. Below we proceed to describe these interactions and the basic ideas to
implement them.


\sst{Decentralized Journal}
\label{cha:platform-description-6}

The platform is designed to fit current publication systems, offering journals
to create a \ii{Decentralized Journal}, a smart contract (see
section~\ref{tb:cryptos:sm}) used to record all interaction users make with the
platform.

\emph{Decentralized Journals} have a title, a description, a list of fields of
interest and an image. This information is shown in the front page of each
journal and it is stored within the file system. Each \ii{decentralized journal}
has an owner capable of editing the journal information, giving account
privileges, assigning reviewers and changing the journal's owner.

\sst{Account privileges}
\label{cha:platform-description-5}

This platform allows journals give ``editor'' privileges to desired users. These
are able to assign reviewers for each paper submitted to the journal. In order
to do so, editors have access to the ``journal management page'' in which they
can view: papers pending of reviewers assignment, papers pending of review,
papers published and papers rejected.

\sst{Submit paper}
\label{cha:platform-description-4}

Users may submit a paper to a journal inside Decentralized Science. First, the
platform asks for the submission information such as the authors, an abstract
and the file containing the paper. Next, this information is uploaded to the
file system and assigned to the journal. Once the submission is complete, the
paper remains pending until an editor assigns the reviewers. In this point, the
paper acts as a preprint (see section~\ref{soa:aps}), remaining visible for all
users until it is reviewed, modifying the paper's state from pending to
published or rejected.

\sst{Assign review task}
\label{cha:platform-description-3}

Once a reviewer is proposed for a pending paper, she has the possibility to
accept or reject the review task. This acceptance has an time limit in which she
can communicate her decision. If the task is accepted, the reviewer has a
deadline to submit the review. In the other hand, if the task is rejected, the
editor can assign another reviewer.

\sst{Submit review}
\label{cha:platform-description-2}

Reviewers can submit one reviews for each paper they have been assigned to. This
process is similar to the paper's submission as it is uploaded to the file
system and then it is assigned to the corresponding paper. Each review
submission has a time limit established by each journal. This time limit is
notified to the reviewers before the accept the review task. In the event of a
reviewer sending a review when the time has expired, a penalty is applied to the
reviewer's reputation inside Decentralized Science's reputation system. Finally,
if all the reviews are favorable, the journal's source code automatically
publishes the paper, featuring continuous publication (see
section~\ref{soa:aps}).

\sst{Rate a review}
\label{cha:platform-description-1}

As a novelty in Decentralized Science, there is the possibility to rate the
reviews of a paper. This rating can only be done by the authors, the reviewers
or the editors of the journal the paper have been submitted to. Each rating
consists in an numeric value (1 or 0) representing if the rater is thinks the
review is good or not.

When deciding whether a review is good or not there are several points of
view~\cite{daniel1993guardians,cole1979fair} and not everyone agrees. For this
purpose, a community guideline is provided to explain what is considered a good
review. This guidelines are created within the scientific community through
academic papers and other sources, and they are always subject to change if the
community requests it. In this way, this could be a solution to make all the
votes in the system as fair as possible.

Subsequently, the ``score'' of a rating affects directly to the reviewer's
reputation in Decentralized Science. For this reason, reputation inside the
system is a value between 1 and 0, meaning that reviewers with reputation closer
to 1 are considered better by the community than the ones with reputation closer
to 0.


\section{Platform's features}

Decentralized Science consists of three main components that decentralize and
try to improve three different aspects related to scientific publication:
\begin{enumerate}

  
\item Scientific papers are traditionally obtained or bought from a centralized
  publisher. We propose a decentralized network to distribute academic works and
  promote free access to science.

  
\item Peer reviewer quality and reliability information is difficult to
  predict~\cite{callaham_relationship_2007}, and it is usually hold private by
  publishers and journals. The system proposes to open this information through
  a decentralized reputation network of peer reviewers over a blockchain.


\item Peer review governance communication is traditionally centralized and
  controlled by editors and publishers. Our proposal opens and decentralizes
  these communications making the process more transparent.

\end{enumerate}

These ideas are further discussed in the following sections.

\subsection{Open Access By-Design}
\label{sec:open-access-design}

Open Access focuses on the free access to scientific knowledge. While publishers
provide free of charge their Open Access content, their control of the science
dissemination infrastructure allows them to impose certain rules, such as
charging authors unreasonable fees to offer their work as Open Access (Gold Open
Access)~\cite{solomon2012study} or the temporal embargo and restrictions on the
dissemination of the final version (Green Open access)~\cite{bjork2014anatomy},
among others.

This system proposes a decentralized infrastructure for science publication.
Academic documents -from first drafts to final versions, including peer reviews-
are shared through IPFS, an open P2P network~\cite{benet_ipfs-content_2014}
described in the previous sections. Thus, the system inherently grants Open
Access by-design of its distributed infrastructure and circumvents the
publishers dominant role. Moreover, the access to all those documents does not
depend on the existence of this platform. Even if it ceases to exists, the
documents could still be retrieved from the network.

\subsection{A Distributed Reviewer Reputation System}
\label{sec:distr-revi-reput}

The information concerning the quality and reliability of reviewers is usually
held private by publishers and journals (and even editors). There is no easy way
to predict the quality of a reviewer from factors such as training and
experience~\cite{callaham_relationship_2007}. Although this information is
valuable, it is kept private, reinforcing the publishers and journals
influential positions.

This project extends traditional peer review communication workflow with the
possibility of rating peer reviews, i.e. building a reputation system for
reviewers~\cite{resnick2000reputation}. Reviewers get rewarded for worthy, fair,
and timely reviews, or penalized otherwise.
    
This open reputation network of reviewers could increase the visibility and
recognition of the reviewing work~\cite{rajpert-de_meyts_rewarding_2016}, as
proposed at~\cite{peere2018}.

In fact, such incentives could even be monetary, using
crypto-currencies~\cite{Jan:2018:SDC:3184558.3191556}.

In addition, creating a public reputation network for reviewers reduces, or at
least exposes, unfair and biased
reviews~\cite{wenneras2001nepotism,ReinventingSocioTech}.

As seen in the previous section, in this approach the only ones allowed to send
a rating are the authors of the paper, the editors who have assigned the
reviewers, and the other reviewers of the article. Nevertheless, in future
implementations, rating a review could be public.

There are two main implications among others:

\begin{itemize}
  \itbf{The cold start problem for new papers and researchers:} Normally papers
  that are published in journals or conferences of low impact take a long time
  to gain visibility in the scientific community. However, if a reviewer
  reputation system is used, if three reviewers with a high reputation make a
  favorable review for a paper for a little-known journal, it is likely to have
  more visibility since the reputation of the reviewers can positively influence
  the impact of the papers in the scientific community.
  
  \itbf{Payed reviews:} In most cases nowadays, the reviewers do not obtain any
  economic benefit for carrying out the revision of a paper. By implementing a
  reputation system, payed reviews dynamics could be formed, in which the most
  reputable reviewers are paid for performing reviews at the request of the
  journal or the authors. This is an incentive for all reviewers of the system
  to make good reviews and gain reputation. In addition, since all interactions
  are in Ethereum, payments and deliveries confirmations can be made through
  personalized smart contracts, eliminating intermediaries, and making the
  process transparent and honest.
 
\end{itemize}

\subsection{Transparent Governance}
\label{sec:transp-govern}

The peer review process is nowadays digitally supported, and yet some argue that
the system remains feudal~\cite{ReinventingRigor}. There are multiple proposals
to improve peer review~\cite{walker_emerging_2015}, and yet its communications
and processes remain closed and in control of journals and publishers, and thus
depend on their infrastructure~\cite{ReinventingSocioTech}.

The proposed system aims to improve the peer review process transparency, speed,
and fairness. In order to do that, the system proposes to support the peer
review interactions in an open and decentralized network. It registers, in a
public decentralized ledger, the following parts of the publication process:
paper submission, reviewers assignment, review submission and paper publication.
Thus, processes like the selection of reviewers, or the contents of the reviews,
are open to the public eye. With the interactions being time-stamped and
tamper-proof thanks to the blockchain technology, they can be monitored,
audited, and held accountable. Besides, more complex iterations of the system
may consider blind reviews, as discussed in Section \ref{sec:privacyRating}.

Opening the peer review process communications to the public could even change
the acceptance dynamics within the system. Currently, high rejection ratios are
encouraged because the risk of rejecting a relevant paper is negligible, while
the acceptance of not so relevant content is
penalized~\cite{ReinventingRigor,garfield2007evolution}. However, within a more
transparent system, the first may be penalized as well.

This transparency, combined with a distributed infrastructure for peer review,
facilitates the exploration of new workflows~\cite{ReinventingSocioTech}.


%%%
%%% Local Variables:
%%% mode: latex
%%% TeX-master: "../Tesis.tex"
%%% End:
