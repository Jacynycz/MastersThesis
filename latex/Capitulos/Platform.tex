\chapter{Platform description}

\begin{FraseCelebre}
  \begin{Frase}
    Try not. Do, or do not. There is no try.
  \end{Frase}
  \begin{Fuente}
    Yoda - The Empire Strikes Back
  \end{Fuente}
\end{FraseCelebre}

We propose a blockchain-enabled decentralized publication system for open
science. It consists of three main components that decentralize and try to
improve three different aspects related to scientific publication:

1) Peer review governance communication is traditionally centralized and
controlled by editors and publishers. Our proposal opens and decentralizes these
communications making the process more transparent.

2) Peer reviewer quality and reliability information is difficult to
predict~\cite{callaham_relationship_2007}, and it is usually hold private by
publishers and journals. The system proposes to open this information through a
decentralized reputation network of peer reviewers over a blockchain.

3) Scientific papers are traditionally obtained or bought from a centralized
publisher. We propose a decentralized network to distribute academic works and
promote free access to science.

These ideas are further discussed in the following sections.

% \section{Decentralized publication system for open science}

% We propose a blockchain-enabled decentralized publication system for open
% science. It consist of three components that decentralizes three different
% aspects of science publication.

% 1) Peer review governance communication is traditionally centralized and
% controlled by editors and publisher. Subsection \ref{workflow} further explore
% the opening and decentralization of this communication with blockchain
% technology.

% 2) Peer reviewer quality and reliability information is difficult to
% predict~\cite{callaham_relationship_2007}, and is usually hold private by
% publishers and journals. The system propose to distribute and open this
% information though a decentralized reputation network of peer reviewers over
% blockchain. Subsection \ref{reputation} further explore this proposal.

% 3) Open access papers are traditionally obtained from a centralized publisher
% infrastructure. Subsection \ref{distributedOA} explores the decentralization
% of this infrastructure.

\section{Transparent Peer Review Governance}
\label{workflow}

The system provides a platform for the peer review process communication, from
paper submission to paper acceptance or rejection. It registers all the
interactions into a blockchain based distributed
ledger. % and provides open access to all relevant content, from the different versions of the paper to the submitted reviews.

The interaction diagram of the system (Figure \ref{InteractionDiagram})
describes the interactions of the supported peer review governance. Following,
this interactions and their implementation are described.

\figura{Uml.png}{width=0.9\linewidth}{umlseq}{Sequence diagram of platform
  interaction}

\begin{LaTeXdescription}
\item[Paper submission] A paper submission is registered by submitting the IPFS
  address of the paper to an Ethereum contract. Then, the Ethereum sender
  address is recorded as the corresponding author, and the submission is
  timestamped in the blockchain.

\item[Reviewer proposal] A journal editor may invite a peer reviewer to review a
  specific paper. The transaction will record the Ethereum address of the
  reviewer and optionally, a deadline to submit the review.

\item[Reviewer acceptance/rejection] An invited reviewer may accept or reject
  the review of a paper. The response will be recorded into the blockchain.

\item[Submit review] A reviewer should make a transaction to deliver the review.
  The transaction will record the acceptance/rejection and the IPFS address of
  the detailed review.

\item[Rate review] A novelty of the system further discussed in Section
  \ref{reputation} is the rating system for reviews. The transaction will record
  the sender address and the rating as well as the rated review and reviewer
  addresses.
\end{LaTeXdescription}

\section{The Peer Review Reputation network}
\label{reputation}

The system proposes the use of a peer review reputation network were the quality
of peer reviews is rated by the authors, editors and reviewers of the system.
The work extends traditional peer review governance with the possibility of
rating the reviews, building a reputation system for
reviewers~\cite{resnick2000reputation}. Reviewers get rewarded for worthy, fair,
an timely reviews, or penalized otherwise.

This network of peer reviewers would enable a better reviewer selection, a fair
recognition of reviewers work and a protection against unfair reviews for
authors. However, it could also rise privacy concerns for both reviewers and
raters~\cite{van1999effect,schaub2016trustless}. We consider these privacy
issues in section \ref{privacy}.

% With this network of rated peer reviews, different metrics of the peer
% reviewers can be provided by the system. Journals can benefit from this
% network by searching for well rated reviewers that respond on time, authors
% can expect shorter review time and forget about unaccountable bad reviews and
% reviewers can get their work publicly recognized. This approach can suppose a
% privacy problem for both reviewers and raters. We contemplate these problems
% and how to solve them in section \ref{PrivacyReviewRating}.

Cada vez que un usuario de la plataforma envíe una revisión, esta será
puntuable. De esta manera cada revisor irá obteniendo una puntuación de cada una
de las revisiones que envía, construyendo el sistema de reputación de revisores.
Inicialmente, las personas que pueden realizar votaciones son: los autores del
paper, los editores que han asignado a los revisores, y los otros revisores del
artículo.

De esta forma se da crédito a los revisores que realizan una buena revisión y se
penaliza a los que no. Dando paso a varias implicaciones:

\begin{itemize}
  \itbf{El problema del arranque en frío de los papers:} Normalmente los papers
  que son publicados en revistas o conferencias de bajo impacto tardan mucho en
  ganar visibilidad en la comunidad científica. Sin embargo si con un sistema de
  reputacion de revisores, si tres revisores con alta reputación hacen una
  revisión favorable para un paper para un journal poco conocido, es probable
  que tenga más visibilidad ya que la reputación de los revisores puede influir
  positivamente en el impacto de los papers en la comunidad científica.
  \itbf{Revisiones de pago:} En la mayoría de los casos hoy en día, los
  revisores no obtienen ningún beneficio económico por realizar la revisión de
  un paper. Al implementar un sistema de reputación, podrían formarse dinámicas
  de pago por revisión, en las que los revisores con mayor reputación son
  pagados por realizar  revisiónes a petición de la revista o de los autores.
  Esto supone un incetivo para todos los revisores del sistema para realizar
  buenas revisiones y ganar reputación. Además, como todas las interacciones
  están en Ethereum, los pagos y las entregas pueden realizarse a través de
  contratos inteligentes personalizados, eliminando intermediarios, y haciendo
  el proceso transparente y honesto.
\end{itemize}

\section{Distributed Open Access infrastructure}
\label{distributedOA}
% \commant{Copied from PEERE Paper.TODO rewrite}
Open Access focuses in the free access to scientific knowledge. While publishers
provide free of charge their Open Access content, their control of the science
dissemination infrastructure allows them to impose certain rules, such as
charging authors unreasonable fees to offer their work as Open
Access~\cite{solomon2012study} (Gold Open Access) or the temporal embargo and
restrictions on the dissemination of the final version (Green Open
access)~\cite{bjork2014anatomy}, among others. 

The system proposes a decentralized infrastructure for science publication.
Academic documents - from first drafts to final versions, including peer
reviews- are shared in IPFS, as mentioned on section~\ref{tech:sec:ipfs}.
Thus, the system inherently grants Open Access by the design of its distributed
infrastructure and circumvents the publishers dominant role.

All the documents are first uploaded to IPFS, generating a \emph{Base58}
address. Then the system uses this address to make a transaction in Ethereum,
cryptographically signing the contract with the user's address. Other users that
want to access the paper, only have to inspect the blockchain, retrieve the IPFS
address and download it though a local node or a gateway. This makes
Decentralized Science a truly open-access publication system, were all the
process is public, transparent and trustful.




%%%
%%% Local Variables:
%%% mode: latex
%%% TeX-master: "../Tesis.tex"
%%% End:
