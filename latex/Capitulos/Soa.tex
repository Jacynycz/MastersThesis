\chapter{State of the art}

\begin{FraseCelebre}
  \begin{Frase}
    The needs of the many outweigh the needs of the few
  \end{Frase}
  \begin{Fuente}
    Spock - The Wrath of Khan
  \end{Fuente}
\end{FraseCelebre}

% -------------------------------------------------------------------
% \section{Cooler Section}
% -------------------------------------------------------------------



\section{Alternative Publication systems}

Publication systems, as seen on section \ref{intro} are vampirizing the
industry. However, there are some attempts to change this paradigm on behalf
of science dissemination.

Open journal systems~\cite{willinsky2005open} is an open software designed to
facilitate the publishing process. This project was created by the Public
Knowledge Project\footnote{https://pkp.sfu.ca/about/} and it targets open-access
online journals that want to speed up the publication processes. The system
provides tools to control the whole publishing process from article submission,
through peer reviewing to the final publication issue.

Mega-journals~(or Multi-journals)~\cite{binfield2013open,wellen2013open} combine
multiple journals into a single journal, allowing the publication of open-access
papers, which have gone through a peer review process. The first journal to
adopt this idea was the \emph{PLOS ONE}
Journal\footnote{http://journals.plos.org/plosone/} as of the project
\emph{Public Library of Science}. This project aims to create a library of
scientific journals under the values of open access and creative commons
licenses. As a result of the success of the \emph{PLOS ONE} journal, other
publishers have started their own mega-journals. Featuring alternative impact
metrics, reusability of figures and data, post-publication discussions and
portable reviews from other journals~\cite{bjork2015have}.

The continuous publication model is based on publishing individual papers migrating
from the previous issue-based model~\cite{anderton2013continuous}. This method
is seen as an altenative for open-access journals as it speeds up the
publication process~\cite{haymanview}. \emph{DPSOS}\footnote{Decentralized
  Publication System for Open Science} adopts this model by design (see
section~\ref{tech:sec:ethereum:sm}) as it publish automatically papers that
meet certain preconditions that are written in the blockchain.

Preprints are scientific papers that have not yet gone through the peer review
process~\cite{harnad2003electronic}. Formerly, the preprints that were sent to
the journals were private, and only accessible by the editors and assigned
reviewers. But nowadays it is common to publish a preprint before sending it to
a journal, uploading it to specialized platforms like arXiv\footnote{https://arxiv.org/} or Preprints\footnote{https://www.preprints.org/}~\cite{brown2001volution}. In fact there is
a correlation between the upload of a preprint
and early citations after the publication of the
paper~\cite{shuai2012scientific}. This system is a possible solution to the
cold-start problem that papers of new researchers who enter the
academic career have~\cite{sugiyama2010scholarly}.

Social networks have also made a dent in the academic world, creating platforms
to contact other researchers and encouraging them to share open access papers.
Some of the well-known are Research Gate\footnote{https://www.researchgate.net},
Mendeley\footnote{https://www.mendeley.com} or Academia\footnote{http://academia.edu}. But despite the good intentions of the creators
of these platforms, many of the journals demand the copyright of the papers they
publish, preventing the authors from sharing them through these services.


Decentralized alternatives, in spite of their promises~\cite{bartlingblockchain},
are still in their infancy. A few proposals, none of them functional to date,
have appeared recently.

One of them is a peer review proposal that tries to solve some of the peer review socio-technical problems
using cryptocurrencies~\cite{tennant2017multi}. It needs a critical threshold of
research community engagement, changing the actual processes and platforms, to
start being implemented.

Blockchain-enabled apps have also been proposed, with voting and storage of
publications. This is the case of Aletheia~\cite{morton2017aletheia}, a software
for getting open access papers published. This platform idea aims to use
blockchain as a decentralized and distributed database as a publishing platform.

Peer review quality control through blockchain-based cohort
trainings~\cite{dhillon2016bench} have been also proposed, with the promise of
transparency and decentralization using a distributed ledger. Research labs can
use this training network to test their technology and reduce the risk for
private investment opportunities.

Finally, some of the off-chain journals are adapting to the demands of the
current scientific community like Ledger\footnote{https://ledgerjournal.org}, a
cryptocurrencies and blockchain-based journal that records the publication
timestamps in the Bitcoin blockchain.

\section{Reputation systems}

Reputation systems today arise from the need to trust unknown
individuals~\cite{resnick2000reputation}. Many of the big internet communities
like Stackexchange\footnote{https://stackexchange.com/} or
reddit\footnote{https://www.reddit.com/} have their own reputation system.
Reputation systems behavior may vary depending on the
platform~\cite{josang2002beta}, but the most usual is the one where users get a score
based on certain interaction with the community.

Reputation systems also have a very large niche in e-commerce webs such as
Ebay\footnote{https://www.ebay.com}, in which people pay for a product sold by
an unknown vendor. There must be a previous trust in the vendor before buying
any product, so a reputation system offers a score given by other users that encourages you to trust
or not that certain seller~\cite{resnick2002trust}.

Reputation systems vary widely in scope, such as one for peer-to-peer
computing~\cite{zhou2007powertrust}, vehicle ad-hoc~\cite{dotzer2005vars} and
even Wikipedia~\cite{adler2007content}. All of them are based on an exchange
of trust between users who use these services.

This same concept was intended to be transferred within the blokchain using a
token as a trust unit, which users exchanged as a sign of trust
deposits among them~\cite{sharples2016blockchain}.


This paper proposes the development of a decentralized publication system for
open science. It aims to challenge the technical infrastructure that supports
the middlemen role of traditional publishers. Due to the successes of the Open
Access movement, some of the scientific knowledge is today freely provided by
the publishers. However, the content is still mostly served from their
infrastructure (i.e. servers, web platforms). This ownership of the
infrastructure gives them a position of power over the scientific community which
produces the contents~\cite{fuster2010governance}. Such central and
oligopolistic position in science dissemination allows them to impose policies
(e.g. copyright ownership, Open Access prices) and concentrate profits.

The proposed system aims to move the infrastructure control from the publishers
to the scientific community. It entails the decentralization of three essential
functions of science dissemination: 1) the peer review process, 2) the selection
and recognition of peer reviewers, and 3) the distribution of scientific
knowledge. The following section provides an overview of the system features,
while the final section discusses its challenges.
%%%
%%% Local Variables:
%%% mode: latex
%%% TeX-master: "../Tesis.tex"
%%% End:
