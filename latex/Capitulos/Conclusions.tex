\chapter{Conclusions and future work}

\begin{FraseCelebre}
  \begin{Frase}
    We don't want to change. Every change is a menace to stability.
  \end{Frase}
  \begin{Fuente}
    Aldous Huxley - Brave New World
  \end{Fuente}
\end{FraseCelebre}

Decentralizing and opening academic publishing and peer review infrastructure
enable many possibilities to foster the Open Access and Open Evaluation vision
of a free and fair science publication. The paper has introduced the
decentralization of three essential functions of science dissemination: 1) the
peer review process, 2) the selection and recognition of peer reviewers, and 3)
the distribution of scientific knowledge.

The transparency provided by opening the peer review allows the construction of
a reputation system of reviewers, but also rise concerns about privacy and
fairness. The paper studies the different privacy settings in the peer review
process and the reputation network.

Blockchain technology enables the introduction of a new review model that
support the accountability of peer reviewing proposed by open peer review models
while keeping the anonymity of blind and double blind reviews to improve
fairness.

This challenging proposal raise many issues. The implications of such
accountable, open and anonymous review models are still to be revealed.
Moreover, a deeper study of the strategies to grant anonymity to different
actors is also to be done. Nevertheless, we believe the proposal opens a debate
worth having.

This paper proposes the opening and decentralization of three of the peer review
and publication functions: 1) the peer review process communication, 2) the
reputation of reviewers, and 3) the distribution of papers and peer reviews.
Arguably, this decentralization of the infrastructure could help to challenge
the central role of middlemen such as traditional publishers.

Distributed technologies such as Blockchain and IPFS may finally realize the
promise of Open Access, while enabling new not-for-profit models of science
dissemination. Opening and decentralizing the infrastructure enhances the
transparency and accountability of the system, and fosters innovation.

Note the proposed system does not rely on the use of cryptocurrencies, since it
is focused on a not-for-profit approach, far from the startup-driven commercial
approaches common in the blockchain space.

% blind: More complex iterations of the system can consider embargo periods or
% exceptions to facilitate blind reviews (technically feasible, as mentioned in
% section \ref{conc}).
This challenging proposal raises multiple issues. The opening of the peer review
process may reduce the privacy of current closed system. Blind review relies in
such privacy, and a lack of this protection can cause a great rejection by the
community. Recent technical cryptographic innovations may be used to circumvent
this issue~\cite{blum1988non} and allow transparency while still allowing double
blind reviews.

The introduction of a new public metric (reviewers' reputation) may also affect
researcher careers, adding pressure to the already straining processes for
academic survival~\cite{de2005publish}.
% \commant{Publish or perish reference here?}

Additionally, the proposed system's infrastructure relies in new technologies
with their own challenges. Blockchain technologies face scalability, transaction
costs, inclusiveness and usability problems that remain open and under
discussion. On the other hand, distributed file systems such as IPFS may be more
resilient, but they still need somebody in charge of preserving and providing
the data, since without that responsible actor, it may result in unpredictable
loss of content.

Other open issues that may be explored in future work are the exploration of
different copyright regimes, the challenging of traditional journal-centered
metrics to rate publication quality, different reputation algorithms, different
levels of openness, and the exploration of decentralized autonomous journals.

Despite the existing challenges, we are confident that decentralizing the
processes that Science relies on, would open up a whole new playing field, with
implications we cannot possibly foresee now. Will its benefits outweigh its
risks? We believe it is a conversation worth having.

\section{Future Work}

One of the first problems to attack would be the anonymity in the peer reviews, because no matter how utopian it may seem, a completely public system in which there is no anonymity of the reviewers and the authors is quite difficult to implement. This problem could be mitigated with some of the solutions proposed in the section \label{sec:privacyReview}, even being able to create an internal blockchain for all universities in the world where anonymity protocols are already implemented.

As a result of this work, new development lines for a much larger project with a high impact were established, making contact with other projects around the world and creating a working group called ``Open Science Ecosystem''.

As extensions to Decentralized Science, the following ideas are proposed:

\begin{itemize}
  \item The incorporation of a system to forward papers already accepted, trying to expand the research they propose, including the possibility of being able to follow the research lines of other authors, completing their papers to build a collaborative scientific community, based on mutual help.
  \item The automation of the assignment of reviewers for a newly submitted article, so that the formal paper of the editor disappears, since the choice will depend entirely on the smart contract of the journal to which it corresponds. This choice is based on the trust that the community has in the reviewers' reputation network, being able to eliminate one of the intermediaries from the system.
  \item Finally, the intermediary of the journals should be eliminated, since, if the assignment of reviewers and publication of the articles is automatic and is in the blockchain, the existence of the journal would be questioned by the scientific community, and could convert the dissemination of science into a large library in which authors publish their papers and are reviewed by random reviewers who choose the system.
\end{itemize}


%%%
%%% Local Variables:
%%% mode: latex
%%% TeX-master: "../Tesis.tex"
%%% End:
