\chapter{Conclusions and future work}

\begin{FraseCelebre}
  \begin{Frase}
    We don't want to change. Every change is a menace to stability.
  \end{Frase}
  \begin{Fuente}
    Aldous Huxley - Brave New World
  \end{Fuente}
\end{FraseCelebre}

This work proposes the opening and decentralization of three of the peer review
and publication functions: 1) the peer review process communication, 2) the
reputation of reviewers, and 3) the distribution of papers and peer reviews. It
also offers a first approach to the platform's implementation with a functional
prototype.

Distributed technologies such as Blockchain and IPFS may finally realize the
promise of Open Access, while enabling new models of science dissemination.

Opening and decentralizing the infrastructure enhances the transparency and
accountability of the system, and may provide a new arena to foster innovation.
Note the proposed system and prototypes does not rely on the use of
cryptocurrencies, since it is focused on a not-for-profit approach, far from the
startup-driven commercial approaches common in the blockchain space.

The transparency provided by opening the peer review process allows the
construction of a reputation system of reviewers, but also raises concerns about
privacy and fairness. Besides, the introduction of a new public metric
(reviewers' reputation) may also affect researcher careers, adding pressure to
the already straining processes for academic survival~\cite{de2005publish}.

Blockchain technologies can be used to replicate the privacy settings currently
used in peer review processes. However, Blockchain can also be used to introduce
a new review model that supports the accountability of peer reviewing while
keeping the anonymity of blind and double blind reviews to improve fairness. The
implications of such accountable, open and anonymous review models are still to
be revealed.

Additionally, the system's infrastructure relies in new technologies with their
own challenges. Blockchain technologies face scalability, transaction costs,
inclusiveness and usability problems that remain open and under discussion. On
the other hand, distributed file systems such as IPFS may be more resilient, but
they still need somebody in charge of preserving and providing the data, since
without that responsible actor, it may result in unpredictable loss of content.

Other open issues that may be explored in future work are the exploration of
different copyright regimes, the challenging of traditional journal-centered
metrics to rate publication quality, different reputation algorithms, different
levels of openness, and the exploration of decentralized autonomous journals.

Despite the existing challenges, decentralizing the processes that Science
relies on could open up a whole new playing field, with implications we cannot
possibly foresee now. Will its benefits outweigh its risks?

\section{Future Work}

One of the first problems to attack would be the anonymity in the peer reviews,
because no matter how utopian it may seem, a completely public system in which
there is no anonymity of the reviewers and the authors is quite difficult to
implement. This problem could be mitigated with some of the solutions proposed
in the section~\ref{sec:privacyRating}. Nevertheless, Decentralized Science's
source code can be migrated to other blockchain technologies. Thus, if in the
future, there a similar technology to Ethereum, where anonymity protocols are
already implemented, a version of this platform could be easily implemented.

As extensions to this work, the following ideas are proposed:

\begin{itemize}
\item The incorporation of a system to forward papers already accepted, trying
  to expand the research they propose, including the possibility of being able
  to follow the research lines of other authors, completing their papers to
  build a collaborative scientific community, based on mutual support.
\item A citation methodology in which both papers and authors are Ethereum's
  addresses replacing identifiers like ISBN or DOI. An author address could give
  information about the author's work, offering the possibility to view the
  papers she sent, the ones she reviewed and collaborations made with other
  authors. A paper's address could give information about the impact factor it
  has, showing all other papers that cite it.
\item The automation of the assignment of reviewers for a newly submitted
  article, so that the formal role of the editor disappears, since the choice
  will depend entirely on the smart contract of the journal to which it
  corresponds. This choice is based on the trust that the community has in the
  reviewers' reputation network, being able to eliminate one of the
  intermediaries from the system.
\item Finally, the intermediary of the journals could be eliminated. If the
  assignment of reviewers and publication of the articles is automatic and is in
  the blockchain, the existence of the journal would be questioned by the
  scientific community. With the support of a fully functional and widely used
  reputation system for reviewers, an ideal way of scientific publishing could
  be a large library in which authors submit their paper, reviewers are randomly
  chosen based on their reputation, and papers are automatically published.
\end{itemize}

To conclude this work, I propose a future PhD with new development lines with
higher impact in the scientific community. This proposal would include: An
exhaustive exploration of all blockchain technologies, interviews with real
users to validate the platform, social simulation using multi-agent systems to
analyze the platform's behavior and reaching potential users to deploy demos and
perform tests.

\section{Conferences and Papers Published}

Tenorio-Fornés, A., Jacynycz, V., Llop, D., Sánchez-Ruiz, A.A., Hassan, S.
\textbf{“A Decentralized Publication System for Open Science using Blockchain
  and IPFS”}. \emph{PEERE International Conference on Peer Review Proceedings.}
Rome (2018)

%%%
%%% Local Variables:
%%% mode: latex
%%% TeX-master: "../Tesis.tex"
%%% End:

