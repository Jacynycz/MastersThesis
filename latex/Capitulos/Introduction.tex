
\chapter{Introduction, objectives and document structure}
\label{intro}
\begin{FraseCelebre}
  \begin{Frase}
    Who Watches the Watchmen?
  \end{Frase}
  \begin{Fuente}
    Watchmen - Alan Moore
  \end{Fuente}
\end{FraseCelebre}

\section{Introduction}

Scientific research nowadays is based on publishing in journals with a high
\bb{impact factor} (\emph{IF})\cite{doi:10.1001/jama.295.1.90}, a researcher's
career can be measured depending on the number of papers published in these
journals. There are different impact factors that determine the quality of a
journal, and one of the most well-known is the Journal Citation Reports
(\emph{JCR}), an indicator that represents, for each indexed journal, the
relation between the number of citable items and the number of citations they
get. This \emph{IF} is calculated every year and it is divided in four quartiles
that determine the raking of a certain journal, meaning that a journal in the
first quartile (Q1) has higher impact factor than one in the second(Q2), third
(Q3) or fourth(Q4) quartiles\footnote{https://jcr.incites.thomsonreuters.com/}.
\emph{JCR} was originally an evolution of the Science Citation Index, born in
1955 \cite{garfield2007evolution} and nowadays managed by a company called
Thomson Reuters\footnote{https://www.thomsonreuters.com}. \emph{ Could the
  scientific community rely on themselves rather than on a private company to
  decide the quality of an academic journal?}

One of the problems in academia is the publishing obsession. Ideally, a research
must achieve publications in indexed journals. This idea causes, in many cases,
to yield to the demands of the reviewers or the editors of a journal,
potentially reducing the originality or novelty of a research
paper~\cite{Frey2003}. In some fields such as marketing, universities are
increasingly pressuring researchers to publish papers in journals with
high-quality journal impact metrics, forcing them to focus their research
projects on generating publishable material~\cite{ortinau2011writing}. \emph{ Is
  there a better way to make a career in academia rather than forcing
  researchers to generate redundant or unoriginal publications?}

Science publication and peer review are build on a paper-based paradigm, with
only a few changes in the last centuries~\cite{spier2002history}. The mentioned
\emph{peer review} is the process to decide if a paper is suitable to be
published. A group of ``experts'' in a certain subject review the paper and
issue a verdict based on whether or not it is published. But this process has
been criticized in several aspects. The reviews are not always entirely
objective, since there are cases of unfavorable reviews due to gender causes,
especially in scientific fields~\cite{wenneras2001nepotism}. The review time of
a paper is usually long, causing the process of academic research to be quite
slow~\cite{huisman2017duration}. The reviews not always ensures the quality of a
paper, and cannot be used to decide if a research is good or
not~\cite{goldbeck1999evidence}. \emph{Would it be possible to find better
  alternatives to improve this process so that it is more honest, fair or fast?}

The benefits of scientific distribution are centralized in a few publishers, nor
the authors, the reviewers or the readers get money from it. The development of
the Internet enabled the proposal of alternatives for science
dissemination~\cite{eysenbach2006citation} and
evaluation~\cite{walker_emerging_2015}. The reduction of distribution costs
enabled a wider access to scientific knowledge, and questioned the role of
traditional publishers~\cite{ReinventingRigor}. Nevertheless, universities
normally have to take charge of the costs to access the papers published in
these journals, paying unfair fees~\cite{bergstrom2004costs}.On the other hand,
Open Access and Open Science movements have successfully reduced the economic
cost of accessing knowledge to readers~\cite{evans2009open}. However, it has not
successfully challenged traditional publishers' business
models~\cite{lariviere2015oligopoly}, who are now combining charging readers and
charging authors~\cite{van2013true}. \emph{Could the scientific community build
  a system to decentralize the benefits of science publication and reward the
  authors and the reviewers for their work?}

Editors who want to assign the review of a paper to a series of reviewers have
to rely on them beforehand. Thus, limiting the spectrum of fields that can be
reviewed to the fields in which those reviewers are experts. The internet offers
the possibility to meet people all around the world, and when it comes to trust
total strangers, it should be a system in which anyone can rely to find trustful
people. Reputation systems are the solution to these problems, since they offer
a good first impression about an unknown person \cite{resnick2000reputation}. If
an editor want to broaden the scope of reviewers with more fields of expertise,
she will need to contact new reviewers. But there is no easy way to predict
reviewer quality from their training and experience factors
\cite{callaham_relationship_2007}. \emph{Could there be a system in which
  reviewers get rated based on their reviews and build up their reputation based
  on good practices and helpful reviews?}

Peer review has suffered multiple criticism, and yet only marginal alternatives
have gathered success~\cite{ware2008peer}. The literature provides multiple
proposals around open peer review~\cite{ford2013defining}, and proposals of
reputation networks for reviewers~\cite{frishauf2009reputation}. In fact, a
start-up, Publons\footnote{https://publons.com/}, provides a platform to
acknowledge reviews and open them up. To sum up, this work aims to solve many of
the problems mentioned before trying to answer the questions asked previously.

\section{Objectives}

\subsection*{Create a decentralized platform for science publishing}

This work proposes the design and development of a platform for open science.
This platform should allow its users to do the following interactions: submit
papers, assign reviewers, submit reviews and rate the reviews. This will be
achieved using decentralized technologies such as Ethereum (a decentralized
ledger where each interaction is recorded in a public blockchain) and IPFS (a
distributed file system).

The platform should be accessible through a web page in a custom server acting
as a bridge between the two technologies mentioned before. Also should give its
users the possibility to download the source code and running it locally, as it
is a fully decentralized platform.

All the information held by the platform will be free and public, granting the
possibility to each user to see all the journals, papers, reviews and ratings.

\subsection*{Create a reputation system for reviewers}

Reviewers rarely get credit for their work. Journals and conferences look for
\emph{volunteers} to review the papers submitted to those, but normally
reviewers remain anonymous and they do not get recognition for their work. To
achieve this recognition, this work proposes a reputation system for reviewers,
in which every review they submit can be rated. This rating builds up a score
for each reviewer determining the \emph{reputation} of a reviewer.

The platform should allow users to send ratings to an specific review. This
rating will be saved and will be used to calculate the score of the reviewer
that sent that particular review.

With this idea, this work pretends to \emph{improve} good and fair reviews, and
avoid bad ones, trying to mitigate the possibility of unfair reviews due to
gender causes, research rivalry or ignorance of a subject.

\subsection*{Analyze the platform}

After developing a functional prototype with which to perform tests, this work
also aims to analyze the behavior of the platform based on the entire process of
scientific publication, from the first paper submission to the final
publication.

Tests will be carried out to calculate price estimates of the entire process,
execution times, resistance to large amounts of information and usability
compared to other platforms.

Finally, comparisons of the results obtained will be shown based on the current
publication process.

%%%%%%%%%%%%%%%%%%%%%%%%%%%%%%%%%%%%%%%%%%%%%%%%%%%%%%%%%%%%%%%%%%%%%%%%%%%%%%%%%%%%
% This work aims to challenge middlemen such as traditional publishers in
% science %
% publication. Particularly, proposes a decentralized publication system for
% open %
% science, allowing 1) paper submissions, 2) assignment of reviewers, 3) peer %
% review and, as a novelty, 4) the rating of peer reviews. With this
% distributed %
% system, we aim to improve the quality and efficiency of reviews and
% knowledge %
% distribution, helping editors, authors, and reviewers: %
% \begin{itemize} %
% \item Editors and journals will be able to find the best peer reviewers in
%   their %
%   fields of interest, and also those that respond quickly. Thus, reducing %
%   time-to-publish and publishing costs. %
% \item Authors will be able to submit papers to time-responsive, free, open %
%   access journals, and forget about slow, unfair and unaccountable anonymous %
%   reviews. %
% \item Reviewers will finally have their work recognized. %
% \end{itemize} %
%                                                                                  %
% We are interested in exploring the following challenges, that could be
%                                                                                  % dealt %
% with our technology: %
% \begin{itemize} %
% \item Reduce time-to-review by rewarding on-time reviewers. %
% \item Measure and prevent sexism, nepotism and other abuses in peer review. %
% \item Develop fully autonomous decentralized journals. %
% \item Explore fully free publication systems for Open Access science, while %
%   enabling innovative business models. %
% \item Explore alternative and open metrics for papers, journals and
%   reviewers. %
% \end{itemize} %
%%%%%%%%%%%%%%%%%%%%%%%%%%%%%%%%%%%%%%%%%%%%%%%%%%%%%%%%%%%%%%%%%%%%%%%%%%%%%%%%%%%%

\section{Document Structure}
In this work there will be the following sections:

\subsubsection*{Part 1: Preface.}
\begin{itemize}
  \itbf{Background and State of the art:} Background about the scope of the
  project and what technologies are trying to change the current publication
  systems and how is affecting the scientific community.

  \itbf{Methodology and Technology:}Methodology followed during the realization
  of this work, and the technologies used to implement the platform's
  architecture.
\end{itemize}
\subsubsection*{Part 2: Decentralized Science.}
\begin{itemize}
  \itbf{Platform description:} Platform general description, featuring its main
  strengths regarding the current platforms and explaining how it works and what
  is the expected behavior if it is widely used in the future.

  \itbf{Architecture:} Technical description of the platform, including the
  front end architecture and the definition of the smart contracts'
  infrastructure, and the process followed to reduce the interaction costs.

  \itbf{Product:} Proof of concept of the platform, how it works and how the
  users can interact with the blockchain and with the p2p file network.

  \itbf{Discussion:} Results obtained after the realization of the work proposed
  in this project, how it will affect the scientific community and how to
  measure the potential impact of the platform.

  \itbf{Conclusion and future work:} Implications of this work in the scientific
  community and the next steps to follow to create an ecosystem of autonomous
  publication systems, without the need of middlemen such as journals or
  editors, and a proposal of a future Ph.D. about this subject.
\end{itemize}


%%%
%%% Local Variables:
%%% mode: latex
%%% TeX-master: "../Tesis.tex"
%%% End:
