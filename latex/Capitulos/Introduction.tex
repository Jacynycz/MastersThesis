\chapter{Introduction}

\begin{FraseCelebre}
  \begin{Frase}
    Who Watches the Watchmen?
  \end{Frase}
  \begin{Fuente}
    Watchmen - Alan Moore
  \end{Fuente}
\end{FraseCelebre}

Scientific research nowadays is based on publications in journals with a high
impact factor\cite{doi:10.1001/jama.295.1.90}, the most well-known is the
Journal Citation Reports (\emph{JCR}). This factor was originally determined by
the Science Citation Index, born in 1955\cite{garfield2007evolution} but
nowadays it is managed by a private company dedicated to benefit from the work
of researchers\cite{toledo2011book}. This poses two important problems when it
comes to entering the world of academic research:

The first is that scientific journals that want to maintain their impact factor,
have to make sure that the articles that come out in their issues have a large
number of citations, so they are always going to look for novel and high-impact
articles. Therefore, the editors of these journals will have a network of
reviewers on which they can trust to review the article. But sometimes these
reviews are not entirely objective, there are many cases of unfavorable reviews
due to gender causes, especially in scientific fields
\cite{wenneras2001nepotism}.

Besides, it is necessary to consider that the time of revision for an article is
excessively long, causing the process of academic investigation being quite slow
\cite{huisman2017duration}.

The second problem is that the benefits of the scientific distribution are
centralized in publication systems, nor the authors, the reviewers or the
readers get money from it. Today, with electronic paper distribution,
universities purchase site licenses for online access to journal contents. This
system implies an additional cost for the universities who want to advance in
their research fields and does not have enough money for it. However, site
licenses are not always disadvantageous. Some journals issued by private
companies and universities adjust their prices to maximize subscriptions
\cite{bergstrom2004costs}. But generally, people who earn money from this
paper-based system only act as an intermediary between the authors, the
reviewers and the readers.

The internet offers the possibility to meet people all around the word, and when
it comes to trust total strangers, you should have a system in which you can
rely to deposit you trust in them. Reputation systems are the solution to this
problems, since they offer you a first good impression of an unknown person
\cite{resnick2000reputation}.

Editors who want to assign the review of a paper to a series of reviewers have
to rely on them beforehand. Thus limiting the spectrum of fields that can be
revised to the fields in which those reviewers are experts. If you want to
broaden the scope of reviewers with more fields of expertise, you need to
contact new reviewers. But there is no easy way to predict reviewer quality from
reviewers training andviktor@jacynycz.es experience factors
\cite{callaham_relationship_2007}, so a rating system of reviewers should be
useful for journals to select the best reviewers. The solution is a reviewer
reputation network, in which reviewers get rated based on their reviews and
build up their reputation based on good practices, and helpful reviews. In this
network, publishers who have to find new reviewers for their papers, do not have
to know them beforehand, since trust is placed in the reputation network instead
of the person itself.

Science publication and peer review are based in a paper-based paradigm, with
few changes in the last centuries~\cite{spier2002history}. Critics to current
science publication and peer review systems include concerns about its
fairness~\cite{wenneras2001nepotism}, quality~\cite{goldbeck1999evidence},
performance~\cite{huisman2017duration}, cost~\cite{bergstrom2004costs}, and
accuracy of its evaluation processes~\cite{doi:10.1001/jama.295.1.90}, among
others.

The development of the Internet enabled the proposal of alternatives for science
dissemination~\cite{eysenbach2006citation} and
evaluation~\cite{walker_emerging_2015}. The reduction of distribution costs
enabled wider access to scientific knowledge, and questioned the role of
traditional publishers~\cite{ReinventingRigor}. It is acknowledged that the Open
Access and Open Science movements have successfully reduced the economic cost of
readers to access knowledge~\cite{evans2009open}. However it has not
successfully challenged traditional publishers business
models~\cite{lariviere2015oligopoly} that are now combining charging readers and
charging authors~\cite{van2013true}.

Peer review has suffered multiple criticism, and yet only marginal alternatives
have gathered success~\cite{ware2008peer}. The literature provides multiple
proposals around open peer review~\cite{ford2013defining}, and proposals of
reputation networks for reviewers~\cite{frishauf2009reputation}. In fact, a
start-up, Publons\footnote{https://publons.com/}, provides a platform to
acknowledge reviews and open them up.

\section{Objective}

We aim to challenge middlemen in science publication such as traditional
publishers. Particularly, we propose a decentralized publication system for open
science, allowing 1) paper submissions, 2) assignment of reviewers, 3) peer
review, and, as a novelty, 4) the rating of peer reviews. With this distributed
system, we aim to improve the quality and efficiency of reviews and knowledge
distribution, helping editors, authors, and reviewers:
\begin{itemize}
\item Editors and journals will be able to find the best peer reviewers in their
  fields of interest, and those that respond quickly. Thus reducing
  time-to-publish and publishing costs.
\item Authors will be able to submit papers to time-responsive, free, open
  access journals, and forget about slow, unfair and unaccountable anonymous
  reviews.
\item Reviewers will finally have their work recognized.
\end{itemize}

We are interested in exploring the following challenges, that could be dealt
with our technology:
\begin{itemize}
\item Reduce time-to-review by rewarding on-time reviewers.
\item Measure and prevent sexism, nepotism and other abuses in peer review.
\item Develop fully autonomous decentralized journals.
\item Explore fully free publication systems for Open Access science, while
  enabling innovative business models.
\item Explore alternative and open metrics for papers, journals and reviewers.
\end{itemize}

  \section{In this work}
  In this work there will be the following sections:
  \begin{itemize}
  \item \textbf{State of the art:} This chapter is about what methods, systems
    and technologies try to change the actual publication systems and how good
    or bad they are.
  \item \textbf{Methodology and Technology:} This chapter is about the
    methodology I followed during the realization of this work, and the
    technologies I have used to face the challenges and why I decided to use
    them.
  \item \textbf{Platform description:} This is the main chapter of this work. It
    contains the platform description, its implementation, how it works, why is
    better than the actual publication systems, the challenges I have faced
    during the realization and the internal structure of the final system.
  \item \textbf{Results and discussion:} This chapter is about the results
    obtained after the realization of the work proposed in this project, how it
    will affect the scientific community and how I measured the potential impact
    if it becomes a wide-used publication system.
  \item \textbf{Conclusion and future work:} This chapter is about the
    implications of this work in the scientific community and settles the next
    steps to follow to create an ecosystem of autonomous publication systems
    without the need of middlemen such as journals or editors, proposing a
    future PhD about this subject.
  \end{itemize}


%%%
%%% Local Variables:
%%% mode: latex
%%% TeX-master: "../Tesis.tex"
%%% End:
