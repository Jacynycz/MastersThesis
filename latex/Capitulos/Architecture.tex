

\chapter{Architecture}
\label{cha:architecture}
\begin{FraseCelebre}
  \begin{Frase}
    There is no secret ingredient. To make something special you just have to
    believe it's special.
  \end{Frase}
  \begin{Fuente}
    Mr. Ping - Kung Fu Panda
  \end{Fuente}
\end{FraseCelebre}

The platform's architecture consists in two main parts as explained in section
\ref{tech}, a decentralized file system in which users can upload all the files
using IPFS (see section \ref{tech:sec:ipfs}), and a smart contact to register
all interactions of the users with the platform (see
section~\ref{tb:cryptos:sm}).

As a decentralized technology, anyone can run a node locally, connecting to the
IPFS network and to the Ethereum's blockchain to interact with the platform, but
this technology is not used commonly, and not all users have the knowledge to
install and run these programs. As a solution, a ``gateway server'' to test the
platform's implementation was created, using a web browser extension called
Metamask\footnote{See section~\ref{jsmm}} to interact with the blockchain and
running an IPFS node to upload the files as explained in the diagram of the
figure \ref{architecture:gw}.

\figura{architecture.png}{width=0.9\linewidth}{architecture:gw}{Architecture
  diagram of a node with IPFS and an Ethereum light client}

\section{First Prototype}
\label{sec:first-prototype}

The first prototype of Decentralized Science was based on ``Structs'' to store
the information inside the blockchain. Structs in Ethereum are similar to
structs in other programming languages such as Java or C , a composite data type
consisting in a group of variables placed in a block of memory. Structs in
Ethereum allows users to ``pack'' information in the contract in a solid and
clear way, facilitating the code clarity and comprehensibility. In addition,
storing information with this data type can suppose an improvement in the
execution costs, making interactions\footnote{Each interaction with the platform
  has a small cost (see section~\ref{ts:at})} with the platform cheaper.

The information of this implementation is stored through an array of structs. In
order to access a particular data (e.g. a paper within the platform), users only
need the index within the array to retrieve it. Listing~\ref{arstruct} shows the
paper struct inside the \ii{PapersLibrary} contract used in the first prototype.

\begin{minipage}{\linewidth}
\begin{lstlisting}[language=C++,commentstyle=\color{olive}\ttfamily,frame=single,caption=Struct used for stroring papers in the
  first prototype,label=arstruct,captionpos=b]
  contract PapersLibrary{

      ...

      struct Paper{
          // IPFS address using Multihash struct
          Multihash _ipfsPaperMultihash;

          // Ethereum's addresses of the authors
          address[] _authors;

          // Check if the paper is acepted
          bool _published;

          // Array of Reviews (see Review Struct) of this paper
          Review[] _reviews; 
        }

        // Papers received are stored in this array
        Paper[] storedPapers;

        ...
    }
  \end{lstlisting}
  \end{minipage}

\subsection*{Smart Contract Architecture}
\label{arch:sca}

Smart contracts in Ethereum can interact with each other, creating an ecosystem
of programs that resemble object oriented programming. The contract structure
and source code is crucial because once a contract is in the blockchain, there
is no way to change it.

As explained in the section~\ref{sec:agile-methodologies}, one of the most
important challenges in designing a contract is to reduce the cost of
transactions, since the cost of these can be very high if the contract design is
inefficient~\cite{croman2016scaling}. Developing different implementations of
the same interaction in a small prototype allows developers to test each
transaction in several different ways. Furthermore, inserting data in the
blockchain is very expensive, so it is advisable to use the data structures that
Ethereum offers to reduce the transaction cost. Section~\ref{arch:trans}
discusses some problems encountered during the development of the platform to
reduce these costs and what solutions have been implemented.

\figura{UML_class_1.png}{width=0.93\linewidth}{contracts:uml}{Diagrama UML del
  la estructura de contratos}

The design of this first approach has been fragmented into 4 smart contracts
that allow to both reduce the cost of transactions, and fragment the information
of the platform, as seen in figure \ref{contracts:uml}. This fragmentation
allows other smart contracts interact with the information stored in
Decentralized Science, without having to rely on a single contract that
centralizes all the transactions, possibly causing a significant number of
transactions to the same address.

\subsection*{Structs implemented in the contracts}
\label{sec:structs-impl-contr}

This implementation uses five different structs to store the information inside
the blockchain:

\begin{itemize}
  \itbf{Multihash struct:} Used to save all IPFS addresses of the platform.
  These addresses act as pointers inside the IPFS network to retrieve all files
  in the platform. It is used to store both papers and reviews in Decentralized
  Science.

  \itbf{Paper struct:} Papers sent to the platform are stored using this struct.
  It contains a Multihash as reference to the file inside IPFS network, the
  Ehtereum's addresses of the authors of the paper and a boolean representing if
  the paper is published. In addition it contains an array of the reviews of the
  paper.

  \itbf{Review:} This struct is used inside the Paper struct to save its
  reviews. In order to do so, it contains a Multihash as a reference to the
  review file inside IPFS, the Ethereum's address of the reviewer and an integer
  number between 1 and 4 showing the level of acceptance/rejection of the
  review.

  \itbf{Rating:} As part of the reputation system mentioned in
  chapter~\ref{cha:platform-description} each review can be rated to give or
  subtract reputation to the reviewer. This prototype uses a Rating struct to
  gather these. It contains an unique string that identifies the rater and the
  reviewer receiving the rating, the rater address and the score of the rating.

  \itbf{Reviewer:} Finally the last struct used in this approach is responsible
  of storing the reviewers. It contains the reviewer's Ethereum address, the
  reputation she or he has, an array representing the fields of expertise and
  the number of reviews done.


  
\end{itemize}

\subsection*{Decentralized Journal}%
\label{sec:decentr-journ}

This is the contract from which all interactions with authors and publishers are
controlled. The authors interact with this contract to send the papers, and the
editors assign reviewers for the peer review process.

This contract has a series of ethereum addresses associated with the editors
(which are the ones who can assign reviewers) and a reference to the contract
address of the ``library contract'', where the papers are sent. In addition,
through this contract, the reviewers send the reviews that have been assigned to
them.

\subsection*{Decentralized Library}
\label{sec:decentr-libr}

This smart contract stores the IPFS addresses of the papers and the reviews and
controls if the papers are published.

To store this data, this contract saves a reference to the IPFS addresses in
\emph{Base58}. This reference is the result of the data transformation is
explained in the section~\ref{arch:trans}.

Then it stores this data in a struct called ``Multihash'' defined in the smart
contract and performs a hash function to obtain an identifier of the file within
the system, which is much more efficient in Ethereum. This hash is the one that
will be used to identify a paper from the other Decentralized Science's
contracts.

In addition, each paper stores all the revisions it has through a ``Review''
struct. This struct has an IPFS address with the review (transformed as a
\emph{Multihash}), the address of the reviewer and an integer representing
whether the paper is accepted or not.

\subsection*{Rating Storage}
\label{sec:rating-storage}

This smart contract stores all the transactions about the ratings of the reviews
made by the reviewers, and it is the one capable to give reputation to each one
of them.

Each ratring is represented by a struct that has: a hash that univocally
identifies the review of a paper made by a reviewer, the address of the person
making the rating and a score that represents the reputation that is given to
the reviewer.

Regarding the reputation system, it was decided to adopt a five-star reputation
system explained in the section \ref{rep:system}.

\subsection*{ReviewersHub}
\label{sec:reviewershub}

It is responsible for storing the addresses, the research fields, and the
reputation of the reviewers who are registered in the platform.

It is responsible for storing the addresses, the research fields, and the
reputation of the reviewers who are registered in the platform. This contract is
used by: the Rating Storage contract to give the reputation scores to the
reviewers, the new reviewers who want to register on the platform and the
editors who want to find new reviewers to carry out the peer review process.

The diagram \ref{contracts: uml} tries to illustrate the structure of the
contracts through a UML, as if the smart contracts of Ethereum were objects.

\subsection*{First prototype problems}
\label{sec:first-prot-probl}

\section{Second prototype}


\section{Ethereum addresses in Decentralized Science}
\label{sec:ether-addr-decentr}

\figura{tx_history.png}{width=0.93\linewidth}{sec:ether-addr-decentr:history}{Researcher's
  transactions list}

As mentioned in section \ref{ts:at}, ethereum address is a string of characters
that identify a user's account or a smart contract.

In this platform users are authors, reviewers, editors or some of these roles
simultaneously. This implies that an ethereum address can be used to identify
and verify an individual's scientific career. Each interaction is registered in
the blockchain as a transaction that have a transaction id, which can be used to
obtain the information of that particular transaction as shown in the figure
\ref{sec:ether-addr-decentr:history}. This could mean that when a researcher
needs to give credit of it's career, he or she will only need to supply his or
her ethereum's address.

\figura{add_citations.png}{width=0.93\linewidth}{sec:ether-addr-decentr:citations}{Use
  of Ethereum's addresses in Decentralized Science}

Ethereum's smart contracts also have an ethereum address.

\section{Reputation system}
\label{rep:system}

Of the many reputation systems discussed in section \ref{soa:rs}, a five-star
rating has been decided for the platform~\cite{kinateder2003architecture}. This
reputation system is present in many online platforms such as
Tripadvisor\footnote{https://www.tripadvisor.com/},
Amazon\footnote{https://www.amazon.com/}, Google
Play\footnote{https://play.google.com/} and other large platforms in which
products and services are voted by users.

Being implemented in the blockchain implies that all interactions are public and
auditable. Everyone can see who has voted which revision and with what score or
rating, so it is a system in which there is no anonymity. This allows dissuading
the problems commented in section \ref{soa:rs} since the users who carry out
unjust directing or rating attacks are exposed publicly in the network. But this
also raises concerns about privacy and anonymity, commented in section
\ref{sec:privacyReview}.

The internal operating mode has several simple steps:
\begin{enumerate}
\item When making a review, a hash is created in the system
  (SHA-3~\cite{aumasson2008sha}) with: the address of the reviewer, the address
  of the paper it reviews and the address of the journal that assigned the
  review. This hash uniquely identifies the review within the system.
  
\item Both authors, editors and the other reviewers of the paper have the
  ability to assign a score from 1 to 5 indicating whether they think the review
  is fair or not.
  
\item For each vote, the system registers the voter and sends him the reputation
     the reviewer, making an exponential smoothing of all the votes that he has
     received~\cite{gardner1985exponential}.
\end{enumerate}

When deciding whether a review is fair or not there are several points of
view~\cite{daniel1993guardians,cole1979fair} and not all people will agree, but
offering guidelines for the community can be an initial solution to make all the
votes in the system as fair as possible.

Within a reputation system there will always be some controversy and there will
always be methods to placate it~\cite{dellarocas2000immunizing}, but the design
presented in this work is a proof of concept that will evolve as new systems are
designed to appease the problems that have arisen.


\section{Functional requirements and activity diagrams}
\label{sec:funct-requ-activ}

In the following section we will discuss the functional requirements and the
activity diagrams for each transaction users can make to interact with the
platform.

Each of these interactions must be done through an Ethereum personal address, as
it works as an identifier of the user performing it. For this reason, all of the
following activity diagrams asks the user for an Ethereum address. In case a new
user uses the platform does not have a personal address, the platform has the
possibility to create one for that user through Metamask\footnote{Used to
  connect a web browser to Ethereum (see section~\ref{jsmm})} extension.
Furthermore, in order to complete each transaction, an essential requirement it
that the user must have an Ethereum account and Metamask installed.

\subsection*{Send paper}

Once an user has an Ethereum account users may send papers. The platform asks
for the authors and for the file through a web form. Javascript collects this
information and uploads it to the file system, generating an IPFS file address.
Finally, the platform creates a transaction with these information and sends
them to the platform.

\figuraAD{AD1}% 1. Table ID
{Send Paper}% 2. Table title
{The user uploads a file to IPFS and creates an Ethereum
  transaction}% 3. Table description
{Authors' addresses and a file}% 4. Inputs
{IPFS address of the paper}% 5. Outputs
{Authors' Ethereum addresses and Metamask installed}% 6. Requirements
{The journal's contract address exists and the IPFS node is
  online}% 7. Preconditions
{There is a transaction to the journal's contract address with the information\\
  & about the authors and the IPFS address of the paper }% 8. Postconditions
{sending a paper}% 9. Caption description

\subsection*{Assign Reviewer}

After receiving a paper, an editor has the possibility to assign reviewers. To
do so, there must be a journal in the platform and the user must have editor
privileges. Next, the editor assigns the reviewers of the paper and the platform
creates the review tasks for each reviewer. These tasks can be accepted or
rejected by the reviewers and they can have a ``deadline'' to receive the
acceptance.

\figuraAD{AD2}% 1. Table ID
{Assign Reviewer}% 2. Table title
{An editor assigns reviewers for a paper}% 3. Table description
{Ethereum's addresses of the reviewers}% 4. Inputs
{_}% 5. Outputs
{The journal's contract address exists} {The address has editor permissions and
  there is a paper pending}% 6. Requirements
{A transaction is created to assign the reviwers}% 7. Preconditions
{assigning a reviewer}% 9. Caption description

\subsection*{Accept review task}

As explained before a reviewer has the possibility to accept or reject a review
task. Before doing so the reviewer must have an Ethereum address. Then, when a
task is received, the reviewer has a time limit to accept it. If it is not
accepted, the editor can assign another reviewer. In the other hand, if it is
accepted, a transaction is created and the reviewer is registered in the paper
to be able to send his review.

\figuraAD{AD3}% 1. Table ID
{Accept Review task}% 2. Table title
{An address authorized as a reviewer accepts the review
  task}% 3. Table description
{A boolean accepting the task}% 4. Inputs
{_}% 5. Outputs
{The journal's contract address exists}% 6. Requirements
{There are a petition pending for the reviewers address}% 7. Preconditions
{The petition is accepted and there is a deadline to submit the
  review}% 8. Postconditions
{accepting a review}% 9. Caption description

\subsection*{Send review}

Sending a review is similar to sending a paper. The main difference is that the
reviewer, once she accepts the review task, has a deadline to submit a review.
If the review is sent before the time limit, the file containing the review is
uploaded to IPFS and the platform created an Ethereum transaction with this
information. Otherwise, if the review is not sent, the reviewer loose reputation
within the reputation system.

% \begin{minipage}{\linewidth}
\figuraAD{AD4}% 1. Table ID
{Send Review}% 2. Table title
{A reviewers submits the review previously accepted via
  IPFS}% 3. Table description
{A file containing the review}% 4. Inputs
{The IPFS address of the review}% 5. Outputs
{The journal's contract address exists}% 6. Requirements
{The submission is within the deadline}% 7. Preconditions
{There is a transaction to the journal's contract address}% 8. Postconditions
{sending a review}% 9. Caption description
% \end{minipage}

\subsection*{Rate review}

As part of the reputation system, all reviews can be rated by the authors and by
other reviewers. When a review is sent to the platform, it generates an
identifier which can be rated with 1 or 0 as explained in
section~\ref{rep:system}.

\figuraAD{AD5}% 1. Table ID
{Rate Review}% 2. Table title
{The authorized address submits a rating about a review}% 3. Table description
{An integer from 1 to 5}% 4. Inputs
{_}% 5. Outputs
{The journal's contract address exists}% 6. Requirements
{The address can submit a review and has not done it yet}% 7. Preconditions
{The reputation of the reviewer is affected by de rating}% 8. Postconditions
{rating a review}% 9. Caption description
% \clearpage
\section{Transaction costs}
\label{arch:trans}

As explained in the section \ref{tech:sec:ethereum} Ethereum's smart contracts
are transaction-based. Each interaction with the platform is registered in the
blockchain with a transaction. The architecture of this platform allows the
following transactions (all the costs are estimated\footnote{Check
  https://ethgasstation.info/ for more info about transaction costs}).

\begin{itemize}
  \itbf{Create the platform:} To create Decentralized Science, there must be an
  initial transaction with the source code to upload it to the blockchain.
  Normally this transaction is the most expensive, and is only necessary to do
  it once.

  This transaction creates all the contracts mentioned in the section
  \ref{arch:sm} and outputs the Ethereum's address of the platform. The gas
  amount required to this transaction is 1696291, approximately from 2 to 4
  euros.

  \itbf{Send Paper:} To send a paper the user have to send a transaction
  containing the Ethereum addresses of the authors and a file containing the
  paper. This file will be uploaded to IPFS through a node and then the
  resulting address will be inserted in the transaction. The gas amount required
  to do this transaction is 114812, around 0.30 euros.

  \itbf{Assign reviewers:} Assigning the reviewers is normally very cheap
  because an address authorized as ``editor'' must create a simple transaction
  with the paper identifier and the addresses of the reviewers. The gas amount
  required to do this transaction is 58707, around 0.10 euros.

  \itbf{Accept review:} This is the cheapest interaction with the platform, a
  reviewer must accept the task of reviewing a paper through a transaction. With
  this system, the reviewer cryptographically signs the contract and compromises
  to review the paper. The gas amount required to do this transaction is 23971,
  around 0.04 euros.

  \itbf{Send Review:} Sending a review is also expensive, because the review is
  also a file and must be uploaded to IPFS. The transaction also contains a
  reference to the paper, and an integer representing the acceptance level of
  the review. The gas amount required to do this transaction is 149760, around
  0.35 euros.

  \itbf{Send Rating:} This transaction contains a reference to the review of a
  paper, and an integer from 1 to 5 representing the rating of a review. This
  transaction initially can only be made by the other reviewers, the authors and
  the editors of the journal. The gas amount required to do this transaction is
  94122, around 0.04 euros.

\end{itemize}

\subsection{Reducing transaction costs}
\label{sec:reduc-trans-costs}

In Ethereum, reducing the transaction costs is critical, because all
interactions have costs to the user. In the Decentralized Science's contract the
following actions have been taken to reduce these costs:

\begin{itemize}
  \itbf{Reduce the amount of data stored in the blockchain:} Storing data in the
  blockchain is slow and expensive, so the platform only registers the minimum
  amount of data necessary to verify the interactions between users. All data
  files are stored in IPFS and the addresses are saved in Ethereum. \itbf{Avoid
    expensive data types:} Working with ``Strings'' in \ii{Solidity} is very
  expensive, that's why it is recommended to avoid using this data type.

  As described in the chapter~\ref{cha:platform-description}. Decentralized
  Science stores IPFS addresses as links between Ethereum and IPFS, but these
  addresses are \ii{Base58} strings, a data type not implemented yet in
  Ethereum. This address has 3 parts: 1) A number representing a hash function,
  2) another number representing the size of the file and 3) a hashed string of
  the data done with the 1) hash function.

  To save an IPFS address in Ethereum, initially a ``String'' data type was
  used. But to avoid using this type, first the \ii{Base58} is decoded into the
  three parts mentioned above. Finally instead of creating a transaction with a
  string representing the IPFS address, the transaction contains the hash ID
  (using an \emph{uint8} data tyoe), the size (using an \emph{uint8} data type)
  and the hashed data (using \emph{bytes32} data type). This allows the platform
  to reduce the transactions involving a file upload around $40\%$.
\end{itemize}

%%% Local Variables:
%%% mode: latex
%%% TeX-master: "../Tesis.tex"
%%% End:
