

\chapter{Architecture}
\label{cha:architecture}
\begin{FraseCelebre}
  \begin{Frase}
    There is no secret ingredient. To make something special you just have to
    believe it's special.
  \end{Frase}
  \begin{Fuente}
    Mr. Ping - Kung Fu Panda
  \end{Fuente}
\end{FraseCelebre}

The platform's architecture consists in two main parts as explained in section
\ref{tech}, a decentralized file system in which users can upload all the files
using IPFS (see section \ref{tech:sec:ipfs}), and a smart contact to register
all interactions of the users with the platform (see
section~\ref{tb:cryptos:sm}).

As a decentralized technology, anyone can run a node locally, connecting to the
IPFS network and to the Ethereum's blockchain to interact with the platform, but
this technology is not used commonly, and not all users have the knowledge to
install and run these programs. As a solution, a ``gateway server'' to test the
platform's implementation was created, using a web browser extension called
Metamask\footnote{See section~\ref{jsmm}} to interact with the blockchain and
running an IPFS node to upload the files as explained in the diagram of the
figure \ref{architecture:gw}.

\figura{architecture.png}{width=0.9\linewidth}{architecture:gw}{Architecture
  diagram of a node with IPFS and an Ethereum light client}

\section{First Prototype}
\label{sec:first-prototype}

The first prototype of Decentralized Science was based on ``Structs'' to store
the information inside the blockchain. Structs in Ethereum are similar to
structs in other programming languages such as Java or C , a composite data type
consisting in a group of variables placed in a block of memory. Structs in
Ethereum allows users to ``pack'' information in the contract in a solid and
clear way, facilitating the code clarity and comprehensibility. In addition,
storing information with this data type can suppose an improvement in the
execution costs, making interactions\footnote{Each interaction with the platform
  has a small cost (see section~\ref{ts:at})} with the platform cheaper.

The information of this implementation is stored through an array of structs. In
order to access a particular data (e.g. a paper within the platform), users only
need the index within the array to retrieve it. Listing~\ref{arstruct} shows the
paper struct inside the \ii{PapersLibrary} contract used in the first prototype.

\begin{minipage}{\linewidth}
\begin{lstlisting}[language=C++,commentstyle=\color{olive}\ttfamily,frame=single,caption=Struct used for stroring papers in the
  first prototype,label=arstruct,captionpos=b]
  contract PapersLibrary{

      ...

      struct Paper{
          // IPFS address using Multihash struct
          Multihash _ipfsPaperMultihash;

          // Ethereum's addresses of the authors
          address[] _authors;

          // Check if the paper is acepted
          bool _published;

          // Array of Reviews (see Review Struct) of this paper
          Review[] _reviews; 
        }

        // Papers received are stored in this array
        Paper[] storedPapers;

        ...
    }
  \end{lstlisting}
  \end{minipage}

\subsection*{Smart Contract Architecture}
\label{arch:sca}

Smart contracts in Ethereum can interact with each other, creating an ecosystem
of programs that resemble object oriented programming. The contract structure
and source code is crucial because once a contract is in the blockchain, there
is no way to change it.

As explained in the section~\ref{sec:agile-methodologies}, one of the most
important challenges in designing a contract is to reduce the cost of
transactions, since the cost of these can be very high if the contract design is
inefficient~\cite{croman2016scaling}. Developing different implementations of
the same interaction in a small prototype allows developers to test each
transaction in several different ways. Furthermore, inserting data in the
blockchain is very expensive, so it is advisable to use the data structures that
Ethereum offers to reduce the transaction cost. Section~\ref{arch:trans}
discusses some problems encountered during the development of the platform to
reduce these costs and what solutions have been implemented.

\figura{UML_class_1.png}{width=0.93\linewidth}{contracts:uml}{First prototype's
  diagram of the contracts structure}

The design of this first approach has been fragmented into 4 smart contracts
that allow to both reduce the cost of transactions, and fragment the information
of the platform, as seen in figure \ref{contracts:uml}. This fragmentation
allows other smart contracts interact with the information stored in
Decentralized Science, without having to rely on a single contract that
centralizes all the transactions, possibly causing a significant number of
transactions to the same address.

\subsection*{Structs description}
\label{sec:structs-impl-contr}

This implementation uses five different structs to store the information inside
the blockchain:

\begin{itemize}
  \itbf{Multihash struct:} Used to save all IPFS addresses of the platform.
  These addresses act as pointers inside the IPFS network to retrieve all files
  in the platform. It is used to store both papers and reviews in Decentralized
  Science.

  \itbf{Paper struct:} Papers sent to the platform are stored using this struct.
  It contains a Multihash as reference to the file inside IPFS network, the
  Ehtereum's addresses of the authors of the paper and a boolean representing if
  the paper is published. In addition it contains an array of the reviews of the
  paper.

  \itbf{Review:} This struct is used inside the Paper struct to save its
  reviews. In order to do so, it contains a Multihash as a reference to the
  review file inside IPFS, the Ethereum's address of the reviewer and an integer
  number between 1 and 4 showing the level of acceptance/rejection of the
  review.

  \itbf{Rating:} As part of the reputation system mentioned in
  chapter~\ref{cha:platform-description} each review can be rated to give or
  subtract reputation to the reviewer. This prototype uses a Rating struct to
  gather these and store them in the platform, containing the ratings and the
  users who submitted it.

  \itbf{Reviewer:} Finally the last struct used in this approach is responsible
  of storing the reviewers. It contains the reviewer's Ethereum address, the
  reputation she has, an array representing the fields of expertise and the
  number of reviews done by her.
  
\end{itemize}

Above these structures, there is a system of smart contracts which interact with
each other to make the platform work. The first Decentralized Science's
prototype consisted of the following contracts:

\subsection*{Decentralized Journal}
\label{sec:decentr-journ}

A Decentralized Journal (see chapter~\ref{cha:platform-description}) is an smart
contract that behaves like a digital journal. It is similar to small program in
blockchain which controls all interactions with authors and editors. The authors
interact with this smart contract to submit papers, and the editors assign
reviewers for the peer review process.

Specifically, this contract has a series of ethereum addresses associated with
the editors (which are the ones who can assign reviewers) and a reference to the
contract address of the ``library contract'', where the papers are stored.

In addition, through this contract, the reviewers send the reviews that have
been assigned to them.

\subsection*{Decentralized Library}
\label{sec:decentr-libr}

This smart contract stores the IPFS addresses of the papers and the reviews and
controls if the papers are published.

To store this data, this contract saves a reference to the IPFS addresses. This
reference is the result of the data transformation is explained in the
section~\ref{arch:trans}, and it is used to identify a paper within
Decentralized Science's platform.

In addition, each paper stores all the revisions it has through a ``Review''
struct. This struct stores the review file, the address of the reviewer and an
integer representing whether the paper is accepted or not.

\subsection*{Rating Storage}
\label{sec:rating-storage}

This smart contract stores all the transactions about the ratings of the reviews
made by the reviewers, and it is the one capable to give reputation to each one
of them.

Each ratring is represented by a struct that has: a hash that univocally
identifies the review of a paper made by a reviewer, the address of the person
making the rating and a score that represents the reputation that is given to
the reviewer.

\subsection*{ReviewersHub}
\label{sec:reviewershub}

It is responsible for storing the addresses, the research fields, and the
reputation of the reviewers who are registered in the platform. This contract is
used by: the Rating Storage contract to give the reputation scores to the
reviewers, the new reviewers who want to register on the platform and the
editors who want to find new reviewers to carry out the peer review process.

\subsection*{First prototype problems}
\label{sec:first-prot-probl}

The most important problem that the first prototype had was that the papers were
centralized in the same contract. This centralization involved having to send
many requests to the same Ethereum address, which could mean an increase in
transaction costs.

Another disadvantage of this implementation is that each of the papers was
identified by the index within the library's contract array. Identifying a paper
with just one number within the system did not have any advantage over the
identifiers for today's digital content. The second prototype tries to alleviate
this problem using Ethereum addresses as a paper unique identifier.

Finally, using arrays and data strucs implied that the data were stored in a
linear way, something that is inadvisable in terms of scalability. This design
could mean a loss of performance of the platform when a critical number of
papers was reached. The second prototype uses hash maps instead of arrays to
store the information, a more efficient and cheaper way to store data in
blockchain.


\section{Second prototype}
\label{sec:second-prototype}

The second and final prototype of the platform is based on the use of smart
contracts as papers, meaning that papers submitted to the platform have an
associated Ethereum address. This implies that one of these addresses can
uniquely identify a paper inside and outside Decentralized Science. Furthermore,
since all interactions with the platform are timestamped, an Ethereum address
could be used to demonstrate the authorship of a scientific paper even if it has
not been published.

\figura{UML_class_2.png}{width=0.93\linewidth}{contracts:uml2}{Second
  prototype's diagram of the contracts structure}

As shown in the figure~\ref{contracts:uml2}, the contract structure has been
modified to eliminate the use of strucs except the one used to store the IPFS
addresses.

In addition, the arrays have been replaced by ``mappings'', a type of data in
Ethereum similar to a hashmap that improves the costs of read and write
operations within smart contracts.

The smart contracts used in the second prototype are the following:

\sst{Journal contract}

This smart contract is similar to the first prototype. Its function is to manage
the publication process. It has an owner who controls the journal and a series
of editors who are responsible for assigning the reviewers of the articles.

The main difference with the previous version is that every time someone sends a
paper to the journal, it creates a paper smart contract the submission
information.

Journal's owners can set time limits for the confirmation time (explained in
chapter \ref{cha:platform-description}) and for the review time. This
information is public, meaning that researchers can decide if they want to
submit a paper to a certain journal depending on the review times they are
looking for.

Finally, this contract has the possibility of adding information such as a
title, a short description, an image, and a list of topics in which the journal
focuses.

\sst{Paper contract}

The biggest contribution of this prototype is that the papers are Ethereum
contracts, therefore they have an associated address (see
section~\ref{tb:cryptos:sm}). Since all the addresses in Ethereum are unique,
they could identify a paper both inside and outside the platform.

The paper contract keeps a reference of the journal to which it has been
submitted. This allows the journal's editors to assign the reviewers for the
peer-review process.

Once the reviewers of a paper have been assigned, they only have to interact
with the paper contract without having to go through the journal, thus freeing
the transaction traffic towards the same address. When the reviews have been
made and if they are favorable, the paper is published automatically.

Besides, each time a review is received, the paper contract allows a rating to
be submitted. In this way, it is the paper itself that controls who can perform
the rating and who has done it already.

\sst{RepHub contract}

This contract is the reputation system, responsible for storing the reputation
of each reviewer. For this, whenever someone wants to make a rating, this
contract communicates with the corresponding paper to verify that the person who
is sending the rating can do so.

If the paper verifies that it can be done, an internal transaction is sent
between the reputation system and the paper to indicate that the person has made
a rating. Once this is done, this contract calculates the reviewer's reputation
and stores it within the system.

\section{Ethereum addresses in Decentralized Science}
\label{sec:ether-addr-decentr}

\figura{add_citations.png}{width=0.93\linewidth}{sec:ether-addr-decentr:citations}{Use
  of Ethereum's addresses in Decentralized Science}

Ethereum addresses (see section~\ref{ts:at}) are strings of characters that
identify an individual or a contract in the blockchain. These addresses are
unique, meaning that two people or contract cannot have the same address. As
mentioned in this chapter, inside Decentralized Science users are authors,
reviewers, editors or some of these roles simultaneously. This implies that an
ethereum address can be used to identify and verify an individual's scientific
career if there are enough interactions with the platform. Each of these is
registered in the blockchain as a transaction that have a transaction id, which
can be used to obtain the information of that particular transaction as shown in
the figure \ref{sec:ether-addr-decentr:citations}. This could mean that in the
future, when researchers need to give credit of their careers, they would only
need to supply their ethereum's addresses (see
Figure~\ref{sec:ether-addr-decentr:history}).

\figura{tx_history.png}{width=0.93\linewidth}{sec:ether-addr-decentr:history}{Researcher's
  transactions list}

Ethereum's smart contracts also have an ethereum address, meaning that all
papers and journals in the platform have its own Ethereum address. Although this
address is unique, it can be used to identify a paper or a journal outside the
blockchain (e.g. with a QR code). This system could potentially replace digital
identifiers like ISBN or DOI.

\section{Reputation system}
\label{rep:system}

As mentioned in the chapter~\ref{cha:platform-description}, paper reviews can be
rated by the authors of the paper, the editors who have assigned the reviewers,
and the other reviewers of the article. This first approach attempts to avoid
direct attacks on the reputation of its users, which limits the possibility of
rating a review to the authors and other reviewers of a paper.

On the other hand, if a review is unfavorable, there is a possibility that the
authors will misqualify the reviewer, submitting a bad rating. For that reason,
all the ratings are public and visible by anyone, trying to discourage this type
of behavior.

To go deeper into the subject, the internal operation of the system follows the
following steps:

\begin{enumerate}
\item When a reviewer submits a review, the paper's smart contract allows the
  authors and the other reviewers to submit one rating for that particular
  review.
  
\item In order to send a rating, users should make a transaction to the RepHub
  smart contract with the score, the paper's address and the reviewer's address.
  These two data univocally identify a review, since the same reviewer can not
  make two different reviews of the same paper.
  
\item For each rating, the system registers the rater and modifies the
  reviewer's reputation, making an exponential smoothing of the score
  received~\cite{gardner1985exponential}. In this case, exponential smoothing is
  used to calculate the average of the score without knowing the total number of
  raters. To be more precise, the alpha value used in the exponential smoothing
  is $0.2$.
\end{enumerate}

Within a reputation system there will always be some controversy and there will
always be methods to placate it~\cite{dellarocas2000immunizing}, but the design
presented in this work is a proof of concept that will evolve as new systems are
designed to appease the problems that have arisen.


\section{Functional requirements and activity diagrams}
\label{sec:funct-requ-activ}

In the following section we will discuss the functional requirements and the
activity diagrams for each transaction users can make to interact with the
platform.

Each of these interactions must be done through an Ethereum personal address, as
it works as an identifier of the user performing it. For this reason, all of the
following activity diagrams asks the user for an Ethereum address. In case a new
user uses the platform does not have a personal address, the platform has the
possibility to create one for that user through Metamask\footnote{Used to
  connect a web browser to Ethereum (see section~\ref{jsmm})} extension.
Furthermore, in order to complete each transaction, an essential requirement it
that the user must have an Ethereum account and Metamask installed.

\subsection*{Send paper}

Once an user has an Ethereum account users may send papers. The platform asks
for the authors and for the file through a web form. Javascript collects this
information and uploads it to the file system, generating an IPFS file address.
Finally, the platform creates a transaction with this data and sends
them to the platform.

\figuraAD{AD1}% 1. Table ID
{Send Paper}% 2. Table title
{The user uploads a file to IPFS and creates an Ethereum
  transaction}% 3. Table description
{Authors' addresses and a file}% 4. Inputs
{IPFS address of the paper}% 5. Outputs
{Authors' Ethereum addresses and Metamask installed}% 6. Requirements
{The journal's contract address exists and the IPFS node is
  online}% 7. Preconditions
{There is a transaction to the journal's contract address with the information\\
  & about the authors and the IPFS address of the paper }% 8. Postconditions
{sending a paper}% 9. Caption description

\subsection*{Assign Reviewer}

After receiving a paper, an editor has the possibility to assign reviewers. To
do so, there must be a journal in the platform and the user must have editor
privileges. Next, the editor assigns the reviewers of the paper and the platform
creates the review tasks for each reviewer. These tasks can be accepted or
rejected by the reviewers and they can have a ``deadline'' to receive the
acceptance.

\figuraAD{AD2}% 1. Table ID
{Assign Reviewer}% 2. Table title
{An editor assigns reviewers for a paper}% 3. Table description
{Ethereum's addresses of the reviewers}% 4. Inputs
{_}% 5. Outputs
{The journal's contract address exists} {The address has editor permissions and
  there is a paper pending}% 6. Requirements
{A transaction is created to assign the reviwers}% 7. Preconditions
{assigning a reviewer}% 9. Caption description

\subsection*{Accept review task}

As explained before a reviewer has the possibility to accept or reject a review
task. Before doing so the reviewer must have an Ethereum address. Then, when a
task is received, the reviewer has a time limit to accept it. If it is not
accepted, the editor can assign another reviewer. In the other hand, if it is
accepted, a transaction is created and the reviewer is registered in the paper
to be able to send his review.

\figuraAD{AD3}% 1. Table ID
{Accept Review task}% 2. Table title
{An address authorized as a reviewer accepts the review
  task}% 3. Table description
{A boolean accepting the task}% 4. Inputs
{_}% 5. Outputs
{The journal's contract address exists}% 6. Requirements
{There are a petition pending for the reviewers address}% 7. Preconditions
{The petition is accepted and there is a deadline to submit the
  review}% 8. Postconditions
{accepting a review}% 9. Caption description

\subsection*{Send review}

Sending a review is similar to sending a paper. The main difference is that the
reviewer, once she accepts the review task, has a deadline to submit a review.
If the review is sent before the time limit, the file containing the review is
uploaded to IPFS and the platform created an Ethereum transaction with this
information. Otherwise, if the review is not sent, the reviewer loose reputation
within the reputation system.

% \begin{minipage}{\linewidth}
\figuraAD{AD4}% 1. Table ID
{Send Review}% 2. Table title
{A reviewers submits the review previously accepted via
  IPFS}% 3. Table description
{A file containing the review}% 4. Inputs
{The IPFS address of the review}% 5. Outputs
{The journal's contract address exists}% 6. Requirements
{The submission is within the deadline}% 7. Preconditions
{There is a transaction to the journal's contract address}% 8. Postconditions
{sending a review}% 9. Caption description
% \end{minipage}

\subsection*{Rate review}

As part of the reputation system, all reviews can be rated by the authors and by
other reviewers. When a review is sent to the platform, it generates an
identifier which can be rated with 1 or 0 as explained in
section~\ref{rep:system}.

\figuraAD{AD5}% 1. Table ID
{Rate Review}% 2. Table title
{The authorized address submits a rating about a review}% 3. Table description
{An integer from 1 to 5}% 4. Inputs
{_}% 5. Outputs
{The journal's contract address exists}% 6. Requirements
{The address can submit a review and has not done it yet}% 7. Preconditions
{The reputation of the reviewer is affected by de rating}% 8. Postconditions
{rating a review}% 9. Caption description
% \clearpage
\section{Transaction costs}
\label{arch:trans}

As explained in the section \ref{tech:sec:ethereum} Ethereum's smart contracts
are transaction-based. Each interaction with the platform is registered in the
blockchain with a transaction. The architecture of this platform allows the
following transactions (all the costs are estimated\footnote{Check
  https://ethgasstation.info/ for more info about transaction costs}).

\begin{itemize}
  \itbf{Create the platform:} To create Decentralized Science, there must be an
  initial transaction with the source code to upload it to the blockchain.
  Normally this transaction is the most expensive, and is only necessary to do
  it once.

  This transaction creates all the contracts required by the platform and
  outputs the Ethereum's address of the platform. The gas amount required to
  this transaction is 1696291, approximately from 2 to 4 euros.

  \itbf{Send Paper:} To send a paper the user have to send a transaction
  containing the Ethereum addresses of the authors and a file containing the
  paper. This file will be uploaded to IPFS through a node and then the
  resulting address will be inserted in the transaction. The gas amount required
  to do this transaction is 114812, around 0.30 euros.

  \itbf{Assign reviewers:} Assigning the reviewers is normally very cheap
  because an address authorized as ``editor'' must create a simple transaction
  with the paper identifier and the addresses of the reviewers. The gas amount
  required to do this transaction is 58707, around 0.10 euros.

  \itbf{Accept review:} This is the cheapest interaction with the platform, a
  reviewer must accept the task of reviewing a paper through a transaction. With
  this system, the reviewer cryptographically signs the contract and compromises
  to review the paper. The gas amount required to do this transaction is 23971,
  around 0.04 euros.

  \itbf{Send Review:} Sending a review is also expensive, because the review is
  also a file and must be uploaded to IPFS. The transaction also contains a
  reference to the paper, and an integer representing the acceptance level of
  the review. The gas amount required to do this transaction is 149760, around
  0.35 euros.

  \itbf{Send Rating:} This transaction contains a reference to the review of a
  paper, and an integer from 1 to 5 representing the rating of a review. This
  transaction initially can only be made by the other reviewers, the authors and
  the editors of the journal. The gas amount required to do this transaction is
  94122, around 0.04 euros.

\end{itemize}

\subsection{Reducing transaction costs}
\label{sec:reduc-trans-costs}

In Ethereum, reducing the transaction costs is critical, because all
interactions have costs to the user. In the Decentralized Science's contract the
following actions have been taken to reduce these costs:

\begin{itemize}
  \itbf{Reduce the amount of data stored in the blockchain:} Storing data in the
  blockchain is slow and expensive, so the platform only registers the minimum
  amount of data necessary to verify the interactions between users. All data
  files are stored in IPFS and the addresses are saved in Ethereum. \itbf{Avoid
    expensive data types:} Working with ``Strings'' in \ii{Solidity} is very
  expensive, that's why it is recommended to avoid using this data type.

  As described in the chapter~\ref{cha:platform-description}. Decentralized
  Science stores IPFS addresses as links between Ethereum and IPFS, but these
  addresses are \ii{Base58} strings, a data type not implemented yet in
  Ethereum. This address has 3 parts: 1) A number representing a hash function,
  2) another number representing the size of the file and 3) a hashed string of
  the data done with the 1) hash function.

  To save an IPFS address in Ethereum, initially a ``String'' data type was
  used. But to avoid using this type, first the \ii{Base58} is decoded into the
  three parts mentioned above. Finally instead of creating a transaction with a
  string representing the IPFS address, the transaction contains the hash ID
  (using an \emph{uint8} data tyoe), the size (using an \emph{uint8} data type)
  and the hashed data (using \emph{bytes32} data type). This allows the platform
  to reduce the transactions involving a file upload around $40\%$.
\end{itemize}

%%% Local Variables:
%%% mode: latex
%%% TeX-master: "../Tesis.tex"
%%% End:
