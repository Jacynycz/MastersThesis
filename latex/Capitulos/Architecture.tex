
\chapter{Architecture}

\begin{FraseCelebre}
  \begin{Frase}
    There is no secret ingredient. To make something special you just have to believe it's special.
  \end{Frase}
  \begin{Fuente}
    Mr. Ping - Kung Fu Panda
  \end{Fuente}
\end{FraseCelebre}

The platform's architecture consists in two main parts as explained in section
\ref{tech}, a decentralized file system in which users can upload all the files
using IPFS (see section \ref{tech:sec:ipfs}), and a smart contact to register
all interactions of the users with the platform (see section
\ref{tech:sec:ethereum:sm}).

As a decentralized technology, anyone can run a node locally, connecting to the
IPFS network and to the Ethereum's blockchain to interact with the platform, but
this technology is not used commonly, and not all users have the knowledge to
install and run these programs. As a solution, a ``gateway server'' to test the
platform's implementation was created, using an web browser extension called
Metamask\footnote[1]{https://metamask.io/} to interact with the blockchain and
running an IPFS node to upload the files as explained in the diagram of the
figure \ref{plat:sec:architecture:diagram}.

\figura{architecture.png}{width=0.9\linewidth}{plat:sec:architecture:diagram}{Architecture
  diagram of a node with IPFS and an Ethereum light client}

\section{Smart Contract Architecture}

Smart contracts in Ethereum can interact with each other, creating an ecosystem
of programs that resemble object oriented programming. The contract structure
and source code is crucial because once a contract is in the blockchain, there
is no way to change it.

Uno de los retos más importantes para diseñar un contrato es reducir el coste de
las transacciones, ya que el coste de estas puede ser muy alto si el diseño del
contrato es ineficiente~\cite{croman2016scaling}. Insertar datos en la
blockchain es muy caro, por eso es recomendable utilizar las estructuras de
datos que nos ofrece Ethereum para reducir el coste de transacción. El problema
se plante a cuando se utilizan direcciones en IPFS, las cuales están en BASE58,
un tamaño que sólo es representable por el tipo ``string'' de Ethereum, el cual
es de los tipos de datos maś costosos para trabajar.

Como solucion, la plataforma realiza la funcion hash en base32 de la dirección
IPFS, convirtiendo el tipo de dato a uno mucho más barato para reducir en un
30\% el coste de transacción.


\figura{contracts.png}{width=0.93\linewidth}{contracts:uml}{Diagrama UML del la
  estructura de contratos}

La estructura dentro de Ethereum está separada en cuatro contratos inteligentes
como se muestra en la figura \ref{contracts:uml}.

\subsection*{Decentralized Journal:} Es el contrato desde el que se controlan
todas las interacciones con los autores y los editores. Los autores interactuan
con este contrato para enviar los papers, y los editores asignan a revisores
para el proceso de revision por pares.

Este contrato tiene una serie de direcciones ethereum asociadas a los editores
(que son los que pueden asignar revisores) y una referencia a la dirección del
contrato de la librería donde se envían los papers. Además a través de este
contrato, los revisores envían las revisones que les han sido asignadas.

\subsection*{Decentralized Library} Este contrato es el que almacena las
direcciones IPFS de los papers y controla si están aceptados por el journal o
no.

Es el que se encarga de convertir una dirección de base58 a bytes32 para que sea
manejada por la plataforma. Esta conversión se realiza en dos pasos: primero el
navegador web a través de javaScrpt deshace la base58 obteniendo los datos de
identificación de un archivo IPFS (ver seccion \ref{tech:sec:ipfs}). Luego
almacena estos datos en el struct ``Multihash'' y realiza una función hash para
obtener un dato mucho más eficiente para trabajar en Ethereum. Este hash es el
que se propagará por el sistema de contratos como idetificador del paper que se
ha enviado.

Además cada paper almacena todas las revisiones que tiene a trasvés de un struct
``Review''. Este struct tiene una dirección IPFS con la review realizada, la
dirección del reviewer y un entero representando si el paper es aceptado o no.

\subsection*{RatingStorage}
Este contrato tiene como finalidad almacenar los ratings de las reviews que han
hecho los reviewers, y se encarga de dar la reputacion a estos.

Cada ratrings es represetnado por un struct que tiene: un hash que identifica
univocamente la review de un paper hecha por un revisor, la dirección de la
persona que realiza el rating y un score que representa la reputacion que se le
da al revisor.

Respecto al sistema de reputación se decidió adoptar un sistema de reputación de
cinco estrellas explicado en la sección \ref{rep:system}

\subsection*{ReviewersHub}

Se encarga de almacenar las direcciones, los campos de investigación, y la
reputación de los revisores que etán registrados en la plataforma.

Con este contrato interactuan el contrato de Rating storage para dar las
puntuaciones de reputación a los revisores, los nuevos revisores que quieran
registrarse en la plataforma y los editores que quieran encontrar a un revisor
para realizar el proceso de revisión por pares.


El diagrama \ref{contracts:uml} intenta ilustrar la estructura de los contratos
a través de un UML, si asemejamos los contratos inteligentes de ethereum como
objetos.


\section{Reputation system}

De los muchos sistemas de reputación comentados en la sección \ref{soa:rs} se ha
decidido para la plataforma, un basado en un rating de cinco
estrellas~\cite{kinateder2003architecture}. Este sistema de reputación esta
presente en muchas plataformas online como
Tripadvisor\footnote{https://www.tripadvisor.com/},
Amazon\footnote{https://www.amazon.com/}, Google
Play\footnote{https://play.google.com/} y otras grandes plataformas en las que
los productos y servicios son votados por los usuarios.

Al estar implementado directamente en la blokchcain, todas las interacciones son
totoalemtne públicas y auditables, con lo que cualquier usuario puede ver quien
ha votado a qué revisión y con qué puntuacion, por lo que es un sistema en el
que no existe de base el anonimato. Esto permite disuadir los problemas
comentados en la sección \ref{soa:rs} ya que los usuarios que realizan ataques
dirigidfos o rating injustas se ven expuestos públicamente en la red.

El funcionamiento interno tiene varios sencillos pasos:
\begin{enumerate}
\item Al hacer una revisión, se crea en el sistema un hash
  (SHA-3~\cite{aumasson2008sha}) con: la dirección del revisor, la dirección del
  paper al que revisa y la dirección del journal que le asignó la revisión. Este
  hash identifica unívocamente a la revisión dentro del sistema.
\item Tanto los autores, como los editores y los otros revisores del paper
  tienen la capacidad de asignar una puntuación del 1 al 5 indicando si piensan
  que la review es justa o no.
\item Por cada votación, el sistema registra al votante y le envía la reputación
  al revisor, realizando un alisado exponencial de todas las votaciones que ha
  recibido~\cite{gardner1985exponential}.
\end{enumerate}

A la hora de decidir si una review es justa o no hay varios puntos de
vista~\cite{daniel1993guardians,cole1979fair} y no todas las personas estarán de
acuerdo, pero si se puede ofrecer unas guía de actuación para la comunidad para
intentar que todas las votaciones del sistema sean lo más justas posibles para
todos los que la utilicen.

Dentro de un sistema de reputación siempre existirá cierta controversia y
siempre existirán métodos para poder aplacarla~\cite{dellarocas2000immunizing},
pero el diseño se presenta en este trabajo es una prueba de concepto que irá
evolucionando a medida que se diseñen nuevos sistemas para aplacar los problemas
que surgen.

\section{Transactions}

As explained in the section \ref{tech:sec:ethereum} Ethereum's smart contracts
are transaction-based. Each interaction with the platform is registered in the
blockchain with a transaction. The architecture of  this platform allows the
following transactions.

\subsection*{Send Paper}
\subsection*{}

\section{Proof of concept showcase}
\label{poc}
La prueba de concepto presentada bajo el nombre de ``Alexandría'' es una
aplicación web que engloba todo el proceso de publicación academica actual. Esta
demo se conecta al contrato descrito en la seccion \ref{contratos}
\figura{homepage.png}{width=0.8\linewidth}{poc:homepage}{Diagrama UML del la
  estructura de contratos}
\figura{topic.png}{width=0.8\linewidth}{poc:topics}{Diagrama UML del la
  estructura de contratos}
\figura{rating.png}{width=0.8\linewidth}{poc:rating}{Diagrama UML del la
  estructura de contratos}
