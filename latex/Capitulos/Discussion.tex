\chapter{Discussion}

\begin{FraseCelebre}
  \begin{Frase}
    Sooner or later you're going to realize just as I did that there's a
    difference between knowing the path and walking the path.
  \end{Frase}
  \begin{Fuente}
    Morpheus - The Matrix
  \end{Fuente}
\end{FraseCelebre}

Como resultado del desarrollo de esta plataforma, el escenario ideal sería que
algunos journals que ya hayan ido migrando al sistemas de publicación
alternativos como los explicados en la sección~\ref{soa:aps} se adapten a esta
plataforma. Al realizar un pequeño análisis del posible impacto de este trabajo
hay dos puntos importantes a destacar.

\section{Monetary Impact}
El impacto monetario sería uno de los más notables tras la implantación de este
sistema en sistemas de publicación científica de hoy en día. Según datos de
investigaciones al respecto, el coste de publicación de un artículo en una
revista de imacto varía de entre 1000\$ hasta los
5000\$~\cite{van2013true,russel2008business}, coste muchas veces inviable para
investigadores que quieran avanzar en la investigación científica.

El coste de la publicación y el acceso a la ciencia a través de el trabajo
propuesto sería únicamente variable en función del precio del ETH\footnote{La
  criptomoneda de Ethereum explicada en la sección \ref{tech:sec:ethereum}}.
Tras realizar un análisis con varias versiones de la plataforma desplegada se
determina que el coste de una transacción varía entre 100000 y 150000 de
gas\footnote{Gas es lo que pagas como comisión a la red de Ethereum por ejecutar
  una transacción}.

Teniendo en cuenta que para que se publique un paper han de realizarse como
mínimo 5 transacciones se puede determinar que el precio actual para publicar un
paper ronda entre los 4\$ y 6\$ segun datos de Ethereum Gas
Station\footnote{Precio de una transacción en Ethereum
  https://ethgasstation.info/}, más de 250 veces más barato que los sistemas de
publicación actual en el mejor de los casos.

\section{Review Time and Quality Impact}

Otro de los impactos importantes sería la reducción del tiempo y el aumento de
la calidad en el proceso de revision por pares.

Si se dispone de una masa crítica de usuarios de la plataforma propuesta, se
crea un ecosistema de usuarios que alimentan tanto la red de reputación de
revisores como los contratos de publicación científica (ver
section~\ref{contracts}). El proceso de Peer review se vería afectado de dos
maneras:

\begin{enumerate}
\item \textbf{El tiempo de revisión:} Los contratos inteligentes permiten
  establecer tiempos límites para la revisión de un artículo, suponiendo una
  penalización a los revisores que no cumplan estos plazos (ver sección
  \ref{reputation}). Si un Decentralized Journal tiene unos tiempos de revisión
  establecidos, y los revisores que asignan aceptan las revisiones,
  probablemente se experimente una mejoría en el tiempo de entrega de las
  revisiones y por lo tanto en el proceso de publicación, con respecto a los
  tiempos muchas veces excesivos de los sistemas de publicacion
  actual~\cite{huisman2017duration}.
\item \textbf{La calidad de las revisones:} Todas las revisiones son rateables
  por la comunidad y afectan directamente a la reputación del revisor, así que
  es altamente probable que la calidad de las revisiones en el sistema sufra una
  mejoría y desaparezcan muchos de los problemas respecto a la revisión por
  pares comentados en la sección \ref{intro}.
\end{enumerate}

\section{Science Distribution}

Toda interacción con la plataforma ha de realizarse mediante una cuenta Ethereum
(ver section \ref{tech:sec:ethereum}) y quedan grabadas en la blockchain de
este. Esto implicaría que a través de la dirección de un investigador científico
se puedan obtener dato de todos los papers que ha publicado y revisado. Todo la
comunidad científica podría sufrir una mejor gracias a este sistema. Además, los
nuevos investigadores que quieren empezar su carrera en el mundo académico
pueden obtener visibilidad si los revisores que revisan los papers que envían a
los Distributed Journals tienen alta reputación o no.

Si el sistema se implantara de manera exitosa, la comunidad científica empezaría
a cuestionarse la existencia de los publishers, y si estos desaparecieran, se
encontrarían nuevas formas de financiación de proyectos.

\section{Problems}
\figura{chart.png}{width=0.99\linewidth}{prob:txfee}{Ethereum transaction fees
  evolution}


Hay varios problemas para implantar este proyecto hoy en día, ya que las
tecnologías propuestas todavía tienen poda expansión y son poco conocidas por el
usuario medio.

La primera es el cambio de plataforma para los investigadores de la comunidad,
ya que la costumbre de utilizar las plataformas de hoy en día es dificil de
cambiar, por lo que proponer un cambio en los sistemas de comunicación y
revisión puede suponer un gran rechazo inicial, pudiendo llevar al proyecto a un
punto muerto, sobretodo si la metodología de conexión a la blockchain de
Ethereum es compleja actualmente.

La segunda es el precio de las transacciones. Ethereum es una moneda que fluctúa
bastante, y ultimamente se ha experimentado una gran crecida de todas los
precios de las criptomonedas con respecto a hace seis meses. Las subidas en el
precio provocan subidas en las transacciones, que a su vez provoca que la
interacción sea más cara para todos los usuarios que la utilizan.




%%%
%%% Local Variables:
%%% mode: latex
%%% TeX-master: "../Tesis.tex"
%%% End:
