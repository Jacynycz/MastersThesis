\chapter{Discussion}

\begin{FraseCelebre}
  \begin{Frase}
    I am looking for someone to share in an adventure that I am arranging, and
    it's very difficult to find anyone.
  \end{Frase}
  \begin{Fuente}
    The Hobbit - J. R. R. Tolkien
  \end{Fuente}
\end{FraseCelebre}

As a result of the development of this platform, the ideal scenario would be
that journals with impact factors like the mentioned \emph{JCR} adopt this
system. There are some important points to consider if this platform becomes
widely used.

\section{Issues and possible solutions}
\label{sec:privacyReview}

\figura{privacyReview.jpg}{width=0.7\linewidth}{PrivacyReviewRating}{Review
  privacy models}

Anonymity of reviewers and authors in peer reviews is traditionally used to
improve the fairness of the process. Thanks to single blind reviews, anonymous
reviewers can honestly criticize a paper without fearing the reactions of the
authors. Double blind reviews also allow to reduce the impact of personal
biases. Finally, open review models propose that both authors and reviewers know
each other. These different privacy settings are shown in the Figure
\ref{PrivacyReviewRating}. Nevertheless, the anonymity of the reviewers can also
be abused. Unfair or low quality reviews were not discouraged by the system due
to the lack of consequences.

\subsection{Privacy for the peer review process}
\label{sec:privacy-peer-review}

As mentioned in previous sections, all the information is stored in either IPFS
(files) or Ethereum (interactions). All this information is public for everyone,
which can be privacy problem for the reviewers if they do not want to reveal
their identity. Exploring different possible configurations for each model of
peer review may solve this problem.

Each of the anonymity options of the system requires different solutions, which
are discussed bellow. The question of whether we can keep the benefits of blind
review while providing accountability and recognition to reviews deserves
special consideration.

\subsubsection*{Blind peer review}
Blind review is the protection of the identity of reviewers in the peer review
process. This way, anonymous reviewers can honestly criticize a paper without
fearing the reactions of the authors. In a blockchain, this protection could be
easily achieved by using single-use addresses previously agreed with the editor.

\subsubsection*{Double blinded peer review}
A double blinded review is a blind review that additionally protects the authors
identity to prevent social bias~\cite{lee2013bias,budden2008double}. Authors
could protect their identities prior to publication by providing a single-use
public address on submission. Later they can reveal their real identity since
they are the only ones with access to that address.

\subsubsection*{Open peer review}

Open review models propose that both authors and reviewers know each
other~\cite{ford2013defining}. While studies found effect on the percentage of
reviewers declining to review~\cite{van1999effect} other implications remain
open to debate~\cite{groves2010open}.



% \subsection*{Blind peer review}
% Blind review is the protection of the identity of reviewers in the peer review
% process. In a blockchain, this protection could be easily achieved by using
% single use addresses or passwords agreed with the editor.

% \subsection*{Double blinded peer review}
% A double blinded review is a blind review that additionally protects the
% authors identity to prevent social
% bias~\cite{lee2013bias}~\cite{budden2008double}. Authors could protect their
% identities prior to publication by providing a single use public key from
% which later they sign their real identity or signing the paper with the hash
% of their names followed by a random constant, revealing the constant after
% acceptance.

% \subsection*{Open peer review}
% Open evaluation proposes the opening and deanonymization of peer
% review~\cite{ford2013defining}. While studies found effect on the percentage
% of reviewers declining to review~\cite{van1999effect} other implications
% remain open to debate~\cite{groves2010open}.

% Signed reviews are easy to implement by maintaining a public identity for the
% reviewer.

\subsection{Privacy for the rating process}
\label{sec:privacyRating}


%%%%%%%%%%%%%%%%%%%%%%%%%%%%%%%%%%%%%%%%%%%%%%%%%%%%%%%%%%%%%%%%%%%%%%%%%%%%%%%%%%%%
% \figura{privacyRating.jpg}{width=0.7\linewidth}{PrivacyRating}{Reviewer %
% reputation privacy models} %
%                                                                                  %
% Different privacy sections for peer review have been discussed in
%                                                                                  % previous %
% subsection. Note, however, that the anonymity of the reviewers can be also %
% abused. Unfair and low quality reviews are not discouraged by the system due
% to %
% the lack of consequences. In order to alleviate this problem, our system %
% proposes the construction of a reputation network of peer reviewers so that %
% reviewers are awarded or criticized according to their work. This reputation %
% network can also adopt different privacy settings, allowing both anonymous
% and %
% signed ratings of either signed or anonymous reviews as depicted on Figure %
% \ref{PrivacyRating}. Next, we discuss the different privacy models for the %
% proposed reviewer reputation network. %
%                                                                                  %
% % where review reports are rated by authors, reviewers and editors. The
%                                                                                  % possible %
% % privacy settings of this system are presented in the following
% % subsections. %
%                                                                                  %
% \subsubsection*{Open
%                                                                                  % Rating} %
% Similarly to open reviews, open ratings are easy to implement by maintaining
% a %
% public identity for the raters and reviewers. %
%                                                                                  %
% \subsubsection*{Anonymous
%                                                                                  % Rating}                                                %
%                                                                                  %
% Protecting the identity of raters is interesting in several reputation
%                                                                                  % systems. %
% We can support this anonymity feature using \emph{blinded tokens} %
% ~\cite{schaub2016trustless} that grant permission to rate without revealing
% the %
% identity of the rater. People authorized to rate a review, such as authors, %
% editors and other reviewers involved in the process, may each get one of
% these %
% tokens. %
%                                                                                  %
% \subsubsection*{Rating Anonymous
%                                                                                  % Reviews}                                        %
%                                                                                  %
% The question of whether we can keep the benefits of blind review while
%                                                                                  % providing %
% accountability and recognition to reviewers deserves special consideration. %
%                                                                                  %
% In a system that support voluntary signing of reviews, unsigned reviews
%                                                                                  % would %
% not affect the reviewers reputation unless they acknowledge their
% authorship. %
% Thus, reviewers may only reveal their identities for well rated reviews, %
% reducing the desired accountability for poor quality, unfair or late
% reviews. %
%                                                                                  %
% A system allowing anonymous, yet accountable, reputation system for
%                                                                                  % peer %
% reviewing is therefore of great interest. Following, we discuss the
% feasibility %
% of adopting different anonymity and accountability approaches to realize
% this %
% novel system. %
%                                                                                  %
% \subsubsection*{Accountability} %
%                                                                                  %
% \emph{Collateral models} are widely used in blockchain technology to ensure
%                                                                                  % that %
% an actor assumes negative consequences of an interaction in order to avoid
% the %
% greater consequence of loosing the collateral. A similar strategy can be
% used %
% for the anonymous reputation network. If a reputation collateral is
% requested %
% from the reviewers, they would be encouraged to claim even negative ratings. %
%                                                                                  %
% \subsubsection*{Anonymity} %
%                                                                                  %
% % \emph{Coin mixing} protocols are designed to obfuscate the relation
%                                                                                  % between %
% % senders and receivers of Bitcoin payments by mixing in a single
% % transaction %
% % many senders and receivers ~\cite{meiklejohn2015privacy}. We can not
% % directly %
% % apply this approach to rate reviews since the receiver identity is known. %
% % However it can be used in collaboration with other techniques discussed
% % below. %
%                                                                                  %
% The previously discussed accountability tools can be combined with
%                                                                                  % anonymity %
% measures to ensure accountable yet anonymous, peer reviews. Following we %
% introduce blockchain anonymity tools that could be used to implement this %
% system.\commsem{ (TODO: I feel like this section is too ethereal, and I think
% we %
% have the risk to be rejected because of that.)} %
%                                                                                  %
% \emph{Reusable payment codes} enable the possibility of using a large amount
%                                                                                  % of %
% addresses to receive a payment ~\cite{harrigan2016unreasonable, %
% ranvierReusable}. Reviewers may share one of these addresses with each of
% the %
% actors with permission to rate and then collect the reputation probably using
% an %
% anonymity layer such as coin mixing ~\cite{meiklejohn2015privacy}. %
% % Using a collateral model would encourage the acceptance of bad ratings. %
%                                                                                  %
% \emph{ZK-Snarks} are a cryptographic tool enabling to prove a statement
%                                                                                  % without %
% revealing anything else than the statement is in fact true (Zero-Knowledge
% Proof %
% of Knowledge) ~\cite{blum1988non,bitansky2013succinct}. They also provide
% this %
% property in a succinct and non-interactive fashion \comm{(i.e. using a %
% relatively small proof and not requiring further communication between
% prover %
% and verifier)}. Zcash uses this technology to build an anonymous %
% cryptocurrency~\cite{sasson2014zerocash}. A similar approach could be used
% to %
% manage anonymous ratings. With this technology, a reviewer could receive the %
% rating of a review she did without revealing from which review report or
% which %
% rating the reputation comes. %
% % As before, a collateral model can be used to encourage the acceptance of
% % low %
% % ratings. %
%%%%%%%%%%%%%%%%%%%%%%%%%%%%%%%%%%%%%%%%%%%%%%%%%%%%%%%%%%%%%%%%%%%%%%%%%%%%%%%%%%%%





\begin{table}[h!]
  \centering
  \begin{tabular}{l|l|l|}
    \cline{2-3}
    & \textbf{\begin{tabular}[c]{@{}l@{}}Public\\ Reviewer\end{tabular}} & \textbf{\begin{tabular}[c]{@{}l@{}}Anonymous\\ Reviewer\end{tabular}} \\ \hline
    \multicolumn{1}{|l|}{\textbf{\begin{tabular}[c]{@{}l@{}}Public\\ Rater\end{tabular}}}    & \begin{tabular}[c]{@{}l@{}}Signed rate\\of open review\end{tabular}              & \begin{tabular}[c]{@{}l@{}}Signed rate\\of blind review\end{tabular}          \\ \hline
    \multicolumn{1}{|l|}{\textbf{\begin{tabular}[c]{@{}l@{}}Anonymous\\ Rater\end{tabular}}} & \begin{tabular}[c]{@{}l@{}}Anonymous rate\\of open review\end{tabular}           & \begin{tabular}[c]{@{}l@{}}Anonymous rate\\of blind review\end{tabular}            \\ \hline
  \end{tabular}%

  \caption{Different configurations to rate a review}
  \label{table-rate}
\end{table}

% Different privacy settings for peer review have been discussed in the previous
% subsection.
This section builds on the previous section, adding a new layer of complexity:
we do not only deal with reviews, but with both reviews and ratings. As already
proposed in Section~\ref{rep:system}, the construction of a reputation network
of reviewers may improve the accountability of the peer review process. Thus,
this section explores the different privacy settings such reputation systems may
have. One of these settings, the rating of blind reviews, is explored in more
detail. Challenges of such system are identified, and will later guide Section
\ref{sec:realizing} discussion on how it may be achieved.

% where review reports are rated by authors, reviewers and editors. The possible
% privacy settings of this system are presented in the following subsections.

\subsection*{Signed Rating}
% \comat{TODO: use signed vs anonymous rating and open vs blind review, double
% check this naming through all the paper} \comsa{This needs more explanation,
% since "open" is ambiguous. It means publicly visible? by those in the network,
% by google, by who? etc; "open" doesn't mean identifiable raters and
% identifiable reviewers, if you mean that, explain it; and btw "open reviews"
% is unclear also. }
Similarly to the open peer review (explained in the previous section), signed
ratings are public and verified ratings of a review. It is straight forward to
implement by maintaining a public identity for the raters.

\subsection*{Anonymous Rating}

Protecting the identity of raters is interesting in several reputation systems
~\cite{schaub2016trustless}. We can support this anonymity feature using
\emph{blinded tokens}~\cite{schaub2016trustless} that grant permission to rate
without revealing the identity of the rater. People authorized to rate a review
in the system, e.g. authors, editors and other reviewers of the paper involved
in the process, may each get one of these tokens.
% \comsa{why are those the authorized and not others? who defines that?
% where?}\comat{rewriten as an example, so there is n need to justify :)}

\subsection*{Rating Blind Reviews}

The question of whether we can keep the benefits of blind review while providing
accountability and recognition to reviewers (and thus rating their reviews)
deserves special consideration.

The following challenges must be considered in order to provide this privacy
setting:
\begin{enumerate}

  \itbf{Anonymity:}\label{ch:anonymity} The reviewer should be able to claim the
  rating received in her review (e.g. to receive positive reputation) without
  revealing that she is the author of the review.

  \itbf{Accountability:} \label{ch:accountability} The reviewer should not be
  able to avoid the effect of negative reviews (e.g. claiming just the positive
  ratings).
  

  \itbf{Authorization:} \label{ch:authorization} The ratings should come from
  authorized raters (i.e. minimizing cheating).


  \itbf{Sybil resistance:} \label{ch:sybil} Having several identities in the
  system should not provide advantages. Note blockchain systems such as Ethereum
  allow the creation of multiple identities per user.


  \itbf{System abuses:} \label{ch:abuses} The anonymity of the interactions may
  hinder the detection and prevention of system abuses. For instance, malicious
  actors may try to submit fake reviews to be rated by accounts they control in
  order to get unfair good ratings. Detecting this behavior would not be trivial
  since reviews and ratings may be anonymous.

\end{enumerate}

A system allowing an anonymous, yet accountable, reputation system for peer
reviewing would enable a new privacy and accountability model for peer review.
However, its implementation face important challenges such as those described
above. Next section provides an overview of how existing techniques may be
applied to tackle the identified
challenges. %\comsa{are the above sections part of this subsection? unclear, confusing}

\subsection{Achieving Accountable Anonymous Reviews}\label{sec:realizing}

The previous section identifies challenges that an anonymous yet accountable
reputation system for peer reviews face. Some existing technologies have been
applied to similar challenges, and others may help to combine their advantages.
This section explains these technologies and how they may be used to tackle the
challenges of this system. First it provides an overview of how the technologies
may be combined, and afterwards a description of the technologies follows.

A simple way of protecting the identity of users is the use of different virtual
identities for each interaction, i.e. \emph{single-use identities}. However,
linking the reputation received by this single-use identities to their real
identity both providing accountability (Challenge \ref{ch:accountability}) and
preserving the anonymity (Challenge \ref{ch:anonymity}) require the use of other
technologies.

In order to provide accountability (Challenge \ref{ch:accountability}), the
system may try to detect when an identity has not claimed back a bad reputation.
For this purpose, a reputation deposit or \emph{collateral} could be requested
for each rating a reviewer may get. This way, users could compare the number of
claimed ratings and the number of unclaimed ratings, and assume bad ratings for
those that are missing. This collateral-based technique should be applied
carefully, avoiding abuses such as trying to use the same collateral for
different ratings. Advanced cryptographic techniques such as \emph{zk-SNARKs}
(explained below) may help to prove that these requirements are met without
compromising the reviewers' identity. These techniques may be used to allow a
reviewer to claim a rating from a review she did without revealing her identity
but proving her authorship (Challenge \ref{ch:anonymity}).

A different issue is to allow ratings to come solely from authorized raters
(Challenge \ref{ch:authorization}). To attend these authorization requirements,
several techniques such as \emph{blind signatures} or \emph{blind tokens} may be
used. These would enable to grant permission to a collection of identities to
perform an action, e.g. rate a review, without revealing which of them voted, or
which voted for what. As previously, \emph{single-use identities} may be used to
provide anonymity; in this case, for raters.

Allowing only authorized rates, as previously explained, may help to prevent
Sybil attacks (Challenge \ref{ch:sybil}). Moreover, the cost of losing a
reputable identity may reduce the attractiveness of starting a new identity just
for the sake of reputation.

The use of the mentioned zk-SNARKs may also help to prevent some system abuses.
For instance by enabling the use of cryptographic proofs that verify that the
ratings come from reviews submitted to reputable journals, would prevent fake
reviews and ratings.

Next, the previously mentioned technologies are explained.

\begin{description}
\item[Single-use identities] New single-use identities may be used as a simple
  technique to support anonymous interactions (Challenge~\ref{ch:anonymity}).
  However, supporting the authorization rules of the system
  (Challenge~\ref{ch:authorization}) and providing accountability
  (Challenge~\ref{ch:accountability}) for those identities are challenges that
  require consideration.

\item[Ring signatures] Ring signatures \cite{rivest2001leak} are a cryptographic
  technique that allows to authorize a collection of identities to perform an
  action, while keeping the privacy of the specific identity that performed the
  action. They may be used to authorize rates to a group of identities, e.g. the
  authorized raters, without revealing who rated what or who rated. Thus, this
  technique may be used to support the authorization requirements of the system
  (Challenge~\ref{ch:authorization}), while providing some anonymity to the
  users (Challenge~\ref{ch:anonymity}). Note that with this technique, the
  identities of who may have signed are known, so the combination with other
  anonymity measures could be of interest.

\item[Blind tokens] In the context of an election and using a cryptographic
  technique called blind signatures~\cite{chaum1983blind}, it is possible to
  create ballots for authorized actors that preserve the anonymity of the vote
  (both hiding who casted a vote and what each actor voted) but ensuring that
  only authorized voters participated. Note that, as with ring signatures, the
  identities of who may signed are known, and thus complementary anonymity
  measures could be used. This technique has been also used to anonymize a
  distributed reputation system~\cite{schaub2016trustless}. Thus, it could be
  used to provide anonymity to reviewers and raters
  (Challenge~\ref{ch:anonymity}) while supporting the authorization rules of the
  system, i.e. who may submit a review or a rating
  (Challenge~\ref{ch:authorization}).

\item[Collateral pattern] In order to secure the funds needed for a blockchain
  application to function, it is common that the application requests the
  participants to pay as collateral the assets they may lose. For instance, a
  betting smart contract will first ask all participants to pay their bets and
  afterwards distribute the prices. This paper calls this technique "collateral
  pattern", and proposes its use to provide accountability (Challenge
  \ref{ch:accountability}) to the reviewers of the reputation system (Section
  \ref{sec:reputation}). For each rating a reviewer may obtain, the reviewer
  must spend as much reputation as she may lose. This encourages the claiming of
  bad ratings, since not claiming them may result in a bigger loss.
  % \comse{No queda claro como se podría participar en el sistema por primera
  % vez, sin reputación que poner en collateral. No tenemos solución, lo
  % enumeraría como problema a solucionar.}\comat{Es cierto, aunque no se si
  % necesitamos dar tanto nivel de detalle aquí}

\item[zk-SNARK] is a cryptographic procedure enabling to prove a statement
  without revealing anything else, i.e. apart from the evaluation if the
  statement is in fact true (zero-knowledge proof of
  knowledge)~\cite{blum1988non,bitansky2013succinct}. The same authors also
  provide this property in a succinct and non-interactive fashion, i.e. using a
  relatively small proof and not requiring further communication between prover
  and verifier. In fact, the popular ZCash project uses this technology to build
  an anonymous cryptocurrency~\cite{sasson2014zerocash}. Proving statements in
  this privacy preserving manner is of great interest for several challenges of
  the proposed accountable anonymous review system. For instance, proving that a
  user controls a single-use identity (explained above) may allow the user to
  claim the reputation given to that identity (Challenge \ref{ch:anonymity}).
  Additionally, a reviewer may prove that she payed the reputation collateral
  (explained above) needed to submit a review without revealing her identity and
  without being able to use the same collateral for another review (Challenge
  \ref{ch:accountability}). Finally, proving that the reputation comes from a
  review submitted to a collection of honest journals that do not allow abuses,
  may help to mitigate the abuses that fake reviews and ratings represents for
  the system (Challenge \ref{ch:abuses}).

  % \item[Reusable payment codes] is a technique that enables the possibility of
  %   using a large amount of addresses to receive a payment
  %   ~\cite{harrigan2016unreasonable, ranvierReusable}. Reviewers may share one
  %   of these addresses with each of the actors with permission to rate and
  %   then collect the reputation probably using an anonymity layer such as coin
  %   mixing ~\cite{meiklejohn2015privacy}. \comsa{and more concepts!! anonymity
  %   layer? coin mixing?} Using a collateral model would encourage the
  %   acceptance of bad ratings.


\end{description}



% \emph{Coin mixing} protocols are designed to obfuscate the relation between
% senders and receivers of Bitcoin payments by mixing in a single transaction
% many senders and receivers ~\cite{meiklejohn2015privacy}. We can not directly
% apply this approach to rate reviews since the receiver identity is known.
% However it can be used in collaboration with other techniques discussed below.

% This subsection introduces blockchain anonymity tools such as
% \emph{blinded-token}, \emph{reusable payment codes} and \emph{zk-SNARKs} that
% are relevant for the proposal of the paper.




%%%
%%% Local Variables:
%%% mode: latex
%%% TeX-master: "../Tesis.tex"
%%% End:


\subsection{Other Problems}
\label{sec:other-problems}

\figura{chart.png}{width=0.99\linewidth}{prob:txfee}{Ethereum transaction fees
  evolution}

Another problem is that the proposed technologies still have little expansion
and are little known by the average user.

For current researchers, a change of platform can be an inconvenience, since
today's technologies such as \ii{Easychair}\footnote{https://easychair.org/} can
be difficult to replace. Proposing a change in the communication and revision
systems can generate a great initial rejection, taking the project to a dead
end, especially since today's methods to connect to Ethereum's blockchain are
still complex. As a possible solution, Decentralized Science could be deployed
as a demo in small and less know journals to introduce this system to the
academic community. If one journal adopts this technology, it can be spread to
other small journals, trying to extend this system as a substitute of the widely
used ones. Furthermore, in order to improve the Ethereum's usability, a small
framework to enhance user experience can be developed.

Another problem relating to the technology is that Ethereum's cryptocurrency
fluctuates a lot, and lately, there has been a rise in all cryptocurrency prices
compared to a year ago\footnote{According to http://coincap.io/}. As shown in
the figure \ref{prob:txfee} this price affects transactions, which may cause an
increase of the prices to interact with the platform.


\section{Monetary Impact}
\label{sec:monetary-impact}

The monetary impact would be one of the most notable after the implementation of
this system in today's scientific publishing systems. According to research
data, the cost of publishing an article in a open access journal with a high
impact index varies from 1,000 to 5,000
euros~\cite{van2013true,russel2008business}, a cost that is often not feasible
for researchers who want to advance in scientific research. In the other hand,
the costs of reading papers published in non open access journals is usually
unfair and expensive as explored in section~\ref{scb:ps}.

The cost of publication and access to science through the proposed work would
only vary depending on the price of the cryptocurrency used by Ethereum (see
section~\ref{tech:sec:ethereum}}). As seen in section~\ref{arch:trans}, the
maximum cost of a transaction in all the publication process would be a few
euros. Bearing in mind that for a paper to be published at least 5 transactions
must be carried out, it can be determined that the current price for publishing
a paper is around 2 euros according to Ethereum Gas Station\footnote{Price of a
  transaction in Ethereum https: //ethgasstation.info/}, more than 500 times
cheaper than the publishing system mentioned before.

\section{Impact in the scientific community}
\label{sec:science-distribution}

Another important impact would be the reduction of time and the increase of
quality in the process of peer review. If a critical mass of researchers use the
proposed platform, the peer review process would be affected in two ways:

\begin{enumerate}
  \itbf{Improvement in the review time:} Smart contracts allow establishing time
  limits for the review of an article, assuming a penalty to reviewers who do
  not meet these deadlines (see section~\ref{archAndDes}). If a Decentralized
  Journal has defined certain review times, and reviewers who are assigned
  accept the review proposal, it is likely that there will be an improvement in
  the delivery time of the reviews. Therefore the publication process will be
  faster regarding the delivery times of the reviews of today's publishing
  systems~\cite{huisman2017duration}.

  \itbf{Improvement in the review quality:} All reviews are rated by the
  community and directly affect the reputation of the reviewer. In this scenario
  it is likely that the quality of the reviews in the system will suffer an
  improvement and some problems regarding the peer review discussed in section
  \ref{intro} may be alliviated.
\end{enumerate}

All interaction with the platform must be done through an Ethereum account and
recorded in its blockchain. This would imply that through the address of a
researcher you can obtain all the papers this researcher have published and
reviewed. The entire scientific community could improve thanks to this platform
because an Ethereum address could be used as an accreditation system. In
addition, new researchers who want to start their career in academia can gain
visibility thanks to the reputation system.

If the system were implemented successfully, the scientific community could
begin to question the existence of the publishers and new forms of project
funding could be found.




%%%
%%% Local Variables:
%%% mode: latex
%%% TeX-master: "../Tesis.tex"
%%% End:
