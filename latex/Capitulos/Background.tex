\chapter{Background and State of the art}

\begin{FraseCelebre}
  \begin{Frase}
    The future is already here – it's just not evenly distributed.
  \end{Frase}
  \begin{Fuente}
    William Gibson
  \end{Fuente}
\end{FraseCelebre}

% -------------------------------------------------------------------
% \section{Cooler Section}
% -------------------------------------------------------------------

\section{Socio-cultural background}
\label{scb}
\subsection{Publication systems}
\label{scb:ps}
The methodology of scientific publications creation was established in
1620~\cite{bacon1878novum}, in which Francis Bacon established certain steps to
elaborate what we know today as scientific papers. But it was 45 years later, in
1665, when appeared the considered first scientific journal: \emph{Philosophical
  Transactions of the Royal Society}~\cite{kronick1976history}. At that time,
editors were the ones who had to carry out the revision of the papers that would
be published in these journals.

% El proceso de generación de publicaciones científicas actual se estableció en
% 1620~\cite{bacon1878novum} https://pbs.twimg.com/media/DY0rMNmU0AAoVQz.jpgy
% fue en 1665 cuando comenzaron a aparecer lo que hoy consideramos como los
% primeros journals científicos~\cite{kronick1976history}. En esa época los
% editores eran los que tenían que llevar a cabo la revisión de los artículos
% que serían publicados en dichos journals.

It was around 100 years later when an alternative system was adopted, instead of
editors doing the work of reviewing all the papers, this would be done by a
group of experts in a certain field, deciding if each paper reviewed is good
enough to be published or not. This is the beginning of the process known today
as ``peer review''~\cite{spier2002history}.

But scientific publication as we know it today was settled down in the 19th
century, with the actual peer review process~\cite{burnham1990evolution},
stablishing the guidelines of the paper-based paradigm that we have in science
nowadays.

Determining the quality of a scientific paper is difficult, but today we have
different ways to do so. To be able to estimate this quality, usually there are
two approaches: before and after publishing.

\textbf{The peer review process} consists in a group of researchers in a certain
field that evaluate a paper's quality, commonly implying it's eligibility to be
published. They read the paper, and submit a review and an ``acceptance score''
representing if they think the paper should be accepted, have a revision or be
directly rejected. Normally these researchers are unknown to the authors and the
reviews are made anonymously (\emph{blind review}). Sometimes reviewers also do
not know who the authors are (\emph{double blind review}). And in rare cases,
both the authors and the reviewers are public (\emph{open review}). Therefore it
could be considered that this process is an indicator of the quality of a paper
before being published~\cite{szklo2006quality}.

Another way to determine this quality is after it is published, obtaining the
number of citations it gets over time~\cite{redner1998popular}. Normally a paper
with a higher number of citations is considered to have better quality than a
paper with lower number of citations. This way of analyzing quality gives rise
to the generation of indexes that determine how good a researcher is. For
example, the H index that assigns a numerical value to each researcher based on
the articles she has written and the citations he has
received~\cite{bornmann2007we}. Regarding these metrics, several alternatives
have been explored, since determining whether a researcher is good or not
through a number is not always correct~\cite{bornmann2008there}. However, in
most cases, it is possible to predict through this index if the quality of
future researcher papers will have good quality or not~\cite{hirsch2007does}.

The H index is one of the most used factors to measure the impact that has a
researcher. There are a series of similar indexes not to measure the impact of a
researcher, but the impact of a scientific journal. Also resorting to the
citations rate, journals size their quality with these indexes. JCR (journal
citations rate)~\cite{doi:10.1001/jama.295.1.90} is one of the most widely used
and recognized in the scientific community (see chapter~\ref{intro}).
Researchers who seek to get their work recognized should publish papers in
journals with high JCR.

Therefore, today's scientific research system is based on publishing papers in
journals and conferences, as the papers of an academic research that are
published in any other medium are part of the called ``grey literature'', that
usually are not taken into account in the research
process~\cite{rothstein2009grey}.

People looking to make a career in the academic world should bear in mind that
one of the priorities is to publish the maximum number of articles in
high-impact journals, as discussed above. In Spain for example, entities such as
the ANECA~\footnote{www.aneca.es/}, determine if a person can teach at the
university based on a number of factors. One of the most important is the number
of papers published in journals with high JRC that person has. In this way a
need is generated, which certain actors take advantage of within this system.

\bb{Publishers} were in charge of producing, printing and distributing the
editions of the journals when they began to emerge. Many major publishers such
as Willey-Blackwell\footnote{https://www.wiley.com},
Elsevier\footnote{https://www.elsevier.com/} or
Springer\footnote{http://www.springer.com/} have been around since the beginning
of the 19th century. Nevertheless, the digital age has meant that copying a
document, which used to be an expensive process, now has a very low cost. Even
so, publishers continue to profit from this system, acting as intermediaries
between the people who create science and those who consume
it~\cite{lariviere2015oligopoly}. In the era in which the replication of
information is not a cost, the role of publishers can be questioned and migrated
to fairer and more honest scientific publication and dissemination systems.

\subsection{Alternative publication systems}
\label{soa:aps}
Publication systems, as seen on the previous section, form an oligopoly of a
few, concentrating the benefits of the industry. However, there are some
attempts to change this paradigm on behalf of science dissemination.

\bb{Open Access} is a concept referring to all the research material that is
free of cost for the readers. This concept also involves various conditions for
something to be considered ``open access''~\cite{bailey2007open}:

\begin{itemize}
  \itbf{Freely accessible online:} Any user can get a copy of the paper online
  without any cost, including reviewed papers and unreviewed papers (preprints).
  \itbf{Lax copyrights:} Any user can download, print, search, copy and link to
  the full text or any part of these articles. \itbf{Public support:} Everyone
  should be able to access to all the material involving the paper, and should
  be uploaded to a public repository supported by an academic institution or
  other well-established organization that seeks to enable open access.
\end{itemize}

But as mentioned in the chapter~\ref{intro}, in most \emph{open-access}
journals, the costs lie with the authors rather than the
readers~\cite{lariviere2015oligopoly,van2013true}, making unaffordable for some
researchers the possibility to publish in these journals.

\bb{Open journal systems}~\cite{willinsky2005open} is an open software designed
to facilitate the publishing process. This project was created by the Public
Knowledge Project\footnote{https://pkp.sfu.ca/about/} and it targets open-access
online journals that want to speed up the publication processes. The system
provides tools to control the whole publishing process from paper submission,
through peer reviewing to the final publication issue.

\bb{Mega-journals}~(or Multi-journals)~\cite{binfield2013open,wellen2013open}
combine multiple journals into a single journal, allowing the publication of
open-access papers, which have gone through a peer review process. The first
journal to adopt this idea was the \emph{PLOS ONE}
Journal\footnote{http://journals.plos.org/plosone/} as of the project
\emph{Public Library of Science}. This project aims to create a library of
scientific journals under the values of open access and creative commons
licenses. As a result of the success of the \emph{PLOS ONE} journal, other
publishers have started their own mega-journals. Featuring alternative impact
metrics, reusability of figures and data, post-publication discussions and
portable reviews from other journals~\cite{bjork2015have}.

The \bb{continuous publication} model is based on publishing individual papers
migrating from the previous issue-based model~\cite{anderton2013continuous}.
This method is seen as an altenative for open-access journals as it speeds up
the publication process~\cite{haymanview}. \emph{Decentralized Science} adopts
this model by design (explained in section \ref{arch:sca}) as the platform
automatically publishes papers that meet certain preconditions, such as getting
at least two out of threee possitive reviews.

\bb{Preprints} are scientific papers that have not yet gone through the peer
review process~\cite{harnad2003electronic}. Formerly, the preprints that were
sent to the journals were private, and only accessible by the editors and
assigned reviewers. Nevertheless, nowadays it is common to publish a preprint
before sending it to a journal, uploading it to specialized platforms like
arXiv\footnote{https://arxiv.org/} or
Preprints\footnote{https://www.preprints.org/}~\cite{brown2001volution}.
Moreover, there is a correlation between the upload of a preprint and early
citations after the publication of the paper~\cite{shuai2012scientific}. This
system is a possible solution to the cold-start problem that papers of new
researchers who enter the academic career have~\cite{sugiyama2010scholarly}.

Social networks have also made a dent in the academic world, creating platforms
to contact other researchers and encouraging them to share their papers. These
platforms allow users to have a public profile, in which they can add relevant
information about their career: research fields, interests, papers published,
etc. The users also have the possibility to connect with other users and add
them as contacts or fellow researchers. In addition, these platforms allow it's
users to upload scientific papers (reviewed or unreviewed), granting the
possibility to share them through the platform's community. Some of the
well-known are Research Gate\footnote{https://www.researchgate.net},
Mendeley\footnote{https://www.mendeley.com} or
Academia\footnote{http://academia.edu}. But despite the good intentions of the
creators of these platforms, many of the journals demand the copyright of the
papers they publish, preventing the authors from sharing them through these
services.

In the current scientific publication process, an important part of the work is
done by reviewers, who remain anonymous for both the authors and the rest of the
scientific community. Decentralized Science proposes, among other things, to
make a reputation system of reviewers so that they can obtain the credit and
recognition of carrying out good reviews, and be penalized otherwise.

\subsection{Reputation systems}
\label{scb:rs}

\bb{Trust} in an individual is hard to acquire. Normally trust can not be bought,
sold or exchanged. When choosing whether to trust someone or not, having the
opinion of other people about that individual can be helpful. If these opinions
about each individual are stored in a system which can be consulted by anyone,
it is called a ``reputation system''.

A reputation system is a technology that allows the users who use it to trust
third parties inside the same system, without prior knowledge of each other. The
reputation system collects, adds and distributes the comments received from the
behavior of the participants, based on interactions with each other. This idea
was born in the early 2000s in which the use of the internet spread throughout
the world, and systems were needed to be able to trust
strangers~\cite{resnick2000reputation}.

The ``reputation'' is normally a value that indicates how much confidence the
community has in a user. This reputation is gained through interactions with the
rest of the people who use the service or platform that implements the
reputation system. Sometimes these platforms offer \emph{privileges} to people
who have a certain level of reputation, unlocking certain actions that can only
be performed at a certain threshold.

The basic idea of reputation systems is to give users the possibility to rate
the interactions that occur between them. One of the pioneering platforms to
implement this system was Ebay \footnote{https://ebay.com}, a second-hand buying
and selling website. In it, each vendor had a reputation based on whether
previous transactions had been honest or not. Each user who purchased products
from that seller had the opportunity to rate the purchase. In this way, all
those vendors who tried to rip off users immediately received a bad reputation.

Many of the large internet communities such as
Stackexchange\footnote{https://stackexchange.com/} or
reddit\footnote{https://www.reddit.com/} have their own reputation system.
Reputation systems behavior may vary depending on the
platform~\cite{josang2002beta}, but the most usual is the one where users get a
score based on certain interaction with the community.

Reputation systems also have a wide niche in e-commerce webs such as Ebay, as
mentioned before, or Amazon\footnote{https://amazon.com}, in
which people pay for a product sold by an unknown vendor. There must be a
previous trust in the vendor before buying any product, so a reputation system
offers a score given by other users that encourages you to trust or not that
certain seller~\cite{resnick2002trust}.

These systems vary widely in scope, such as one for peer-to-peer
computing~\cite{zhou2007powertrust}, vehicle ad-hoc~\cite{dotzer2005vars}, web
services~\cite{moore2008reputation} and even Wikipedia~\cite{adler2007content}.
All of them are based on an exchange of trust between users who use these
services.

This same concept was intended to be implemented using a
token as a \bb{trust unit}, which users exchanged as a sign of trust deposits among
them~\cite{sharples2016blockchain}.

Nevertheless, reputation systems also have problems when it comes to defend the
users from attacks to individuals~\cite{hoffman2009survey} and unfair ratings
~\cite{whitby2004filtering}.

\section{Technical background}
\label{tb}
\subsection{Network Architectures}
\label{tb:na}
\figura{architectures.png}{width=0.95\linewidth}{tb:na:diagram}{Digram of the
  diferent network architectures}

One of the most important concepts about Decentralized Science is precisely a
distributed network architecture. This architecture proposes removing the
centralized infrastructure, spreading its functionality and privileges to all
the peers in a network

There are three important types of network architecture (See
figure~\ref{tb:na:diagram}~\cite{baran1964distributed}):

\begin{itemize}
  \itbf{Centralized Architecture:} All the structure is managed and controlled
  by a single node. This type of architecture is used nowadays in web services
  in which all of its users have to connect to a centralized server. Some of the
  most used internet platforms like
  Facebook\footnote{https://www.facebook.com/},
  Github\footnote{https://github.com/} or
  Amazon\footnote{https://www.amazon.com/} use this architecture.

  
  \itbf{Decentralized Architecture:} This type of structure is divided into a
  series of nodes that work in its own centralized structure. Fragmenting the
  centralized server in smaller ones interlinked between them. Each one is
  managed by its community. Besides, all its users can access all the data
  stored in all the nodes. This infrastructure is used in platforms like GNU
  Social\footnote{https://www.gnu.org/software/social/},
  Buddycloud\footnote{https://buddycloud.com/} and
  Diaspora\footnote{https://diasporafoundation.org/} among others

  \itbf{Distributed Architecture:} This is a fully decentralized architecture
  and its usually called ``Peer to peer'' or ``P2P''. All the data is spread
  among the nodes called \ii{peers} which have the same privileges in the
  structure. Users can control its contribution to the network and everyone can
  join or leave at any time without affecting the network architecture. Today
  this structure is used in platforms like
  BitTorrent\footnote{www.bittorrent.com/},
  Bitcoin\footnote{https://bitcoin.org/} and
  Ethereum\footnote{https://www.ethereum.org/}
\end{itemize}


One of the examples mentioned above is the architecture of the Bitcoins network,
in which all users can see all the transactions made between them, forming a
fully transparent structure. Ethereum also uses this architecture to build a
network of decentralized, free and open software, granting everyone access to
all the source code.

\subsection{Criptocurrencies}
\label{tb:cryptos}
In the decade of the 80s and the 90s, decentralized forms of payment began to
appear, such as ecash~\cite{chaum1995introduction} that offered a currency with
a high level of privacy; It was then that the concept of ``anonymous electronic
money'' began to emerge.

Wei Dai in 1998 published his proposal for electronic money called
B-money~\cite{bmoney} and from this idea, other proposals have emerged such as
Bit gold~\cite{bitgold}, improving the implementation of a cryptocurrency using
RPOW~\cite{finney2005rpow}, an extension of the Hashcash work test
system~\cite{back2002hashcash}.

\subsection{Bitcoin}
\label{tb:bc}

\figura{transaction.png}{width=0.9\linewidth}{tb:bc:transaction}{The change of
  state of Bitcoin accounts done by transactions}

In 2009, the idea of a decentralized currency first emerged, when Satoshi
Nakamoto published the first version of Bitcoin~\cite{nakamoto2008bitcoin}. The
purpose of this currency was to create a fully decentralized electronic payment
system, using cryptographic tests instead of trust through a concept called
proof of work (\emph{POW}).

All monetary transactions between accounts are stored in a data block until
reaching a specific size, giving a snapshot of all the state of all the accounts
(see figure~\ref{tb:bc:transaction}). In this way, a Bitcoin address does not
contain money but the collection of transactions from other addresses to this.

The idea of \emph{proof of work} is to gather a number of pending transactions
to a data block. Then add a random number (called ``nonce'') to that data so
that when performing a hash function of the whole block has a certain number of
leading zeros. Once a block is created, it is used as reference to the next
block of data, adding the hash of the previous block to the next one. This
system is called the blockchain~\cite{antonopoulos2014mastering} (see
figure~\ref{tech:sec:eth:bc:diagram}).

\figura{blockchain.png}{width=0.9\linewidth}{tech:sec:eth:bc:diagram}{Blockchain
  representation diagram}

All the nodes compete to find the next block, because when one of them manages
to find the nonce to create the hash with leading zeros, notifies it to all the
nodes of the network and gains 12.5 bitcoins~\cite{barber2012bitter}.

This system makes it practically impossible to falsify a transaction in the
blockchain, since the minimum change would cause an totally different hash from
the blockchains of the other nodes, provoking a desyncronization to the
peer-to-peer network. However, this system is still vulnerable to attacks aimed
at specific users, such as man in the middle attacks~\cite{moore2013beware}.

\subsection{Ethereum}
\label{tb:eth}
Ethereum~\cite{buterin2014ethereum} is a very novel technology that allows the
creation of distributed applications that run in an arbitrary large and
trust-less network of nodes. Ethereum's strength rely on three main concepts:
blockchain, smart contract and transactions.

\subsubsection*{Smart Contracts}
\label{tb:cryptos:sm}
\figura{cryptomap.png}{width=0.7\linewidth}{criptomap}{Existing cryptocurrencies
  represented by purchase-sale
  volume~\footnote{https://cryptocoincharts.info/coins/graphicalComparison}.}

Ethereum was born inspired by the Bitcoin concept to offer any user a tool to
develop decentralized and secure applications in a simple
way~\cite{buterin2014ethereum}. These applications are called ``smart
contracts'' and are written in the Ethereum's blockchain.

Ethereum's cryptocurrency is called ``ETH'' and not only works like Bitcoin, to
exchange money between users, but to \emph{fuel} the smart contracts' execution,
running its source code for a small amount of ETH.

Smart contracts are written in a programming language called
Solidity\footnote{http://solidity.readthedocs.io/en/develop/} provided by the
Ethereum's developers. This language is called contract-oriented and it was
influenced by C++, Python and JavaScript. Solidity offers the possibility to
create a wide range of decentralized applications in the blockchain in which
users do not have to trust a centralized organization.

These contracts also have the capacity to store an transfer money, making them
the perfect tool to implement a wide range of decentralized applications like:
gambling games~\cite{piasecki2016gaming}, voting
systems~\cite{mccorry2017smart}, crowdfunding~\cite{jacynycz2016betfunding},
prediction markets~\cite{peterson2015augur}, transparency
systems~\cite{bonneau2016ethiks} and so on. Today it is the most exchanged
currency within cryptocurrencies (see figure~\ref{criptomap}.

Smart contracts offer us a framework to design distributed platforms like the
one proposed in this work, in which all transactions that interact with the
scientific publication process can be cryptographically signed (see
section~\ref{plat:trans}).

\subsubsection*{Accounts and Transactions}
\label{ts:at}
An address in ethereum is a 32-byte string that symbolizes a person's account or
an smart contract:
\begin{itemize}
\item \textbf{Personal accounts:} These are the accounts of the users who want
  to interact with the Ethereum network. Each one has its address and a balance.
\item \textbf{Contract accounts:} A smart contract also is identified by an
  Ethereum address. It contains the source code, a balance with the available
  money, and its own internal memory where it saves all the contract's
  information.
\end{itemize}

In this way, users and contract behave very similar in the blockchain, and to
communicate these addresses, Ethereum uses \bb{transactions}.

\figura{sc.png}{width=0.97\linewidth}{tb:eth:transaction}{State change of
  Ethereum done by transactions}

Smart contracts behavior is transaction-based~\cite{wood2014ethereum}. once a
contract is deployed in the blockchain, users (or other contracts) may send
transactions to run its code. Each transaction has a payload, containing the
data required to execute the desired part of the contract. This execution has a
fee that users have to pay to the network based on how complex is the code they
want to run (see figure~\ref{tb:eth:transaction}).

Ethereum's transaction-based smart contracts have changed the paradigm of modern
software development, since the priority when developing a smart contract is to
reduce the transaction costs of each interaction~\cite{delmolino2016step}.


\section{State of the art}


Decentralized alternatives, in spite of their
promises~\cite{bartlingblockchain}, are still in their infancy. A few proposals,
none of them functional to date, have appeared recently.

One of them is a peer review proposal that tries to solve some of the peer
review socio-technical problems using cryptocurrencies~\cite{tennant2017multi}.
It needs a critical threshold of research community engagement, changing the
actual processes and platforms, to start being implemented.

Blockchain-enabled apps have also been proposed, with voting and storage of
publications. This is the case of Aletheia~\cite{morton2017aletheia}, a software
for getting open access papers published. This platform idea aims to use
blockchain as a decentralized and distributed database as a publishing platform.

Peer review quality control through blockchain-based cohort
trainings~\cite{dhillon2016bench} have been also proposed, with the promise of
transparency and decentralization using a distributed ledger. Research labs can
use this training network to test their technology and reduce the risk for
private investment opportunities.

Finally, some of the off-chain journals are adapting to the demands of the
current scientific community like Ledger\footnote{https://ledgerjournal.org}, a
cryptocurrencies and blockchain-based journal that records the publication
timestamps in the Bitcoin blockchain.

\subsection{Reputation systems in academia}
\label{soa:rs}
\todo{SOA REP SYS EN ACADEMIA}

In Decentralized Science, two types of scoring systems have been considered: the
first is a 5-star scoring system, in which users score 1 to 5 for other users, 1
being bad reputation and 5 good reputation~\cite{} This reputation system is
used in many e-commerce websites. The second one is the like/dislike, in which
the interactions are evaluated with positive evaluation or negative evaluation,
depending on the interaction that two users have had~\cite{}.

This system has been extended to many internet services in which there is no
centralized entity in which users have to rely, so building a fully distributed
system requires trust between users that can offer a good reputation system. The
section \ref{soa:rs} explores different systems that exist today and ways to
improve them.

This paper proposes the development of a decentralized publication system for
open science. It aims to challenge the technical infrastructure that supports
the middlemen role of traditional publishers. Due to the successes of the Open
Access movement, some of the scientific knowledge is today freely provided by
the publishers. However, the content is still mostly served from their
infrastructure (i.e. servers, web platforms). This ownership of the
infrastructure gives them a position of power over the scientific community
which produces the contents~\cite{fuster2010governance}. Such central and
oligopolistic position in science dissemination allows them to impose policies
(e.g. copyright ownership, Open Access prices) and concentrate profits.

The proposed system aims to move the infrastructure control from the publishers
to the scientific community. It entails the decentralization of three essential
functions of science dissemination: 1) the peer review process, 2) the selection
and recognition of peer reviewers, and 3) the distribution of scientific
knowledge. The following section provides an overview of the system features,
while the final section discusses its challenges.


%%%
%%% Local Variables:
%%% mode: latex
%%% TeX-master: "../Tesis.tex"
%%% End:
