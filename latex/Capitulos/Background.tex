\chapter{Background and State of the art}

\begin{FraseCelebre}
  \begin{Frase}
    The future is already here – it's just not evenly distributed.
  \end{Frase}
  \begin{Fuente}
    William Gibson
  \end{Fuente}
\end{FraseCelebre}

% -------------------------------------------------------------------
% \section{Cooler Section}
% -------------------------------------------------------------------

\section{Socio-cultural background}
\label{scb}
\subsection{Publication systems}
\label{scb:ps}
The methodology of scientific publications creation was established in
1620~\cite{bacon1878novum}, in which Francis Bacon established certain steps to
elaborate what we know today as scientific papers. But it was 45 years later, in
1665, when appeared the considered first scientific journal: \emph{Philosophical
  Transactions of the Royal Society}~\cite{kronick1976history}. At that time,
editors were the ones who had to carry out the revision of the papers that would
be published in these journals.

It was around 100 years later when an alternative system was adopted, instead of
editors doing the work of reviewing all the papers, this would be done by a
group of experts in a certain field, deciding if each paper reviewed is good
enough to be published or not. This is the beginning of the process known today
as ``peer review''~\cite{spier2002history}.

But scientific publication as we know it today was settled down in the 19th
century, with the actual peer review process~\cite{burnham1990evolution},
stablishing the guidelines of the paper-based paradigm that we have in science
nowadays.

Determining the quality of a scientific paper is difficult, but today we have
different ways to do so. To be able to estimate this quality, usually there are
two approaches: before and after publishing.

\textbf{The peer review process} consists in a group of researchers in a certain
field that evaluate a paper's quality, commonly implying it's eligibility to be
published. They read the paper, and submit a review and an ``acceptance score''
representing if they think the paper should be accepted, have a revision or be
directly rejected. Normally these researchers are unknown to the authors and the
reviews are made anonymously (\emph{blind review}). Sometimes reviewers also do
not know who the authors are (\emph{double blind review}). And in rare cases,
both the authors and the reviewers are public (\emph{open review}). Therefore it
could be considered that this process is an indicator of the quality of a paper
before being published~\cite{szklo2006quality}.

Another way to determine this quality is after it is published, obtaining the
number of citations it gets over time~\cite{redner1998popular}. Normally a paper
with a higher number of citations is considered to have better quality than a
paper with lower number of citations. This way of analyzing quality gives rise
to the generation of indexes that determine how good a researcher is. For
example, the H index that assigns a numerical value to each researcher based on
the articles she has written and the citations he has
received~\cite{bornmann2007we}. Regarding these metrics, several alternatives
have been explored, since determining whether a researcher is good or not
through a number is not always correct~\cite{bornmann2008there}. However, in
most cases, it is possible to predict through this index if the quality of
future researcher papers will have good quality or not~\cite{hirsch2007does}.

The H index is one of the most used factors to measure the impact that has a
researcher. There are a series of similar indexes not to measure the impact of a
researcher, but the impact of a scientific journal. Also resorting to the
citations rate, journals size their quality with these indexes. JCR (journal
citations rate)~\cite{doi:10.1001/jama.295.1.90} is one of the most widely used
and recognized in the scientific community (see chapter~\ref{intro}).
Researchers who seek to get their work recognized should publish papers in
journals with high JCR.

Therefore, today's scientific research system is based on publishing papers in
journals and conferences, as the papers of an academic research that are
published in any other medium are part of the called ``grey literature'', that
usually are not taken into account in the research
process~\cite{rothstein2009grey}.

People looking to make a career in the academic world should bear in mind that
one of the priorities is to publish the maximum number of articles in
high-impact journals, as discussed above. In Spain for example, entities such as
the ANECA~\footnote{www.aneca.es/}, determine if a person can teach at the
university based on a number of factors. One of the most important is the number
of papers published in journals with high JRC that person has. In this way a
need is generated, which certain actors take advantage of within this system.

\bb{Publishers} were in charge of producing, printing and distributing the
editions of the journals when they began to emerge. Many major publishers such
as Willey-Blackwell\footnote{https://www.wiley.com},
Elsevier\footnote{https://www.elsevier.com/} or
Springer\footnote{http://www.springer.com/} have been around since the beginning
of the 19th century. Nevertheless, the digital age has meant that copying a
document, which used to be an expensive process, now has a very low cost. Even
so, publishers continue to profit from this system, acting as intermediaries
between the people who create science and those who consume
it~\cite{lariviere2015oligopoly}. In the era in which the replication of
information is not a cost, the role of publishers can be questioned and migrated
to fairer and more honest scientific publication and dissemination systems.

\subsection{Alternative publication systems}
\label{soa:aps}

Publication systems, as seen on the previous section, form an oligopoly of a
few, concentrating the benefits of the industry. However, there are some
attempts to change this paradigm on behalf of science dissemination.

\bb{Open Access} is a concept referring to all the research material that is
free of cost for the readers. This concept also involves various conditions for
something to be considered ``open access''~\cite{bailey2007open}:

\begin{itemize}
  
  \itbf{Freely accessible online:} Any user can get a copy of the paper online
  without any cost, including reviewed papers and unreviewed papers (preprints).
  
  \itbf{Lax copyrights:} Any user can download, print, search, copy and link to
  the full text or any part of these articles.

  \itbf{Public support:} Everyone should be able to access to all the material
  involving the paper, and should be uploaded to a public repository supported
  by an academic institution or other well-established organization that seeks
  to enable open access.
\end{itemize}

But as mentioned in the chapter~\ref{intro}, in most \emph{open-access}
journals, the costs lie with the authors rather than the
readers~\cite{lariviere2015oligopoly,van2013true}, making unaffordable for some
researchers the possibility to publish in these journals.

\bb{Open journal systems}~\cite{willinsky2005open} is an open software designed
to facilitate the publishing process. This project was created by the Public
Knowledge Project\footnote{https://pkp.sfu.ca/about/} and it targets open-access
online journals that want to speed up the publication processes. The system
provides tools to control the whole publishing process from paper submission,
through peer reviewing to the final publication issue.

\bb{Mega-journals}~(or Multi-journals)~\cite{binfield2013open,wellen2013open}
combine multiple journals into a single journal, allowing the publication of
open-access papers, which have gone through a peer review process. The first
journal to adopt this idea was the \emph{PLOS ONE}
Journal\footnote{http://journals.plos.org/plosone/} as of the project
\emph{Public Library of Science}. This project aims to create a library of
scientific journals under the values of open access and creative commons
licenses. As a result of the success of the \emph{PLOS ONE} journal, other
publishers have started their own mega-journals. Featuring alternative impact
metrics, reusability of figures and data, post-publication discussions and
portable reviews from other journals~\cite{bjork2015have}.

The \bb{continuous publication} model is based on publishing individual papers
migrating from the previous issue-based model~\cite{anderton2013continuous}.
This method is seen as an altenative for open-access journals as it speeds up
the publication process~\cite{haymanview}. \emph{Decentralized Science} adopts
this model by design (explained in section \ref{arch:sca}) as the platform
automatically publishes papers that meet certain preconditions, such as getting
at least two out of threee possitive reviews.

\bb{Preprints} are scientific papers that have not yet gone through the peer
review process~\cite{harnad2003electronic}. Formerly, the preprints that were
sent to the journals were private, and only accessible by the editors and
assigned reviewers. Nevertheless, nowadays it is common to publish a preprint
before sending it to a journal, uploading it to specialized platforms like
arXiv\footnote{https://arxiv.org/} or
Preprints\footnote{https://www.preprints.org/}~\cite{brown2001volution}.
Moreover, there is a correlation between the upload of a preprint and early
citations after the publication of the paper~\cite{shuai2012scientific}. This
system is a possible solution to the cold-start problem that papers of new
researchers who enter the academic career have~\cite{sugiyama2010scholarly}.

Social networks have also made a dent in the academic world, creating platforms
to contact other researchers and encouraging them to share their papers. These
platforms allow users to have a public profile, in which they can add relevant
information about their career: research fields, interests, papers published,
etc. The users also have the possibility to connect with other users and add
them as contacts or fellow researchers. In addition, these platforms allow it's
users to upload scientific papers (reviewed or unreviewed), granting the
possibility to share them through the platform's community. Some of the
well-known are Research Gate\footnote{https://www.researchgate.net},
Mendeley\footnote{https://www.mendeley.com} or
Academia\footnote{http://academia.edu}. But despite the good intentions of the
creators of these platforms, many of the journals demand the copyright of the
papers they publish, preventing the authors from sharing them through these
services.

In the current scientific publication process, an important part of the work is
done by reviewers, who remain anonymous for both the authors and the rest of the
scientific community. Decentralized Science proposes, among other things, to
make a reputation system of reviewers so that they can obtain the credit and
recognition of carrying out good reviews, and be penalized otherwise.

\subsection{Reputation systems}
\label{scb:rs}

\bb{Trust} in an individual is hard to acquire. Normally trust can not be
bought, sold or exchanged. When choosing whether to trust someone or not, having
the opinion of other people about that individual can be helpful. If these
opinions about each individual are stored in a system which can be consulted by
anyone, it is called a ``reputation system''.

A reputation system is a technology that allows the users who use it to trust
third parties inside the same system, without prior knowledge of each other. The
reputation system collects, adds and distributes the comments received from the
behavior of the participants, based on interactions with each other. This idea
was born in the early 2000s in which the use of the internet spread throughout
the world, and systems were needed to be able to trust
strangers~\cite{resnick2000reputation}.

The ``reputation'' is normally a value that indicates how much confidence the
community has in a user. This reputation is gained through interactions with the
rest of the people who use the service or platform that implements the
reputation system. Sometimes these platforms offer \emph{privileges} to people
who have a certain level of reputation, unlocking certain actions that can only
be performed at a certain threshold.

The basic idea of reputation systems is to give users the possibility to rate
the interactions that occur between them. One of the pioneering platforms to
implement this system was Ebay \footnote{https://ebay.com}, a second-hand buying
and selling website. In it, each vendor had a reputation based on whether
previous transactions had been honest or not. Each user who purchased products
from that seller had the opportunity to rate the purchase. In this way, all
those vendors who tried to rip off users immediately received a bad reputation.

Many of the large internet communities such as
Stackexchange\footnote{https://stackexchange.com/} or
reddit\footnote{https://www.reddit.com/} have their own reputation system.
Reputation systems behavior may vary depending on the
platform~\cite{josang2002beta}, but the most usual is the one where users get a
score based on certain interaction with the community.

Reputation systems also have a wide niche in e-commerce webs such as Ebay, as
mentioned before, or Amazon\footnote{https://amazon.com}, in which people pay
for a product sold by an unknown vendor. There must be a previous trust in the
vendor before buying any product, so a reputation system offers a score given by
other users that encourages you to trust or not that certain
seller~\cite{resnick2002trust}.

These systems vary widely in scope, such as one for peer-to-peer
computing~\cite{zhou2007powertrust}, vehicle ad-hoc~\cite{dotzer2005vars}, web
services~\cite{moore2008reputation} and even Wikipedia~\cite{adler2007content}.
All of them are based on an exchange of trust between users who use these
services.

This same concept was intended to be implemented using a token as a \bb{trust
  unit}, which users exchanged as a sign of trust deposits among
them~\cite{sharples2016blockchain}.

Nevertheless, reputation systems also have problems when it comes to defend the
users from attacks to individuals~\cite{hoffman2009survey} and unfair ratings
~\cite{whitby2004filtering}. In the other hand, being implemented in the
blockchain implies that all interactions are public and auditable. Everyone can
see all votes and ratings, so it is a system in which there is no anonymity.
This could allow dissuading the mentioned problems since the users who carry out
unjust directing or rating attacks are exposed publicly in the network.
Nevertheless, this also raises concerns about privacy and anonymity.

Nowadays the most widely used reputation systems are based on a five-star
rating~\cite{kinateder2003architecture}. This reputation system is present in
many online platforms in some of the mentioned before and others like
Tripadvisor\footnote{https://www.tripadvisor.com/},
AliExpress\footnote{https://aliexpress.com/}, Google
Play\footnote{https://play.google.com/} and other large platforms in which
products and services are voted by users.

\section{Technical background}
\label{tb}
\subsection{Network Architectures}
\label{tb:na}
\figura{architectures.png}{width=0.95\linewidth}{tb:na:diagram}{Digram of the
  diferent network architectures}

Most of the services that we use everyday are on the internet. All the users or
these platforms need to be connected to the network in order to use them. These
services are usually offered by centralized entities to which users have to
connect in order to access them. On the other hand, there are also other
platforms that offer their services without centralizing all the traffic in the
same site. For each one of these services, there are different architectures,
which can be divided into three different groups (see
figure~\ref{tb:na:diagram}~\cite{baran1964distributed}):

\begin{itemize}
  \itbf{Centralized Architecture:} All the structure is managed and controlled
  by a single node. This structure is vulnerable, meaning that attacks to the
  central node can compromise the entire network~\cite{baran1964distributed}.
  This type of architecture is used nowadays in web services in which all of its
  users have to connect to a centralized server. Some of the most used internet
  platforms like Facebook\footnote{https://www.facebook.com/},
  Github\footnote{https://github.com/} or
  Amazon\footnote{https://www.amazon.com/} use this architecture.
  
  \itbf{Decentralized Architecture:} This type of structure shows a hierarchical
  structure to a set of \emph{starts} connected in the form of a larger star
  with an additional link forming a loop. Each \emph{star} is managed by its
  community. Attacks to a single \emph{star} can suppose a loss of communication
  between nodes but does not compromise the entire
  system~\cite{baran1964distributed}. Besides, all its users can access all the
  data stored in all the nodes. This infrastructure is used in platforms like
  GNU Social\footnote{https://www.gnu.org/software/social/},
  Buddycloud\footnote{https://buddycloud.com/} and
  Diaspora\footnote{https://diasporafoundation.org/} among others.

  \itbf{Distributed Architecture:} This is a fully decentralized architecture
  and its usually called ``Peer to peer'' or ``P2P''. All the nodes (or
  \emph{peers}) share part of its own resources, and these shared resources are
  necessary to provide the service and content offered by the network. They are
  accessible by other peers directly, without intermediary
  entities~\cite{schollmeier2001definition}. All the data is spread among the
  nodes which have the same privileges in the structure. Users can control its
  contribution to the network and everyone can join or leave at any time without
  affecting the network architecture. Today this structure is used in platforms
  like BitTorrent\footnote{www.bittorrent.com/},
  Bitcoin\footnote{https://bitcoin.org/} and
  Ethereum\footnote{https://www.ethereum.org/}
\end{itemize}

Despite the disadvantages, most sites choose a centralized structure since in
many cases they are controlled by private companies. However, distributed
architectures are increasingly being used by new technologies since, as
explained above, they offer more advantages than centralized ones. One of the
clear examples of the use of this architecture is that of cryptocurrencies.

\subsection{Digital Currencies and Cryptocurrencies}
\label{tb:cryptos}

In the 80s and 90s, with the expansion of the world wide web, several services
began to be online (stores, newspapers, etc). Some of these services required
monetary transactions through the internet to be used so at this time, the
concepts of electronic money or digital currencies began to emerge. One of the
first forms of digital payment was Ecash~\cite{chaum1995introduction} in 1995,
an idea of a digital currency that offered a high level of privacy.
Unfortunately at that time the use of the internet was not so widespread and the
idea did not work.

Three years later, in 1998, Wei Dai published his proposal for electronic money
called B-money~\cite{dai1998b} an "anonymous, distributed electronic cash
system", a revolutionary idea for the time, since it added the concept of
anonymity to monetary transactions. B-money added a layer of cryptography to
secure these transactions that is why these ideas began to be called
``cryptocurrencies''.

Hashcash~\cite{back2002hashcash} was another digital currency idea that was born
in 2002 and introduced the concept of \emph{proof of work}. A \emph{proof of
  work} is a piece of data which is difficult to generate, because it is time
consuming or have high cost, but easy for others to verify. Hashcash used
\emph{Hash Functions} with certain requirements to encrypt transactions of money
between users.

Despite the fact that these projects were never carried out, they inspired
others that would come later such as Nick Szabo's Bit gold~\cite{szabo2008bit} a
similar approach as B-money but using P2P protocols and
PRPOW~\cite{finney2005rpow} an invention by Hal Finney intended as a prototype
for a digital cash.

None of these currencies was implemented successfully enough until 4 years later
with the arrival of the most widely used cryptocurrency today: \bb{Bitcoin}.

\subsection{Bitcoin}
\label{tb:bc}

\figura{transaction.png}{width=0.9\linewidth}{tb:bc:transaction}{The change of
  state of Bitcoin accounts done by transactions}

In 2009, the idea of a \bb{fully decentralized currency} first emerged, when
Satoshi Nakamoto, a fictional identity, published the first version of
Bitcoin~\cite{nakamoto2008bitcoin}. Decentralizing a currency implied that there
was no entity that controlled it, not even banks, governments or companies.
Bitcoin is controlled by all users who use it, since it uses a P2P network in
which everyone can see and verify all money transactions. This technology used
concepts and ideas of previous digital currencies such as B-money, Hashcash and
Bit gold (see section~\ref{tb:cryptos}).

In order to understand this technology, two important concepts must be defined:
addresses and transactions.

Each user who wants to use this cryptocurrency, must first have an account, a
randomly generated text string called \bb{address}. Each address can be
generated with no cost and user can have multiple addresses. These addresses act
as an identifier inside the Bitcoin's database and are used to transfer money.

Money transfers between two \ii{addresses} are called \bb{transactions}. Each
transaction is a small data fragment containing the source address, the
destination address and the amount of money transferred.

Bitcoin's users can observe all \ii{transactions} between all \ii{addresses}. In
order to do so, this technology stores a certain number of transactions into a
data fragment, called block. Next, this block is linked cryptographically to the
previous one containing all the foregoing transactions, forming a chain of
interlinked data called \ii{blockchain} (see figure~\ref{tb:bc:transaction}). .
To sum up, Bitcoin's blockchain is a large chain of data blocks containing all
the transactions between all accounts since the beginning of this technology.

Bitcoin's transactions must be verified by all peers, meaning that there has to
be a consensus among all the nodes for a transaction to be completed. This
solved the double-spending problem by which a dishonest actor may try to spend
twice the same coin in decentralized currency systems~\cite{chohan2017double}.

Bitcoin also introduces incentives to maintain the security of this ledger, both
rewarding nodes that contribute computational power for the security of the
network, and requiring at least half of the computing power of the network to
alter the state of the blockchain (thus, the blockchain is secure if at least
half of the computing power is provided by honest peers).

To deepen more into Bitcoin's inner functioning, this technology injects money
into the network using the idea of \emph{proof of work} or POW.

POW consists in all users competing with each other to encrypt a data block with
certain restrictions. The winner notifies its neighbors with the data block
successfully encrypted and gets its reward in money~\cite{barber2012bitter}.

Specifically, to encrypt a block, peers need to gather a number of pending
transactions into a data structure. Next, they add a random number (called
``nonce'') to that structure so that when performing a hash function of the
whole block has a certain number of leading zeros. Once a block is created, it
is used as reference to the next block of data, adding the hash of the previous
block to the next one~\cite{antonopoulos2014mastering} (see
figure~\ref{tech:sec:eth:bc:diagram}).

\figura{blockchain.png}{width=0.9\linewidth}{tech:sec:eth:bc:diagram}{Blockchain
  representation diagram}

This system makes it practically impossible to falsify a transaction in the
blockchain, since all nodes need to reach consensus. However, this system is
still vulnerable to attacks aimed at specific users, such as man in the middle
attacks~\cite{moore2013beware}.

This technology enabled a new wave of decentralization of applications such as
domain name registries~\cite{benshoof2016distributed} or microblogging platforms
\cite{freitas2013twister}. A second wave of blockchain based decentralization
was started by Ethereum~\cite{buterin2014ethereum}, as described below.

\subsection{Ethereum}
\label{tb:eth}
Ethereum~\cite{buterin2014ethereum} is a very novel technology that allows the
creation of distributed applications that run in an arbitrary large and
trust-less network of nodes. Ethereum is based in the Bitcoin's blockchain
technology, a public database where everyone can watch all transactions.

Using this idea, Ethereum uses its own blockchain (see section~\ref{tb:bc}) to
deploy and execute fragments of code in a distributed network. This fragments of
code are called ``smart contracts'' and they are uploaded to the blockchain in
order to be executed.

Working very similar to Bitcoin's blockchain, peers have to reach consensus in
each smart contract execution, making a smart contract source code almost
impossible to hack.

Apart from this, Ethereum uses its own cryptocurrency called ``Ether''. This
currency not only works like Bitcoin, to exchange money between users, but to
\emph{fuel} the smart contracts' code execution, running its inner functions for
a small amount of \ii{Ether}.

\subsubsection*{Smart Contracts}
\label{tb:cryptos:sm}
\figura{cryptomap.png}{width=0.7\linewidth}{criptomap}{Existing cryptocurrencies
  represented by purchase-sale
  volume~\footnote{https://cryptocoincharts.info/coins/graphicalComparison}.}

Smart contracts are written in a programming language called
Solidity\footnote{http://solidity.readthedocs.io/en/develop/} provided by the
Ethereum's developers. This language is called contract-oriented and it was
influenced by C++, Python and JavaScript. Solidity offers the possibility to
create a wide range of decentralized applications in the blockchain in which
users do not have to trust a centralized organization.

These contracts also have the capacity to store an transfer money, making them
the perfect tool to implement a wide range of decentralized applications like:
gambling games~\cite{piasecki2016gaming}, voting
systems~\cite{mccorry2017smart}, crowdfunding~\cite{jacynycz2016betfunding},
prediction markets~\cite{peterson2015augur}, transparency
systems~\cite{bonneau2016ethiks} and so on. Today it is the most exchanged
currency within cryptocurrencies (see figure~\ref{criptomap}.

Smart contracts offer us a framework to design distributed platforms like the
one proposed in this work, in which all transactions that interact with the
scientific publication process can be cryptographically verified (see
section~\ref{sec:second-prototype}).

\subsubsection*{Addresses and Transactions}
\label{ts:at}

As in Bitcoin's network, Ethereum's user accounts are called \bb{addresses} and
the transfers between them are called \bb{transactions}. The main difference
between Ethereum and its predecessor is that an address can also be assigned to
a smart contract. Thus, there are two types of addresses within the Ethereum
network:

\begin{itemize}

\item \textbf{Personal addresses:} These are the addresses of the users who want
  to interact with the Ethereum network. Each one has its address and a balance.
  
\item \textbf{Contract addresses:} Once a smart contract is deployed in the
  Ethereum blockchain, the network generates an address and assigns it to the
  deployed contract. It contains a pointer to the source code, a balance with
  the available money, and its own internal memory where it saves all the
  contract's information.

\end{itemize}

Each of the addresses within the Ethereum network is unique, so they have
several useful applications when it comes to develop an smart contract.
Addresses can be used to identify who executes a contract as well as to restrict
the use of certain contracts to a list of allowed addresses. Listing
\ref{contrxample} shows an example code of a very simple smart contact. This
example contains comments to explain the address restrictions mentioned before.

It should be noted that these address restrictions can also be applied to
contract addresses, and may limit the execution of a certain fragment to an
address associated with another contract.

\begin{minipage}{\linewidth}
\begin{lstlisting}[language=C++,commentstyle=\color{olive}\ttfamily,frame=single,caption=Example
  of an smart contract,label=contrxample,captionpos=b]

  contract Example{
    
    // The owner of the contract
    address owner;
    
    constructor() public{
        // The contract assigns the owner to 
        // the account that creates the contract
        owner = msg.sender;
    }
    
    function doSomethingPublic() public{
      // This function can be called
      // once the contract is deployed
    }
    
    function doSomethingRestricted() {
       if ( msg.sender != owner ){
            // If the address that calls this function is not
            // the owner the transaction is cancelled
            revert();
        }
        else{
            // This function can be called only by the owner
        }
    }
}
  
  \end{lstlisting}
  \end{minipage}

Smart contracts behavior is transaction-based~\cite{wood2014ethereum}. Once a
contract is deployed in the blockchain, users (or other contracts) may send
transactions to run specific functions in its code. Each transaction has a
payload, containing the data required to execute the desired function. This
execution has a fee that users have to pay to the network based on how complex
is the code they want to run (see figure~\ref{tb:eth:transaction}).

Ethereum's transaction-based smart contracts have changed the paradigm of modern
software development, since the priority when developing a smart contract is to
reduce the transaction costs of each interaction~\cite{delmolino2016step}.


\section{State of the art}
\label{sec:state-art}

As mentioned in the previous sections, scientific publication process has
undergone few changes in the last centuries. Nevertheless, journals and
conferences are trying to adapt the submission systems to the internet era,
using software to manage the process from submission to publication.



Decentralized alternatives, in spite of their
promises~\cite{bartlingblockchain}, are still in their infancy. A few proposals,
none of them functional to date, have appeared recently.

One of them is a peer review proposal that tries to solve some of the peer
review socio-technical problems using cryptocurrencies~\cite{tennant2017multi}.
It needs a critical threshold of research community engagement, changing the
actual processes and platforms, to start being implemented.

Blockchain-enabled apps have also been proposed, with voting and storage of
publications. This is the case of Aletheia~\cite{morton2017aletheia}, a software
for getting open access papers published. This platform idea aims to use
blockchain as a decentralized and distributed database as a publishing platform.

Peer review quality control through blockchain-based cohort
trainings~\cite{dhillon2016bench} have been also proposed, with the promise of
transparency and decentralization using a distributed ledger. Research labs can
use this training network to test their technology and reduce the risk for
private investment opportunities.

Finally, some of the off-chain journals are adapting to the demands of the
current scientific community like Ledger\footnote{https://ledgerjournal.org}, a
cryptocurrencies and blockchain-based journal that records the publication
timestamps in the Bitcoin blockchain.


In Decentralized Science, two types of scoring systems have been considered: the
first is a 5-star scoring system, in which users score 1 to 5 for other users, 1
being bad reputation and 5 good reputation~\cite{} This reputation system is
used in many e-commerce websites. The second one is the like/dislike, in which
the interactions are evaluated with positive evaluation or negative evaluation,
depending on the interaction that two users have had~\cite{}.

This system has been extended to many internet services in which there is no
centralized entity in which users have to rely, so building a fully distributed
system requires trust between users that can offer a good reputation system. The
section \ref{soa:rs} explores different systems that exist today and ways to
improve them.

This paper proposes the development of a decentralized publication system for
open science. It aims to challenge the technical infrastructure that supports
the middlemen role of traditional publishers. Due to the successes of the Open
Access movement, some of the scientific knowledge is today freely provided by
the publishers. However, the content is still mostly served from their
infrastructure (i.e. servers, web platforms). This ownership of the
infrastructure gives them a position of power over the scientific community
which produces the contents~\cite{fuster2010governance}. Such central and
oligopolistic position in science dissemination allows them to impose policies
(e.g. copyright ownership, Open Access prices) and concentrate profits.

The proposed system aims to move the infrastructure control from the publishers
to the scientific community. It entails the decentralization of three essential
functions of science dissemination: 1) the peer review process, 2) the selection
and recognition of peer reviewers, and 3) the distribution of scientific
knowledge. The following section provides an overview of the system features,
while the final section discusses its challenges.


%%%
%%% Local Variables:
%%% mode: latex
%%% TeX-master: "../Tesis.tex"
%%% End:
