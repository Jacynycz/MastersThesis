%---------------------------------------------------------------------
%
%                      agradecimientos.tex
%
%---------------------------------------------------------------------
%
% agradecimientos.tex
% Copyright 2009 Marco Antonio Gomez-Martin, Pedro Pablo Gomez-Martin
%
% This file belongs to the TeXiS manual, a LaTeX template for writting
% Thesis and other documents. The complete last TeXiS package can
% be obtained from http://gaia.fdi.ucm.es/projects/texis/
%
% Although the TeXiS template itself is distributed under the
% conditions of the LaTeX Project Public License
% (http://www.latex-project.org/lppl.txt), the manual content
% uses the CC-BY-SA license that stays that you are free:
%
%    - to share & to copy, distribute and transmit the work
%    - to remix and to adapt the work
%
% under the following conditions:
%
%    - Attribution: you must attribute the work in the manner
%      specified by the author or licensor (but not in any way that
%      suggests that they endorse you or your use of the work).
%    - Share Alike: if you alter, transform, or build upon this
%      work, you may distribute the resulting work only under the
%      same, similar or a compatible license.
%
% The complete license is available in
% http://creativecommons.org/licenses/by-sa/3.0/legalcode
%
%---------------------------------------------------------------------
%
% Contiene la pígina de agradecimientos.
%
% Se crea como un capítulo sin numeraciín.
%
%---------------------------------------------------------------------

\chapter{Agradecimientos}

\cabeceraEspecial{Agradecimientos}

\begin{FraseCelebre}
\begin{Frase}
I find your lack of faith disturbing\end{Frase}
\begin{Fuente}
Darth Vader, Star Wars: A New Hope.
\end{Fuente}
\end{FraseCelebre}

Groucho Marx decía que encontraba a la televisiín muy educativa porque
cada vez que alguien la encendía, íl se iba a otra habitaciín a leer
un libro. Utilizando un esquema similar, nosotros queremos agradecer
al Word de Microsoft el habernos forzado a utilizar \LaTeX. Cualquiera
que haya intentado escribir un documento de mís de 150 píginas con
esta aplicaciín entenderí a quí nos referimos. Y lo decimos porque
nuestra andadura con \LaTeX\ comenzí, precisamente, despuís de
escribir un documento de algo mís de 200 píginas. Una vez terminado
decidimos que nunca mís pasaríamos por ahí. Y entonces caímos en
\LaTeX.

Es muy posible que hubíeramos llegado al mismo sitio de todas formas,
ya que en el mundo acadímico a la hora de escribir artículos y
contribuciones a congresos lo mís extendido es \LaTeX. Sin embargo,
tambiín es cierto que cuando intentas escribir un documento grande
en \LaTeX\ por tu cuenta y riesgo sin un enlace del tipo ``\emph{Author
  instructions}'', se hace cuesta arriba, pues uno no sabe por donde
empezar.

Y ahí es donde debemos agradecer tanto a Pablo Gervís como a Miguel
Palomino su ayuda. El primero nos ofrecií el cídigo fuente de una
programaciín docente que había hecho unos aíos atrís y que nos sirvií
de inspiraciín (por ejemplo, el fichero \texttt{guionado.tex} de
\texis\ tiene una estructura casi exacta a la suya e incluso puede
que el nombre sea el mismo). El segundo nos dejí husmear en el cídigo
fuente de su propia tesis donde, ademís de otras cosas mís
interesantes pero menos curiosas, descubrimos que aín hay gente que
escribe los acentos espaíoles con el \verb+\'{\i}+.

No podemos tampoco olvidar a los numerosos autores de los libros y
tutoriales de \LaTeX\ que no sílo permiten descargar esos manuales sin
coste adicional, sino que tambiín dejan disponible el cídigo fuente.
Estamos pensando en Tobias Oetiker, Hubert Partl, Irene Hyna y
Elisabeth Schlegl, autores del famoso ``The Not So Short Introduction
to \LaTeXe'' y en Tomís Bautista, autor de la traducciín al espaíol. De
ellos es, entre otras muchas cosas, el entorno \texttt{example}
utilizado en algunos momentos en este manual.

Tambiín estamos en deuda con Joaquín Ataz Lípez, autor del libro
``Creaciín de ficheros \LaTeX\ con {GNU} Emacs''. Gracias a íl dejamos
de lado a WinEdt y a Kile, los editores que por entonces utilizíbamos
en entornos Windows y Linux respectivamente, y nos pasamos a emacs. El
tiempo de escritura que nos ahorramos por no mover las manos del
teclado para desplazar el cursor o por no tener que escribir
\verb+\emph+ una y otra vez se lo debemos a íl; nuestro ocio y vida
social se lo agradecen.

Por íltimo, gracias a toda esa gente creadora de manuales, tutoriales,
documentaciín de paquetes o respuestas en foros que hemos utilizado y
seguiremos utilizando en nuestro quehacer como usuarios de
\LaTeX. Sabíis un montín.

Y para terminar, a Donal Knuth, Leslie Lamport y todos los que hacen y
han hecho posible que hoy puedas estar leyendo estas líneas.

\endinput
% Variable local para emacs, para  que encuentre el fichero maestro de
% compilaciín y funcionen mejor algunas teclas rípidas de AucTeX
%%%
%%% Local Variables:
%%% mode: latex
%%% TeX-master: "../Tesis.tex"
%%% End:
