
\documentclass[11pt,a4paper,twoside]{book}

%
% Definimos  el   comando  \compilaCapitulo,  que   luego  se  utiliza
% (opcionalmente) en config.tex. Quedar_a  mejor si tambi_n se definiera
% en  ese fichero,  pero por  el modo  en el  que funciona  eso  no es
% posible. Puedes consultar la documentaci_n de ese fichero para tener
% m_s  informaci_n. Definimos tambi_n  \compilaApendice, que  tiene el
% mismo  cometido, pero  que se  utiliza para  compilar  _nicamente un
% ap_ndice.
%
%
% Si  queremos   compilar  solo   una  parte  del   documento  podemos
% especificar mediante  \includeonly{...} qu_ ficheros  son los _nicos
% que queremos  que se incluyan.  Esto  es _til por  ejemplo para s_lo
% compilar un cap_tulo.
%
% El problema es que todos aquellos  ficheros que NO est_n en la lista
% NO   se  incluir_n...  y   eso  tambi_n   afecta  a   ficheros  de
% la plantilla...
%
% Total,  que definimos  una constante  con los  ficheros  que siempre
% vamos a querer compilar  (aquellos relacionados con configuraci_n) y
% luego definimos \compilaCapitulo.
\newcommand{\ficherosBasicosTeXiS}{%
TeXiS/TeXiS_pream,TeXiS/TeXiS_cab,TeXiS/TeXiS_bib,TeXiS/TeXiS_cover,%
TeXiS/TeXiS_part%
}
\newcommand{\ficherosBasicosTexto}{%
constantes,guionado,Cascaras/bibliografia,config%
}
\newcommand{\compilaCapitulo}[1]{%
\includeonly{\ficherosBasicosTeXiS,\ficherosBasicosTexto,Capitulos/#1}
}

\newcommand{\compilaApendice}[1]{%
\includeonly{\ficherosBasicosTeXiS,\ficherosBasicosTexto,Apendices/#1}
}

%---------------------------------------------------------------------
%
%                          config.tex
%
%---------------------------------------------------------------------
%
% Contiene la  definici�n de constantes  que determinan el modo  en el
% que se compilar� el documento.
%
%---------------------------------------------------------------------
%
% En concreto, podemos  indicar si queremos "modo release",  en el que
% no  aparecer�n  los  comentarios  (creados  mediante  \com{Texto}  o
% \comp{Texto}) ni los "por  hacer" (creados mediante \todo{Texto}), y
% s� aparecer�n los �ndices. El modo "debug" (o mejor dicho en modo no
% "release" muestra los �ndices  (construirlos lleva tiempo y son poco
% �tiles  salvo  para   la  versi�n  final),  pero  s�   el  resto  de
% anotaciones.
%
% Si se compila con LaTeX (no  con pdflatex) en modo Debug, tambi�n se
% muestran en una esquina de cada p�gina las entradas (en el �ndice de
% palabras) que referencian  a dicha p�gina (consulta TeXiS_pream.tex,
% en la parte referente a show).
%
% El soporte para  el �ndice de palabras en  TeXiS es embrionario, por
% lo  que no  asumas que  esto funcionar�  correctamente.  Consulta la
% documentaci�n al respecto en TeXiS_pream.tex.
%
%
% Tambi�n  aqu� configuramos  si queremos  o  no que  se incluyan  los
% acr�nimos  en el  documento final  en la  versi�n release.  Para eso
% define (o no) la constante \acronimosEnRelease.
%
% Utilizando \compilaCapitulo{nombre}  podemos tambi�n especificar qu�
% cap�tulo(s) queremos que se compilen. Si no se pone nada, se compila
% el documento  completo.  Si se pone, por  ejemplo, 01Introduccion se
% compilar� �nicamente el fichero Capitulos/01Introduccion.tex
%
% Para compilar varios  cap�tulos, se separan sus nombres  con comas y
% no se ponen espacios de separaci�n.
%
% En realidad  la macro \compilaCapitulo  est� definida en  el fichero
% principal tesis.tex.
%
%---------------------------------------------------------------------


% Comentar la l�nea si no se compila en modo release.
% TeXiS har� el resto.
% ���Si cambias esto, haz un make clean antes de recompilar!!!
\def\release{1}


% Descomentar la linea si se quieren incluir los
% acr�nimos en modo release (en modo debug
% no se incluir�n nunca).
% ���Si cambias esto, haz un make clean antes de recompilar!!!
%\def\acronimosEnRelease{1}


% Descomentar la l�nea para establecer el cap�tulo que queremos
% compilar

% \compilaCapitulo{01Introduccion}
% \compilaCapitulo{02EstructuraYGeneracion}
% \compilaCapitulo{03Edicion}
% \compilaCapitulo{04Imagenes}
% \compilaCapitulo{05Bibliografia}
% \compilaCapitulo{06Makefile}

% \compilaApendice{01AsiSeHizo}

% Variable local para emacs, para  que encuentre el fichero maestro de
% compilaci�n y funcionen mejor algunas teclas r�pidas de AucTeX
%%%
%%% Local Variables:
%%% mode: latex
%%% TeX-master: "./Tesis.tex"
%%% End:


% Paquete de la plantilla

\usepackage{TeXiS/TeXiS}
\usepackage[lmargin=4cm,rmargin=3cm]{geometry}
\linespread{1.2}\selectfont
\setcitestyle{square}


% Incluimos el fichero con comandos de constantes
%---------------------------------------------------------------------
%
%                          constantes.tex
%
%---------------------------------------------------------------------
%
% Fichero que  declara nuevos comandos LaTeX  sencillos realizados por
% comodidad en la escritura de determinadas palabras
%
%---------------------------------------------------------------------

%%%%%%%%%%%%%%%%%%%%%%%%%%%%%%%%%%%%%%%%%%%%%%%%%%%%%%%%%%%%%%%%%%%%%%
% Comando: 
%
%       \titulo
%
% Resultado: 
%
% Escribe el t�tulo del documento.
%%%%%%%%%%%%%%%%%%%%%%%%%%%%%%%%%%%%%%%%%%%%%%%%%%%%%%%%%%%%%%%%%%%%%%
\def\titulo{\textsc{TeXiS}: Una plantilla de \LaTeX\
  para Tesis y otros documentos}

%%%%%%%%%%%%%%%%%%%%%%%%%%%%%%%%%%%%%%%%%%%%%%%%%%%%%%%%%%%%%%%%%%%%%%
% Comando: 
%
%       \autor
%
% Resultado: 
%
% Escribe el autor del documento.
%%%%%%%%%%%%%%%%%%%%%%%%%%%%%%%%%%%%%%%%%%%%%%%%%%%%%%%%%%%%%%%%%%%%%%
\def\autor{Marco Antonio y Pedro Pablo G\'omez Mart\'in}

% Variable local para emacs, para  que encuentre el fichero maestro de
% compilaci�n y funcionen mejor algunas teclas r�pidas de AucTeX

%%%
%%% Local Variables:
%%% mode: latex
%%% TeX-master: "tesis.tex"
%%% End:
%también


\typeout{Copyright Marco Antonio and Pedro Pablo Gomez Martin}

%
% "Metadatos" para el PDF
%
\ifpdf\hypersetup{%
    pdftitle = {\titulo},
    pdfsubject = {Master's Thesis},
    pdfkeywords = {blockchain},
    pdfauthor = {\textcopyright\ \autor},
    pdfcreator = {\LaTeX\ con el paquete \flqq hyperref\frqq},
    pdfproducer = {pdfeTeX-0.\the\pdftexversion\pdftexrevision},
    }
    \pdfinfo{/CreationDate (\today)}
\fi


%- - - - - - - - - - - - - - - - - - - - - - - - - - - - - - - - - - -
%                        Documento
%- - - - - - - - - - - - - - - - - - - - - - - - - - - - - - - - - - -
\begin{document}

% Incluimos el  fichero de definici_n de guionado  de algunas palabras
% que LaTeX no ha dividido como deber_a
%----------------------------------------------------------------
%
%                          guionado.tex
%
%----------------------------------------------------------------
%
% Fichero con algunas divisiones de palabras que LaTeX no
% hace correctamente si no se le da alguna ayuda.
%
%----------------------------------------------------------------

\hyphenation{
% a
abs-trac-to
abs-trac-tos
abs-trac-ta
abs-trac-tas
ac-tua-do-res
a-gra-de-ci-mien-tos
ana-li-za-dor
an-te-rio-res
an-te-rior-men-te
apa-rien-cia
a-pro-pia-do
a-pro-pia-dos
a-pro-pia-da
a-pro-pia-das
a-pro-ve-cha-mien-to
a-que-llo
a-que-llos
a-que-lla
a-que-llas
a-sig-na-tu-ra
a-sig-na-tu-ras
a-so-cia-da
a-so-cia-das
a-so-cia-do
a-so-cia-dos
au-to-ma-ti-za-do
% b
batch
bi-blio-gra-fía
bi-blio-grá-fi-cas
bien
bo-rra-dor
boo-l-ean-expr
% c
ca-be-ce-ra
call-me-thod-ins-truc-tion
cas-te-lla-no
cir-cuns-tan-cia
cir-cuns-tan-cias
co-he-ren-te
co-he-ren-tes
co-he-ren-cia
co-li-bri
co-men-ta-rio
co-mer-cia-les
co-no-ci-mien-to
cons-cien-te
con-si-de-ra-ba
con-si-de-ra-mos
con-si-de-rar-se
cons-tan-te
cons-trucción
cons-tru-ye
cons-tru-ir-se
con-tro-le
co-rrec-ta-men-te
co-rres-pon-den
co-rres-pon-dien-te
co-rres-pon-dien-tes
co-ti-dia-na
co-ti-dia-no
crean
cris-ta-li-zan
cu-rri-cu-la
cu-rri-cu-lum
cu-rri-cu-lar
cu-rri-cu-la-res
% d
de-di-ca-do
de-di-ca-dos
de-di-ca-da
de-di-ca-das
de-rro-te-ro
de-rro-te-ros
de-sa-rro-llo
de-sa-rro-llos
de-sa-rro-lla-do
de-sa-rro-lla-dos
de-sa-rro-lla-da
de-sa-rro-lla-das
de-sa-rro-lla-dor
de-sa-rro-llar
des-cri-bi-re-mos
des-crip-ción
des-crip-cio-nes
des-cri-to
des-pués
de-ta-lla-do
de-ta-lla-dos
de-ta-lla-da
de-ta-lla-das
di-a-gra-ma
di-a-gra-mas
di-se-ños
dis-po-ner
dis-po-ni-bi-li-dad
do-cu-men-ta-da
do-cu-men-to
do-cu-men-tos
% e
edi-ta-do
e-du-ca-ti-vo
e-du-ca-ti-vos
e-du-ca-ti-va
e-du-ca-ti-vas
e-la-bo-ra-do
e-la-bo-ra-dos
e-la-bo-ra-da
e-la-bo-ra-das
es-co-llo
es-co-llos
es-tu-dia-do
es-tu-dia-dos
es-tu-dia-da
es-tu-dia-das
es-tu-dian-te
e-va-lua-cio-nes
e-va-lua-do-res
exis-ten-tes
exhaus-ti-va
ex-pe-rien-cia
ex-pe-rien-cias
% f
for-ma-li-za-do
% g
ge-ne-ra-ción
ge-ne-ra-dor
ge-ne-ra-do-res
ge-ne-ran
% h
he-rra-mien-ta
he-rra-mien-tas
% i
i-dio-ma
i-dio-mas
im-pres-cin-di-ble
im-pres-cin-di-bles
in-de-xa-do
in-de-xa-dos
in-de-xa-da
in-de-xa-das
in-di-vi-dual
in-fe-ren-cia
in-fe-ren-cias
in-for-ma-ti-ca
in-gre-dien-te
in-gre-dien-tes
in-me-dia-ta-men-te
ins-ta-la-do
ins-tan-cias
% j
% k
% l
len-gua-je
li-be-ra-to-rio
li-be-ra-to-rios
li-be-ra-to-ria
li-be-ra-to-rias
li-mi-ta-do
li-te-ra-rio
li-te-ra-rios
li-te-ra-ria
li-te-ra-rias
lo-tes
% m
ma-ne-ra
ma-nual
mas-que-ra-de
ma-yor
me-mo-ria
mi-nis-te-rio
mi-nis-te-rios
mo-de-lo
mo-de-los
mo-de-la-do
mo-du-la-ri-dad
mo-vi-mien-to
% n
na-tu-ral
ni-vel
nues-tro
% o
obs-tan-te
o-rien-ta-do
o-rien-ta-dos
o-rien-ta-da
o-rien-ta-das
% p
pa-ra-le-lo
pa-ra-le-la
par-ti-cu-lar
par-ti-cu-lar-men-te
pe-da-gó-gi-ca
pe-da-gó-gi-cas
pe-da-gó-gi-co
pe-da-gó-gi-cos
pe-rio-di-ci-dad
per-so-na-je
plan-te-a-mien-to
plan-te-a-mien-tos
po-si-ción
pre-fe-ren-cia
pre-fe-ren-cias
pres-cin-di-ble
pres-cin-di-bles
pri-me-ra
pro-ble-ma
pro-ble-mas
pró-xi-mo
pu-bli-ca-cio-nes
pu-bli-ca-do
% q
% r
rá-pi-da
rá-pi-do
ra-zo-na-mien-to
ra-zo-na-mien-tos
re-a-li-zan-do
re-fe-ren-cia
re-fe-ren-cias
re-fe-ren-cia-da
re-fe-ren-cian
re-le-van-tes
re-pre-sen-ta-do
re-pre-sen-ta-dos
re-pre-sen-ta-da
re-pre-sen-ta-das
re-pre-sen-tar-lo
re-qui-si-to
re-qui-si-tos
res-pon-der
res-pon-sa-ble
% s
se-pa-ra-do
si-guien-do
si-guien-te
si-guien-tes
si-guie-ron
si-mi-lar
si-mi-la-res
si-tua-ción
% t
tem-pe-ra-ments
te-ner
trans-fe-ren-cia
trans-fe-ren-cias
% u
u-sua-rio
Unreal-Ed
% v
va-lor
va-lo-res
va-rian-te
ver-da-de-ro
ver-da-de-ros
ver-da-de-ra
ver-da-de-ras
ver-da-de-ra-men-te
ve-ri-fi-ca
% w
% x
% y
% z
}
% Variable local para emacs, para que encuentre el fichero
% maestro de compilación
%%%
%%% Local Variables:
%%% mode: latex
%%% TeX-master: "./Tesis.tex"
%%% End:


% Marcamos  el inicio  del  documento para  la  numeraci_n de  p_ginas
% (usando n_meros romanos para esta primera fase).
\frontmatter


%---------------------------------------------------------------------
%
%                          configCover.tex
%
%---------------------------------------------------------------------
%
% cover.tex
% Copyright 2009 Marco Antonio Gomez-Martin, Pedro Pablo Gomez-Martin
%
% This file belongs to the TeXiS manual, a LaTeX template for writting
% Thesis and other documents. The complete last TeXiS package can
% be obtained from http://gaia.fdi.ucm.es/projects/texis/
%
% Although the TeXiS template itself is distributed under the
% conditions of the LaTeX Project Public License
% (http://www.latex-project.org/lppl.txt), the manual content
% uses the CC-BY-SA license that stays that you are free:
%
%    - to share & to copy, distribute and transmit the work
%    - to remix and to adapt the work
%
% under the following conditions:
%
%    - Attribution: you must attribute the work in the manner
%      specified by the author or licensor (but not in any way that
%      suggests that they endorse you or your use of the work).
%    - Share Alike: if you alter, transform, or build upon this
%      work, you may distribute the resulting work only under the
%      same, similar or a compatible license.
%
% The complete license is available in
% http://creativecommons.org/licenses/by-sa/3.0/legalcode
%
%---------------------------------------------------------------------
%
% Fichero que contiene la configuración de la portada y de la
% primera hoja del documento.
%
%---------------------------------------------------------------------


% Pueden configurarse todos los elementos del contenido de la portada
% utilizando comandos.

%%%%%%%%%%%%%%%%%%%%%%%%%%%%%%%%%%%%%%%%%%%%%%%%%%%%%%%%%%%%%%%%%%%%%%
% Tótulo del documento:
% \tituloPortada{titulo}
% Nota:
% Si no se define se utiliza el del \titulo. Este comando permite
% cambiar el tótulo de forma que se especifiquen dónde se quieren
% los retornos de carro cuando se utilizan fuentes grandes.
%%%%%%%%%%%%%%%%%%%%%%%%%%%%%%%%%%%%%%%%%%%%%%%%%%%%%%%%%%%%%%%%%%%%%%

\tituloPortada{
Blockchain-based reputation system for peer reviewing
}

%%%%%%%%%%%%%%%%%%%%%%%%%%%%%%%%%%%%%%%%%%%%%%%%%%%%%%%%%%%%%%%%%%%%%%
% Autor del documento:
% \autorPortada{Nombre}
% Se utiliza en la portada y en el valor por defecto del
% primer subtótulo de la segunda portada.
%%%%%%%%%%%%%%%%%%%%%%%%%%%%%%%%%%%%%%%%%%%%%%%%%%%%%%%%%%%%%%%%%%%%%%
\autorPortada{Viktor Jacynycz García}

%%%%%%%%%%%%%%%%%%%%%%%%%%%%%%%%%%%%%%%%%%%%%%%%%%%%%%%%%%%%%%%%%%%%%%
% Fecha de publicación:
% \fechaPublicacion{Fecha}
% Puede ser vacóo. Aparece en la óltima lónea de ambas portadas
%%%%%%%%%%%%%%%%%%%%%%%%%%%%%%%%%%%%%%%%%%%%%%%%%%%%%%%%%%%%%%%%%%%%%%
\fechaPublicacion{Curso 2017/2018}

%%%%%%%%%%%%%%%%%%%%%%%%%%%%%%%%%%%%%%%%%%%%%%%%%%%%%%%%%%%%%%%%%%%%%%
% Imagen de la portada (y escala)
% \imagenPortada{Fichero}
% \escalaImagenPortada{Numero}
% Si no se especifica, se utiliza la imagen TODO.pdf
%%%%%%%%%%%%%%%%%%%%%%%%%%%%%%%%%%%%%%%%%%%%%%%%%%%%%%%%%%%%%%%%%%%%%
\imagenPortada{Imagenes/Vectorial/escudoUCM}
\escalaImagenPortada{0.2}

%%%%%%%%%%%%%%%%%%%%%%%%%%%%%%%%%%%%%%%%%%%%%%%%%%%%%%%%%%%%%%%%%%%%%%
% Tipo de documento.
% \tipoDocumento{Tipo}
% Para el texto justo debajo del escudo.
% Si no se indica, se utiliza "TESIS DOCTORAL".
%%%%%%%%%%%%%%%%%%%%%%%%%%%%%%%%%%%%%%%%%%%%%%%%%%%%%%%%%%%%%%%%%%%%%%
\tipoDocumento{Master's Thesis}

%%%%%%%%%%%%%%%%%%%%%%%%%%%%%%%%%%%%%%%%%%%%%%%%%%%%%%%%%%%%%%%%%%%%%%
% Institución/departamento asociado al documento.
% \institucion{Nombre}
% Puede tener varias lóneas. Se utiliza en las dos portadas.
% Si no se indica apareceró vacóo.
%%%%%%%%%%%%%%%%%%%%%%%%%%%%%%%%%%%%%%%%%%%%%%%%%%%%%%%%%%%%%%%%%%%%%%
\institucion{
Máster en ingeniería informática \leavevmode \\[0.3em]
Masters degree in software engeenering \leavevmode \\[0.3em]
Facultad de Informática \leavevmode \\[0.3em]
Universidad Complutense de Madrid
}

%%%%%%%%%%%%%%%%%%%%%%%%%%%%%%%%%%%%%%%%%%%%%%%%%%%%%%%%%%%%%%%%%%%%%%
% Director del trabajo.
% \directorPortada{Nombre}
% Se utiliza para el valor por defecto del segundo subtótulo, donde
% se indica quión es el director del trabajo.
% Si se fuerza un subtótulo distinto, no hace falta definirlo.
%%%%%%%%%%%%%%%%%%%%%%%%%%%%%%%%%%%%%%%%%%%%%%%%%%%%%%%%%%%%%%%%%%%%%%
\directorPortada{Samer Hassan Collado and Antonio Sánchez Ruiz-Granados}

%                      %%%%%%%%%%%%%%%%%%%%%%%%%%%%%%%%%%%%%%%%%%%%%%%%%%%%%%%%%%%%%%%%%%%%%
% Texto del primer subtótulo de la segunda portada.
% \textoPrimerSubtituloPortada{Texto}
% Para configurar el primer "texto libre" de la segunda portada.
% Si no se especifica se indica "Memoria que presenta para optar al
% tótulo de Doctor en Informótica" seguido del \autorPortada.
%%%%%%%%%%%%%%%%%%%%%%%%%%%%%%%%%%%%%%%%%%%%%%%%%%%%%%%%%%%%%%%%%%%%%%
%\textoPrimerSubtituloPortada{%
%\textit{Informe técnico del departamento}  \\ [0.3em]
%\textbf{Ingeniería del Software e Inteligencia Artificial} \\ [0.3em]
%\textbf{IT/2009/3}
%}

%%%%%%%%%%%%%%%%%%%%%%%%%%%%%%%%%%%%%%%%%%%%%%%%%%%%%%%%%%%%%%%%%%%%%%
% Texto del segundo subtótulo de la segunda portada.
%
%\textoSegundoSubtituloPortada{curso 2017/2018}% Para configurar el segundo "texto libre" de la segunda portada.
% Si no se especifica se indica "Dirigida por el Doctor" seguido
% del

%\directorPortada.
%%%%%%%%%%%%%%%%%%%%%%%%%%%%%%%%%%%%%%%%%%%%%%%%%%%%%%%%%%%%%%%%%%%%%%
%\textoSegundoSubtituloPortada{%
%\textit{Versión \texisVer}
%}

%%%%%%%%%%%%%%%%%%%%%%%%%%%%%%%%%%%%%%%%%%%%%%%%%%%%%%%%%%%%%%%%%%%%%%
% \explicacionDobleCara
% Si se utiliza, se aclara que el documento estó preparado para la
% impresión a doble cara.
%%%%%%%%%%%%%%%%%%%%%%%%%%%%%%%%%%%%%%%%%%%%%%%%%%%%%%%%%%%%%%%%%%%%%%
\explicacionDobleCara

%%%%%%%%%%%%%%%%%%%%%%%%%%%%%%%%%%%%%%%%%%%%%%%%%%%%%%%%%%%%%%%%%%%%%%
% \isbn
% Si se utiliza, apareceró el ISBN detrós de la segunda portada.
%%%%%%%%%%%%%%%%%%%%%%%%%%%%%%%%%%%%%%%%%%%%%%%%%%%%%%%%%%%%%%%%%%%%%%
%\isbn{978-84-692-7109-4}


%%%%%%%%%%%%%%%%%%%%%%%%%%%%%%%%%%%%%%%%%%%%%%%%%%%%%%%%%%%%%%%%%%%%%%
% \copyrightInfo
% Si se utiliza, apareceró información de los derechos de copyright
% detrós de la segunda portada.
%%%%%%%%%%%%%%%%%%%%%%%%%%%%%%%%%%%%%%%%%%%%%%%%%%%%%%%%%%%%%%%%%%%%%%
\copyrightInfo{\autor}

\makeCover




%---------------------------------------------------------------------
%
%                      dedicatoria.tex
%
%---------------------------------------------------------------------
%
% dedicatoria.tex
% Copyright 2009 Marco Antonio Gomez-Martin, Pedro Pablo Gomez-Martin
%
% This file belongs to the TeXiS manual, a LaTeX template for writting
% Thesis and other documents. The complete last TeXiS package can
% be obtained from http://gaia.fdi.ucm.es/projects/texis/
%
% Although the TeXiS template itself is distributed under the
% conditions of the LaTeX Project Public License
% (http://www.latex-project.org/lppl.txt), the manual content
% uses the CC-BY-SA license that stays that you are free:
%
%    - to share & to copy, distribute and transmit the work
%    - to remix and to adapt the work
%
% under the following conditions:
%
%    - Attribution: you must attribute the work in the manner
%      specified by the author or licensor (but not in any way that
%      suggests that they endorse you or your use of the work).
%    - Share Alike: if you alter, transform, or build upon this
%      work, you may distribute the resulting work only under the
%      same, similar or a compatible license.
%
% The complete license is available in
% http://creativecommons.org/licenses/by-sa/3.0/legalcode
%
%---------------------------------------------------------------------
%
% Contiene la página de dedicatorias.
%
%---------------------------------------------------------------------

\dedicatoriaUno{%

}

\dedicatoriaDos{%
\emph{%
  A mi madre, por ser una mujer con super poderes capaz de afrontar cualquier
  dificultad, siempre has estado ahi \\%
  \mbox{ }\\%
  A mi padre, por ayudarme a ser como soy ahora, nunca me rendiré contigo\\%
  \mbox{ }\\%
  A mi hermano, por ser el mejor, haces que quiera seguir superándome para alcanzarte\\%
  \mbox{ }\\%
  A Jenny, por cambiarme la vida, siempre tiras de mi cuando lo necesito\\% 
}%
}

\makeDedicatorias

% Variable local para emacs, para que encuentre el fichero
% maestro de compilación
%%%
%%% Local Variables:
%%% mode: latex
%%% TeX-master: "../Tesis.tex"
%%% End:


%---------------------------------------------------------------------
%
%                      agradecimientos.tex
%
%---------------------------------------------------------------------
%
% agradecimientos.tex
% Copyright 2009 Marco Antonio Gomez-Martin, Pedro Pablo Gomez-Martin
%
% This file belongs to the TeXiS manual, a LaTeX template for writting
% Thesis and other documents. The complete last TeXiS package can
% be obtained from http://gaia.fdi.ucm.es/projects/texis/
%
% Although the TeXiS template itself is distributed under the
% conditions of the LaTeX Project Public License
% (http://www.latex-project.org/lppl.txt), the manual content
% uses the CC-BY-SA license that stays that you are free:
%
%    - to share & to copy, distribute and transmit the work
%    - to remix and to adapt the work
%
% under the following conditions:
%
%    - Attribution: you must attribute the work in the manner
%      specified by the author or licensor (but not in any way that
%      suggests that they endorse you or your use of the work).
%    - Share Alike: if you alter, transform, or build upon this
%      work, you may distribute the resulting work only under the
%      same, similar or a compatible license.
%
% The complete license is available in
% http://creativecommons.org/licenses/by-sa/3.0/legalcode
%
%---------------------------------------------------------------------
%
% Contiene la pígina de agradecimientos.
%
% Se crea como un capítulo sin numeraciín.
%
%---------------------------------------------------------------------

\chapter{Agradecimientos}

\cabeceraEspecial{Agradecimientos}

\begin{FraseCelebre}
\begin{Frase}
I find your lack of faith disturbing\end{Frase}
\begin{Fuente}
Darth Vader, Star Wars: A New Hope.
\end{Fuente}
\end{FraseCelebre}

Groucho Marx decía que encontraba a la televisiín muy educativa porque
cada vez que alguien la encendía, íl se iba a otra habitaciín a leer
un libro. Utilizando un esquema similar, nosotros queremos agradecer
al Word de Microsoft el habernos forzado a utilizar \LaTeX. Cualquiera
que haya intentado escribir un documento de mís de 150 píginas con
esta aplicaciín entenderí a quí nos referimos. Y lo decimos porque
nuestra andadura con \LaTeX\ comenzí, precisamente, despuís de
escribir un documento de algo mís de 200 píginas. Una vez terminado
decidimos que nunca mís pasaríamos por ahí. Y entonces caímos en
\LaTeX.

Es muy posible que hubíeramos llegado al mismo sitio de todas formas,
ya que en el mundo acadímico a la hora de escribir artículos y
contribuciones a congresos lo mís extendido es \LaTeX. Sin embargo,
tambiín es cierto que cuando intentas escribir un documento grande
en \LaTeX\ por tu cuenta y riesgo sin un enlace del tipo ``\emph{Author
  instructions}'', se hace cuesta arriba, pues uno no sabe por donde
empezar.

Y ahí es donde debemos agradecer tanto a Pablo Gervís como a Miguel
Palomino su ayuda. El primero nos ofrecií el cídigo fuente de una
programaciín docente que había hecho unos aíos atrís y que nos sirvií
de inspiraciín (por ejemplo, el fichero \texttt{guionado.tex} de
\texis\ tiene una estructura casi exacta a la suya e incluso puede
que el nombre sea el mismo). El segundo nos dejí husmear en el cídigo
fuente de su propia tesis donde, ademís de otras cosas mís
interesantes pero menos curiosas, descubrimos que aín hay gente que
escribe los acentos espaíoles con el \verb+\'{\i}+.

No podemos tampoco olvidar a los numerosos autores de los libros y
tutoriales de \LaTeX\ que no sílo permiten descargar esos manuales sin
coste adicional, sino que tambiín dejan disponible el cídigo fuente.
Estamos pensando en Tobias Oetiker, Hubert Partl, Irene Hyna y
Elisabeth Schlegl, autores del famoso ``The Not So Short Introduction
to \LaTeXe'' y en Tomís Bautista, autor de la traducciín al espaíol. De
ellos es, entre otras muchas cosas, el entorno \texttt{example}
utilizado en algunos momentos en este manual.

Tambiín estamos en deuda con Joaquín Ataz Lípez, autor del libro
``Creaciín de ficheros \LaTeX\ con {GNU} Emacs''. Gracias a íl dejamos
de lado a WinEdt y a Kile, los editores que por entonces utilizíbamos
en entornos Windows y Linux respectivamente, y nos pasamos a emacs. El
tiempo de escritura que nos ahorramos por no mover las manos del
teclado para desplazar el cursor o por no tener que escribir
\verb+\emph+ una y otra vez se lo debemos a íl; nuestro ocio y vida
social se lo agradecen.

Por íltimo, gracias a toda esa gente creadora de manuales, tutoriales,
documentaciín de paquetes o respuestas en foros que hemos utilizado y
seguiremos utilizando en nuestro quehacer como usuarios de
\LaTeX. Sabíis un montín.

Y para terminar, a Donal Knuth, Leslie Lamport y todos los que hacen y
han hecho posible que hoy puedas estar leyendo estas líneas.

\endinput
% Variable local para emacs, para  que encuentre el fichero maestro de
% compilaciín y funcionen mejor algunas teclas rípidas de AucTeX
%%%
%%% Local Variables:
%%% mode: latex
%%% TeX-master: "../Tesis.tex"
%%% End:


% ---------------------------------------------------------------------
%
% resumen.tex
%
% ---------------------------------------------------------------------
%
% Contiene el cap_tulo del resumen.
%
% Se crea como un cap_tulo sin numeraci_n.
%
% ---------------------------------------------------------------------

\chapter{Abstract}
\cabeceraEspecial{Abstract}

\begin{FraseCelebre}
  \begin{Frase}
    I invent, transform, create and destroy for a living and when I don't like something about the World I change it. 
  \end{Frase}
  \begin{Fuente}
    Rick and Morty -  Dan Harmon, Justin Roiland
  \end{Fuente}
\end{FraseCelebre}

Science publication and peer review raises concerns about fairness, quality,
performance, cost or accuracy. The Open Access movements has been unable to
fulfill all its promises, and middlemen publishers can still impose policies and
concentrate profits. This work, using emerging distributed technologies such as
Blockchain and IPFS, proposes a decentralized publication system for open
science called Decentralized Science\footnote{Available at
  https://decentralized.science}. It provides transparent governance, a
distributed reviewer reputation system, and open access by-design.

\bb{Keywords:} Peer review, reputation network, open acess, blockchain,
Ethereum, IPFS.


\chapter{Resumen}
\cabeceraEspecial{Resumen}

\begin{FraseCelebre}
  \begin{Frase}
    We do what we must because we can.
  \end{Frase}
  \begin{Fuente}
    GLaDOS - Portal.
  \end{Fuente}
\end{FraseCelebre}

El proceso de publicación científica y la revisión por pares generan inquietudes
sobre la equidad, calidad, rendimiento, coste o precision de este. Los movimientos del
\emph{Open Access} no han podido cumplir todas sus promesas, ya que las
editoriales, que actúan como 
intermediarios, todavía pueden imponer políticas y concentrar gran parte de los
beneficios económicos de este sistema.
Este trabajo, utilizando tecnologías distribuidas emergentes como Ethereum o
IPFS, propone un sistema descentralizado de publicación científica para la
\emph{ciencia libre} llamado Decentralized Science\footnote{Disponible en
  https://decentralized.science}. Proporciona un sistema de gobernanza
tranparente, un sistema de reputación de revisores distribuido y un diseño
totalmente \emph{Open Access}.

\bb{Palabras Clave:} Revisión por pares, red de reputación, open access,
blockchain, Ethereum, IPFS.
\endinput
% Variable local para emacs, para que encuentre el fichero maestro de
% compilaci_n y funcionen mejor algunas teclas r_pidas de AucTeX
%%%
%%% Local Variables:
%%% mode: latex
%%% TeX-master: "../Tesis.tex"
%%% End:


\ifx\generatoc\undefined
\else
%---------------------------------------------------------------------
%
%                          TeXiS_toc.tex
%
%---------------------------------------------------------------------
%
% TeXiS_toc.tex
% Copyright 2009 Marco Antonio Gomez-Martin, Pedro Pablo Gomez-Martin
%
% This file belongs to TeXiS, a LaTeX template for writting
% Thesis and other documents. The complete last TeXiS package can
% be obtained from http://gaia.fdi.ucm.es/projects/texis/
%
% This work may be distributed and/or modified under the
% conditions of the LaTeX Project Public License, either version 1.3
% of this license or (at your option) any later version.
% The latest version of this license is in
%   http://www.latex-project.org/lppl.txt
% and version 1.3 or later is part of all distributions of LaTeX
% version 2005/12/01 or later.
%
% This work has the LPPL maintenance status `maintained'.
%
% The Current Maintainers of this work are Marco Antonio Gomez-Martin
% and Pedro Pablo Gomez-Martin
%
%---------------------------------------------------------------------
%
% Contiene  los  comandos  para  generar los  _ndices  del  documento,
% entendiendo por _ndices las tablas de contenidos.
%
% Genera  el  _ndice normal  ("tabla  de  contenidos"),  el _ndice  de
% figuras y el de tablas. Tambi_n  crea "marcadores" en el caso de que
% se est_ compilando con pdflatex para que aparezcan en el PDF.
%
%---------------------------------------------------------------------


% Primero un poquito de configuraci_n...


% Pedimos que inserte todos los ep_grafes hasta el nivel \subsection en
% la tabla de contenidos.
\setcounter{tocdepth}{2}

% Le  pedimos  que nos  numere  todos  los  ep_grafes hasta  el  nivel
% \subsubsection en el cuerpo del documento.
\setcounter{secnumdepth}{3}


% Creamos los diferentes _ndices.

% Lo primero un  poco de trabajo en los marcadores  del PDF. No quiero
% que  salga una  entrada  por cada  _ndice  a nivel  0...  si no  que
% aparezca un marcador "_ndices", que  tenga dentro los otros tipos de
% _ndices.  Total, que creamos el marcador "_ndices".
% Antes de  la creaci_n  de los _ndices,  se a_aden los  marcadores de
% nivel 1.

\ifpdf
   \pdfbookmark{Indexes}{indexes}
\fi

% Tabla de contenidos.
%
% La  inclusi_n  de '\tableofcontents'  significa  que  en la  primera
% pasada  de  LaTeX  se  crea   un  fichero  con  extensi_n  .toc  con
% informaci_n sobre la tabla de contenidos (es conceptualmente similar
% al  .bbl de  BibTeX, creo).  En la  segunda ejecuci_n  de  LaTeX ese
% documento se utiliza para  generar la verdadera p_gina de contenidos
% usando la  informaci_n sobre los  cap_tulos y dem_s guardadas  en el
% .toc
\ifpdf
   \pdfbookmark[1]{Table of contents}{table of contents}
\fi

\cabeceraEspecial{Index}

\tableofcontents

\newpage

% _ndice de figuras
%
% La idea es semejante que para  el .toc del _ndice, pero ahora se usa
% extensi_n .lof (List Of Figures) con la informaci_n de las figuras.



\ifpdf
   \pdfbookmark[1]{Index of figures}{index of figures}
\fi

\cabeceraEspecial{Index of Figures}

\listoffigures

\newpage

% _ndice de tablas
% Como antes, pero ahora .lot (List Of Tables)

\ifpdf
   \pdfbookmark[1]{Index of tables}{index of tables}
\fi

\cabeceraEspecial{Index of tables}

\listoftables

\newpage

% Variable local para emacs, para  que encuentre el fichero maestro de
% compilaci_n y funcionen mejor algunas teclas r_pidas de AucTeX

%%%
%%% Local Variables:
%%% mode: latex
%%% TeX-master: "../Tesis.tex"
%%% End:

\fi

% Marcamos el  comienzo de  los cap_tulos (para  la numeraci_n  de las
% p_ginas) y ponemos la cabecera normal
\mainmatter
\restauraCabecera

%%---------------------------------------------------------------------
%
%                          Parte 1
%
%---------------------------------------------------------------------
%
% Parte1.tex
% Copyright 2009 Marco Antonio Gomez-Martin, Pedro Pablo Gomez-Martin
%
% This file belongs to the TeXiS manual, a LaTeX template for writting
% Thesis and other documents. The complete last TeXiS package can
% be obtained from http://gaia.fdi.ucm.es/projects/texis/
%
% Although the TeXiS template itself is distributed under the
% conditions of the LaTeX Project Public License
% (http://www.latex-project.org/lppl.txt), the manual content
% uses the CC-BY-SA license that stays that you are free:
%
%    - to share & to copy, distribute and transmit the work
%    - to remix and to adapt the work
%
% under the following conditions:
%
%    - Attribution: you must attribute the work in the manner
%      specified by the author or licensor (but not in any way that
%      suggests that they endorse you or your use of the work).
%    - Share Alike: if you alter, transform, or build upon this
%      work, you may distribute the resulting work only under the
%      same, similar or a compatible license.
%
% The complete license is available in
% http://creativecommons.org/licenses/by-sa/3.0/legalcode
%
%---------------------------------------------------------------------

% Definición de la primera parte del manual

\partTitle{Primera parte}

\partDesc{Descripción de la primera parte}

\partBackText{Este es el backtext de la descrición de la primera parte}

\makepart

%\chapter{Abstract}

\begin{FraseCelebre}
  \begin{Frase}
    The needs of the many outweigh the needs of the few
  \end{Frase}
  \begin{Fuente}
    Spock - The Wrath of Khan
  \end{Fuente}
\end{FraseCelebre}

% -------------------------------------------------------------------
\section{Cool Section}
% -------------------------------------------------------------------

Lorem ipsum dolor sit amet, consectetur adipiscing elit. Praesent fermentum orci
a justo sagittis, at tincidunt enim luctus. Curabitur imperdiet mauris sed
mattis semper. Aenean augue risus, viverra vel porta a, auctor ac enim. Sed quis
auctor tellus. Suspendisse potenti. Vestibulum nec lectus turpis. Morbi luctus
eros ante, eu consequat magna maximus ut. Donec nec dui sagittis, ornare lorem
a, condimentum odio. Vestibulum ante ipsum primis in faucibus orci luctus et
ultrices posuere cubilia Curae; Nunc quis ipsum eget tellus placerat facilisis.
Curabitur tortor nunc, elementum at imperdiet a, hendrerit id augue. Aenean
vitae lacus eget diam posuere aliquet vel in elit.

Aenean purus est, tempus eget tristique nec, fringilla a enim. Sed in volutpat
eros. Sed commodo congue metus ac aliquam. Quisque auctor dolor libero, vitae
mattis ante luctus sit amet. Aliquam iaculis urna nec lorem rhoncus, commodo
dignissim mi ultrices. Mauris sem augue, luctus ac tincidunt id, sollicitudin at
ipsum. Suspendisse mattis venenatis dolor. Pellentesque velit sem, pulvinar
vitae tortor tempor, blandit aliquet velit. Nunc mattis urna diam, ac maximus
libero molestie non.

Orci varius natoque penatibus et magnis dis parturient montes, nascetur
ridiculus mus. Quisque id egestas tellus. Nunc suscipit ex ac quam pretium, sit
amet vehicula risus hendrerit. Suspendisse faucibus ante metus, sed pulvinar
odio pulvinar condimentum. Donec vel arcu egestas, posuere metus quis, varius
mauris. Suspendisse ut ligula id justo eleifend vestibulum sit amet faucibus
neque. Praesent dignissim risus quis consectetur porta. Maecenas faucibus velit
non pretium ullamcorper. Praesent fringilla pharetra purus. Aenean ullamcorper
nisi gravida sagittis dapibus. Nunc commodo arcu nec cursus venenatis. Integer
vel turpis convallis, feugiat sapien id, euismod arcu. Aliquam sit amet iaculis
lacus. Vestibulum ut ex ac libero efficitur varius at in nisl. Morbi vel posuere
diam, a porttitor turpis.

Pellentesque non justo est. Nunc luctus ullamcorper tincidunt. Aliquam eu nisl a
orci aliquam cursus. Mauris id sollicitudin mauris. Suspendisse quis dolor id
magna porttitor mollis eu sit amet diam. Donec a elementum nibh. Sed egestas id
sapien nec aliquet. Vivamus porta dignissim bibendum. Donec at odio volutpat,
vestibulum elit at, fringilla ipsum. Donec vehicula lectus efficitur est
fermentum, ac tincidunt quam maximus. Proin lectus sem, sodales quis nunc ut,
maximus ullamcorper erat. Mauris lobortis justo quis malesuada viverra. Sed
suscipit, elit quis tincidunt condimentum, ipsum augue ornare velit, sit amet
fermentum leo orci ac risus. Vestibulum elementum viverra porta.

Sed gravida, risus nec scelerisque egestas, metus elit maximus leo, sit amet
lacinia erat purus at mauris. Aliquam eget libero velit. Proin luctus risus et
maximus efficitur. Praesent eget cursus ipsum. In hac habitasse platea dictumst.
Duis fringilla purus eu enim hendrerit, ut gravida tortor scelerisque. Donec
quis pellentesque ex, in auctor turpis. Ut non turpis purus.

Duis porttitor turpis purus, eget volutpat diam posuere eget. Morbi porttitor
risus quis tortor pellentesque varius. Nullam eu odio a augue tincidunt cursus.
Proin semper sapien augue, sit amet maximus turpis vestibulum ac. Quisque semper
justo nunc, sed efficitur augue ullamcorper ac. Praesent egestas eget neque quis
consectetur. Ut quis nulla rutrum, placerat leo nec, euismod quam. Phasellus
dapibus ligula vitae lacus lacinia blandit. Morbi vel ligula iaculis, aliquet ex
ut, scelerisque sem. Donec rutrum lacus quis odio vulputate ultrices. Quisque
ullamcorper rhoncus mauris, ac vestibulum magna blandit et. Suspendisse eu diam
rhoncus, sagittis quam vitae, maximus est. Donec augue diam, euismod a elit ac,
imperdiet accumsan nunc.

Aenean quis metus sed urna fringilla scelerisque. Proin tellus lectus, laoreet
tincidunt commodo eget, euismod sit amet nisi. Morbi eu massa eu arcu lobortis
malesuada. Orci varius natoque penatibus et magnis dis parturient montes,
nascetur ridiculus mus. Donec ut enim vel lacus eleifend suscipit. Curabitur id
ex vitae leo tempor elementum. In vestibulum mi eu ligula sagittis, eu efficitur
est auctor. Morbi ornare molestie rutrum. Aliquam sit amet fermentum enim. Fusce
sed tempus sem. In elementum dolor nec justo tempus hendrerit. Cras sit amet
cursus est, sit amet porta lacus. Nulla pretium at dolor quis volutpat.
Pellentesque condimentum ultricies urna, sed aliquam ipsum. Praesent consequat
faucibus massa id porta. Nullam nibh purus, maximus eu tristique ut, convallis
sit amet sapien.

Integer ac nisi leo. Quisque sollicitudin eros lorem, in suscipit orci congue
eget. Suspendisse gravida augue nec faucibus interdum. Integer blandit ligula
nec ex dapibus sodales. Aenean accumsan ante nibh, ac varius nisl pharetra in.
Duis mollis convallis neque, vitae finibus dui tincidunt ac. Aliquam a leo sed
elit porttitor dignissim blandit sit amet mauris. Proin viverra id ex quis
maximus. Etiam ultrices tristique faucibus. Donec egestas lorem enim, vel rutrum
erat interdum eu. Vestibulum turpis libero, aliquet non neque nec, placerat
tempus nulla.

Phasellus accumsan lacus ut enim rutrum posuere. In sit amet enim mi. Nulla et
fringilla justo. Etiam ut posuere velit, non fringilla risus. Praesent mattis
diam nec convallis finibus. Donec malesuada felis eu consectetur volutpat. Nulla
mattis neque eu arcu fermentum, ac accumsan quam placerat. Maecenas iaculis
blandit leo, ut semper orci vestibulum et. Quisque ut lorem pretium, condimentum
lorem vel, volutpat nisl. Maecenas ut enim sed elit maximus aliquam eget in
diam. Donec vulputate nunc sed diam accumsan tempus.

Phasellus mollis tempor lectus vitae pretium. Praesent ullamcorper suscipit est
eleifend aliquet. Nulla nec diam consectetur, venenatis arcu non, finibus odio.
Ut vel nisi condimentum, imperdiet enim id, ultrices felis. Duis congue dui
sapien, eu mattis ipsum mollis at. Donec a ex a orci tincidunt fermentum.
Aliquam pretium non orci nec pharetra. Donec tellus erat, maximus et
sollicitudin eget, elementum a mauris. Phasellus tempus mauris vitae tortor
elementum, eleifend viverra nisl luctus.

Aliquam in est sed velit elementum dictum. Phasellus auctor est sit amet egestas
vulputate. Nunc volutpat nibh lacus, eu dignissim libero gravida sit amet. Sed
accumsan dui sit amet leo pretium, id porttitor nisl fringilla. Mauris vehicula
pretium cursus. Integer bibendum tellus a ante eleifend tempor. Phasellus vitae
efficitur felis, a vulputate nulla. In finibus massa ut ante consequat auctor.
Phasellus elementum odio eget viverra lobortis. Lorem ipsum dolor sit amet,
consectetur adipiscing elit. Nam et quam et tellus ullamcorper consequat non non
mauris. Sed molestie ut turpis quis consectetur. Ut hendrerit justo eget orci
faucibus dignissim. Aenean ut viverra nulla. Proin vestibulum vehicula dapibus.

Cras posuere laoreet dui sit amet bibendum. Phasellus mollis euismod sapien eget
iaculis. Donec vel commodo ante. Sed vitae elit eget felis suscipit rhoncus. Nam
lacus leo, porttitor et vestibulum sodales, convallis eget sapien. Fusce quam
tellus, finibus eget feugiat at, ultrices sed risus. Cras ac semper odio.

Cras egestas rhoncus lorem ac cursus. Sed ut ipsum ut urna commodo tincidunt sit
amet id nulla. Integer purus tortor, iaculis vel nisi sit amet, commodo varius
diam. Etiam accumsan, diam sit amet posuere laoreet, felis mauris tincidunt
ante, vitae hendrerit nibh sapien in sem. Suspendisse rhoncus a nunc eget
ornare. Quisque posuere mauris nunc, ac convallis nisi condimentum vel. Integer
pulvinar neque at ligula interdum, ac blandit augue ultricies. Aliquam fermentum
enim in metus volutpat ullamcorper. Nam rutrum posuere mi, sit amet molestie
erat convallis eget. Vestibulum eu dictum lorem. Ut commodo cursus nisl, non
elementum quam ullamcorper eget.

Vivamus in tincidunt enim, quis tempus metus. Nam eu tristique nibh, sit amet
consectetur quam. Nulla nisi erat, tristique eget urna sit amet, viverra congue
metus. Cras eu nisl vel diam lobortis tempor non at tortor. Morbi maximus tellus
placerat elementum elementum. Praesent nec dignissim dolor. Sed eget purus
tortor. Nunc quis ullamcorper lacus.

Aenean quis lectus congue, vehicula leo porta, viverra velit. Suspendisse
potenti. Etiam fermentum tempus enim. Integer ac pharetra odio, eget facilisis
velit. Praesent auctor risus nec mauris imperdiet maximus. Mauris venenatis nisi
vitae aliquam fermentum. Ut volutpat bibendum tincidunt. Nam quis risus at orci
porttitor blandit ac a nulla. Ut ipsum quam, dapibus eu ligula id, molestie
accumsan elit. Nam rhoncus sem sit amet elit finibus, posuere porttitor augue
facilisis.

%%%
%%% Local Variables:
%%% mode: latex
%%% TeX-master: "../Tesis.tex"
%%% End:

\chapter{Introduction}

\begin{FraseCelebre}
  \begin{Frase}
    Who Watches the Watchmen?
  \end{Frase}
  \begin{Fuente}
    Watchmen - Alan Moore
  \end{Fuente}
\end{FraseCelebre}

Scientific research nowadays is based on publications in journals with a high
impact factor\cite{doi:10.1001/jama.295.1.90}, the most well-known is the
Journal Citation Reports (\emph{JCR}). This factor was originally determined by
the Science Citation Index, born in 1955\cite{garfield2007evolution} but
nowadays it is managed by a private company dedicated to benefit from the work
of researchers\cite{toledo2011book}. This poses two important problems when it
comes to entering the world of academic research:

The first is that scientific journals that want to maintain their impact factor,
have to make sure that the articles that come out in their issues have a large
number of citations, so they are always going to look for novel and high-impact
articles. Therefore, the editors of these journals will have a network of
reviewers on which they can trust to review the article. But sometimes these
reviews are not entirely objective, there are many cases of unfavorable reviews
due to gender causes, especially in scientific fields
\cite{wenneras2001nepotism}.

Besides, it is necessary to consider that the time of revision for an article is
excessively long, causing the process of academic investigation being quite slow
\cite{huisman2017duration}.

The second problem is that the benefits of the scientific distribution are
centralized in publication systems, nor the authors, the reviewers or the
readers get money from it. Today, with electronic paper distribution,
universities purchase site licenses for online access to journal contents. This
system implies an additional cost for the universities who want to advance in
their research fields and does not have enough money for it. However, site
licenses are not always disadvantageous. Some journals issued by private
companies and universities adjust their prices to maximize subscriptions
\cite{bergstrom2004costs}. But generally, people who earn money from this
paper-based system only act as an intermediary between the authors, the
reviewers and the readers.

The internet offers the possibility to meet people all around the word, and when
it comes to trust total strangers, you should have a system in which you can
rely to deposit you trust in them. Reputation systems are the solution to this
problems, since they offer you a first good impression of an unknown person
\cite{resnick2000reputation}.

Editors who want to assign the review of a paper to a series of reviewers have
to rely on them beforehand. Thus limiting the spectrum of fields that can be
revised to the fields in which those reviewers are experts. If you want to
broaden the scope of reviewers with more fields of expertise, you need to
contact new reviewers. But there is no easy way to predict reviewer quality from
reviewers training and experience factors \cite{callaham_relationship_2007}, so
a rating system of reviewers should be useful for journals to select the best
reviewers. The solution is a reviewer reputation network, in which reviewers get
rated based on their reviews and build up their reputation based on good
practices, and helpful reviews. In this network, publishers who have to find new
reviewers for their papers, do not have to know them beforehand, since trust is
placed in the reputation network instead of the person itself.

\section{Objective}

We aim to challenge middlemen in science publication such as traditional
publishers. Particularly, we propose a decentralized publication system for open
science, allowing 1) paper submissions, 2) assignment of reviewers, 3) peer
review, and, as a novelty, 4) the rating of peer reviews. With this distributed
system, we aim to improve the quality and efficiency of reviews and knowledge
distribution, helping editors, authors, and reviewers:
\begin{itemize}
  \item Editors and journals will be able to find the best peer reviewers in
    their fields of interest, and those that respond quickly.  Thus reducing time-to-publish and publishing costs.
  \item Authors will be able to submit papers to time-responsive, free, open access
      journals, and forget about slow, unfair and unaccountable anonymous reviews.
  \item Reviewers will finally have their work recognized.
 \end{itemize}

 We are interested in exploring the following challenges, that could be dealt with our technology:
\begin{itemize}
  \item Reduce time-to-review by rewarding on-time reviewers.
  \item Measure and prevent sexism, nepotism and other abuses in peer review.
  \item Develop fully autonomous decentralized journals.
  \item Explore fully free publication systems for Open Access science, while enabling innovative business models.
  \item Explore alternative and open metrics for papers, journals and reviewers.
  \end{itemize}

  \section{In this work}
  In this work there will be the  following sections:
  \begin{itemize}
    \item \textbf{State of the art:} This chapter is about what
      methods, systems and technologies try to change the actual publication
      systems and how good or bad they are.
    \item \textbf{Methodology and Technology:} This chapter is about the
      methodology I followed during the realization of this work, and the
      technologies I have used to face the challenges and why I decided to use
      them.
    \item \textbf{Platform description:} This is the main chapter of this work.
      It contains the platform description, its implementation, how it works,
      why is better than the actual publication systems, the challenges I have
      faced during the realization and the internal structure of the final
      system.
    \item \textbf{Results and discussion:} This chapter is about the results
      obtained after the realization of the work proposed in this project, how
      it will affect the scientific community and how I measured the potential
      impact if it becomes a wide-used publication system.
    \item \textbf{Conclusion and future work:} This chapter is about the
      implications of this work in the scientific community and settles the next
      steps to follow to create an ecosystem of autonomous publication systems
      without the need of middlemen such as journals or editors, proposing a
      future PhD about this subject.
  \end{itemize}


%%%
%%% Local Variables:
%%% mode: latex
%%% TeX-master: "../Tesis.tex"
%%% End:

\chapter{State of the art}

\begin{FraseCelebre}
  \begin{Frase}
    The future is already here – it's just not evenly distributed.
  \end{Frase}
  \begin{Fuente}
    William Gibson
  \end{Fuente}
\end{FraseCelebre}

% -------------------------------------------------------------------
% \section{Cooler Section}
% -------------------------------------------------------------------



\section{Alternative Publication systems}
\label{soa:aps}
Publication systems, as seen on section \ref{intro} are vampirizing the
industry. However, there are some attempts to change this paradigm on behalf of
science dissemination.

Open journal systems~\cite{willinsky2005open} is an open software designed to
facilitate the publishing process. This project was created by the Public
Knowledge Project\footnote{https://pkp.sfu.ca/about/} and it targets open-access
online journals that want to speed up the publication processes. The system
provides tools to control the whole publishing process from article submission,
through peer reviewing to the final publication issue.

Mega-journals~(or Multi-journals)~\cite{binfield2013open,wellen2013open} combine
multiple journals into a single journal, allowing the publication of open-access
papers, which have gone through a peer review process. The first journal to
adopt this idea was the \emph{PLOS ONE}
Journal\footnote{http://journals.plos.org/plosone/} as of the project
\emph{Public Library of Science}. This project aims to create a library of
scientific journals under the values of open access and creative commons
licenses. As a result of the success of the \emph{PLOS ONE} journal, other
publishers have started their own mega-journals. Featuring alternative impact
metrics, reusability of figures and data, post-publication discussions and
portable reviews from other journals~\cite{bjork2015have}.

The continuous publication model is based on publishing individual papers
migrating from the previous issue-based model~\cite{anderton2013continuous}.
This method is seen as an altenative for open-access journals as it speeds up
the publication process~\cite{haymanview}. \emph{DPSOS}\footnote{Decentralized
  Publication System for Open Science} adopts this model by design (see
section~\ref{tech:sec:ethereum:sm}) as it publish automatically papers that meet
certain preconditions that are written in the blockchain.

Preprints are scientific papers that have not yet gone through the peer review
process~\cite{harnad2003electronic}. Formerly, the preprints that were sent to
the journals were private, and only accessible by the editors and assigned
reviewers. But nowadays it is common to publish a preprint before sending it to
a journal, uploading it to specialized platforms like
arXiv\footnote{https://arxiv.org/} or
Preprints\footnote{https://www.preprints.org/}~\cite{brown2001volution}. In fact
there is a correlation between the upload of a preprint and early citations
after the publication of the paper~\cite{shuai2012scientific}. This system is a
possible solution to the cold-start problem that papers of new researchers who
enter the academic career have~\cite{sugiyama2010scholarly}.

Social networks have also made a dent in the academic world, creating platforms
to contact other researchers and encouraging them to share open access papers.
Some of the well-known are Research Gate\footnote{https://www.researchgate.net},
Mendeley\footnote{https://www.mendeley.com} or
Academia\footnote{http://academia.edu}. But despite the good intentions of the
creators of these platforms, many of the journals demand the copyright of the
papers they publish, preventing the authors from sharing them through these
services.


Decentralized alternatives, in spite of their
promises~\cite{bartlingblockchain}, are still in their infancy. A few proposals,
none of them functional to date, have appeared recently.

One of them is a peer review proposal that tries to solve some of the peer
review socio-technical problems using cryptocurrencies~\cite{tennant2017multi}.
It needs a critical threshold of research community engagement, changing the
actual processes and platforms, to start being implemented.

Blockchain-enabled apps have also been proposed, with voting and storage of
publications. This is the case of Aletheia~\cite{morton2017aletheia}, a software
for getting open access papers published. This platform idea aims to use
blockchain as a decentralized and distributed database as a publishing platform.

Peer review quality control through blockchain-based cohort
trainings~\cite{dhillon2016bench} have been also proposed, with the promise of
transparency and decentralization using a distributed ledger. Research labs can
use this training network to test their technology and reduce the risk for
private investment opportunities.

Finally, some of the off-chain journals are adapting to the demands of the
current scientific community like Ledger\footnote{https://ledgerjournal.org}, a
cryptocurrencies and blockchain-based journal that records the publication
timestamps in the Bitcoin blockchain.

\section{Reputation systems}
\label{soa:rs}
Reputation systems today arise from the need to trust unknown
individuals~\cite{resnick2000reputation}. Many of the big internet communities
like Stackexchange\footnote{https://stackexchange.com/} or
reddit\footnote{https://www.reddit.com/} have their own reputation system.
Reputation systems behavior may vary depending on the
platform~\cite{josang2002beta}, but the most usual is the one where users get a
score based on certain interaction with the community.

Reputation systems also have a very large niche in e-commerce webs such as
Ebay\footnote{https://www.ebay.com}, in which people pay for a product sold by
an unknown vendor. There must be a previous trust in the vendor before buying
any product, so a reputation system offers a score given by other users that
encourages you to trust or not that certain seller~\cite{resnick2002trust}.

Reputation systems vary widely in scope, such as one for peer-to-peer
computing~\cite{zhou2007powertrust}, vehicle ad-hoc~\cite{dotzer2005vars}, web
services~\cite{moore2008reputation} and even Wikipedia~\cite{adler2007content}.
All of them are based on an exchange of trust between users who use these
services.

This same concept was intended to be transferred within the blokchain using a
token as a trust unit, which users exchanged as a sign of trust deposits among
them~\cite{sharples2016blockchain}.

But reputation systems also have problems when it comes to defend the users from
attacks to individuals~\cite{hoffman2009survey} and unfair ratings
~\cite{whitby2004filtering}, so the architecture chosen for it must consider
these weak points and try to mitigate them.

This paper proposes the development of a decentralized publication system for
open science. It aims to challenge the technical infrastructure that supports
the middlemen role of traditional publishers. Due to the successes of the Open
Access movement, some of the scientific knowledge is today freely provided by
the publishers. However, the content is still mostly served from their
infrastructure (i.e. servers, web platforms). This ownership of the
infrastructure gives them a position of power over the scientific community
which produces the contents~\cite{fuster2010governance}. Such central and
oligopolistic position in science dissemination allows them to impose policies
(e.g. copyright ownership, Open Access prices) and concentrate profits.

The proposed system aims to move the infrastructure control from the publishers
to the scientific community. It entails the decentralization of three essential
functions of science dissemination: 1) the peer review process, 2) the selection
and recognition of peer reviewers, and 3) the distribution of scientific
knowledge. The following section provides an overview of the system features,
while the final section discusses its challenges.
%%%
%%% Local Variables:
%%% mode: latex
%%% TeX-master: "../Tesis.tex"
%%% End:

\chapter{Methodology \& Technology}

\begin{FraseCelebre}
  \begin{Frase}
    The needs of the many outweigh the needs of the few
  \end{Frase}
  \begin{Fuente}
    Spock - The Wrath of Khan
  \end{Fuente}
\end{FraseCelebre}

\section{Methodology}

Decentralized Science is based on a project idea outlined in a blockchain event
in September 2017. In this event we were a group of 4 developers with one month
to design an idea of a platform or service using blockchain technologies to
solve a particular problem. This Master's Thesis is a first approach to the
software design and implementation of that project sketch.

\subsection{Project Timeline}
\label{methodology:timeline}


Decentralized Science's project timeline has three different parts:

\begin{itemize}
  \itbf{Pre-project:} \emph{Blockchain for Social Impact} was an online event
  that took place in September 2017. The main goal of that event was to design a
  blockchain-powered platform to solve any social related problem, and the best
  one would receive funding. Me and other partners formed a team to develop the
  first Decentralized Science's idea within the month that the event lasted. In
  this period we had to make a series of deliverables every week to show that we
  were working on one project. In order to make this, the team met 3 times a
  week to design the platform using methodologies such as Brainstorming and
  Value Proposition Canvas. The main idea of the platform and the problems it
  had to solve were identified in this period. Unfortunately the project was not
  one of the winners of the event and the team decided to pause the project.

  \itbf{Project implementation:} Since I was the only one with programming
  knowledge in blockchain, I decided to continue with the project on my own.
  Consequently, and as my Master's Thesis, I started the project's
  implementation and software design from October 2017 until February 2018. To
  achieve this, I used agile methodologies such as scrum and extreme programming
  to be able to develop incremental prototypes as first approach to the
  platform. During this period I have developed two functional prototypes using
  different blockchain architectures. Afterwards, I have tested them to compare
  the two implementations in terms of performance, execution cost and
  compatibility with other possible platforms. These prototypes are developed
  under an \bb{open source license} and the last version is available in
  Github\footnote{see https://decentralized.science}. During these period I had
  monthly reunions with my directors to update the status of the project.
  Finally, since January of 2018 I have started this manuscript as the master's
  thesis memory.

  \itbf{Post-project:} From March 2018 the team decided to continue working in
  the project's idea, trying to find supporters and funding to develop a final
  platform. Decentralized Science has been in three European conferences on
  blockchain in which several projects were discovered in the same field:
  ETHCC2018 (Paris), SPOBC2018 (Vienna) and PEERE2018 (Rome). In addition,
  Decentralized Science has formed a group with other similar projects called
  Open Science Ecosystem with more than 30 projects in open science around the
  world. Currently the team are sending scientific papers to different
  conferences to validate the ideas of this platform with the scientific
  community. One article has been published in a small conference and another
  has been accepted with changes in a CORE-A conference so far (see
  section~\ref{sec:conferences-papers}). This papers and the feedback received
  by the community allowed me to improve the platform source code and complete
  this manuscript.
  
\end{itemize}

\subsection{Brainstorming}

Brainstorming was born as a method to increase creativity in groups and
organizations. There are only few rules on this method: do not criticize any of
the given ideas, quantity is desired over quality, try to combine suggested
ideas and give all the ideas that come to mind, no matter if they are possible
or not~\cite{osborn1953applied}.

This method is used nowadays in companies and work groups as part of the process
of the creatfile uploadsion of a product, although there are some critics about
brainstorming and in some cases instead of encouragind creativity, inhibits
it~\cite{sutton1996brainstorming,mullen1991productivity}

As part of the \bb{pre-project}(see section~\ref{methodology:timeline}) the team
decided to use this methodology and do a brainstorming session to define what
kind of problem should be solved with blockchain. Many ideas emerged and were
capture into a white board without discrimination, no matter how hard or easy to
implement they were.

After saying enough ideas to fill the board we filtered the ones that were
impossible to achieve. Then, each one voted the best three, making a ranking of
the 3-4 best projects to start working on. Some of the ideas we came up to were
creating a distributed \emp{wikipedia} with governance models, an application to
contact people from minority groups in countries where they are persecuted
collectives, a distributed and community driven NGO\footnote{Non-governmental
  organization}, and a crowdfunding platform for \emph{whisteblowers}\footnote{A
  person who exposes any kind of information or activity that is deemed
  illegal}.

Finally we decided to create an approach to a distributed platform for open
science which we initially called Alexandria, in honor of Alexandra Elbakyan,
the creator of Sci-hub. Nevertheless we changed the name of the project to
Decentralized Science because there were a similar project about distributed
system for media files using blockchain with the same
name\footnote{https://www.alexandria.io/}.

\subsection{Value proposition canvas}
\label{sec:value-prop-canv}

\figura{vpc.jpg}{width=0.9\linewidth}{vpcimage}{Image of the value proposition
  canvas after the session}

A value proposition canvas is a tool to create, design and implement a product
idea. This method is commonly used by businesses and entrepreneurs to find the
balance between customer profile and product design, but there are other cases
of use for this tool outside business
scope~\cite{pokorna2015value,meertens2012mapping}.

The process is divided in two parts, customer profile and value map, each of
these divided in other three parts:~\cite{osterwalder2014value}:

\begin{itemize}
\item \textbf{Customer profile:} This step is to identify the profile of the
  final user of the platform. This section is divided in three parts: 1)
  \emph{Customer jobs:} things the customer are trying to get done, 2)
  \emph{Customer pains:} undesired costs and situations, 3) \emph{Customer
    gains:} benefits, social gains and cost savings expected.
\item \textbf{Value Map:} This section is about what the final product has to
  have and what does not, and its also divided in: 1) \emph{Product and
    services:} which products and services are offered that help the customer
  get a job done, 2) \emph{Pain relievers:} how the customer pains are going to
  be alleviated, 3) \emph{Gain creators:} how the products and services create
  customer gains
\end{itemize}

We decided to use this methodology in the \bb{pre-project}(see
section~\ref{methodology:timeline}) for the definition of the final platform,
since it established the general development framework of the application. As a
result, the following items were the most relevant of the project's Value
Proposition Canvas:

\begin{itemize}
  \itbf{Customer Jobs:} Get recognition as researcher, share a research work,
  collaborate with peers, build a professional network, publish papers, be able
  to finish a researcher's PhD.

  \itbf{Customer Pains:} Slow reviews, publish in indexed journals,
  non-transparent process, takes more time than other jobs, low salaries.

  \itbf{Customer Gains:} Get to know other researchers, get knowledge, love for
  a research field, impact in the world, recognition as researcher.

  \itbf{Product and services:} Paper auto-formatter, facebook of researchers,
  distributed journal, automated state of the art generator, universal
  publishing platform, github for researchers.

  \itbf{Pain relievers:} Reviewer reputation, faster reviews, get donations for
  research, public and fair reviews, easier accreditation system.

  \itbf{Gain creators:} Free knowledge, open access, automatic review, find
  people in a research field, easier communication.
\end{itemize}

\subsection{Agile methodologies}
\label{sec:agile-methodologies}

Traditional software development methodologies are being eclipsed by new light
or agile methodologies. These methodologies are characterized by continuous
integration, iterative development and the ability to assume changes in business
requirements~\cite{boehm2005management,livermore2008factors}.

One of the most popular is known as \bb{Extreme
  Programming}~\cite{lindstrom2004extreme} based on a series of basic concepts
when carrying out the development of a program: code simplicity and rapid
prototyping, continuous customer communication with the development team,
responsibility of the code of all the members of the group, short and quick
meetings, refactoring and continuous integration~\cite{theunissen2005search}.

Another well-known method within agile methodologies is
\bb{SCRUM}~\cite{rising2000scrum}, which uses two-week frameworks to perform
development sprints and planning meetings. The use of these methodologies allows
developers to create better quality software in shorter periods of time and are
designed for small teams from three to nine developers.

These two methodologies were used to achieve the \bb{project implementation}
(see section~\ref{methodology:timeline}). Programming smart contracts in
Ethereum is a difficult task because once the source code is deployed in the
blockchain, there is no way to change it. For this reason, these methodologies
allowed me to develop small prototypes and run several test for each one. Every
two weeks I started a development cycle consisting on programming the next
prototype in the first week and running tests in the second week.

During the whole \bb{project implementation} phase, meetings were held with the
thesis directors at least twice a month to discuss the progress made and set
short-term objectives.

As a complement to the previous methodologies, \bb{Kanban} was used to manage
the status of the project throughout the development process. Kanban was created
by the company Toyota to be able to see the status of a project in an easy and
fast way, but nowadays it is widely used in the field of software
engineering~\cite{ahmad2013kanban}. Web services like
Trello\footnote{https://trello.com/} or Github project
boards\footnote{https://help.github.com/articles/about-project-boards/} feature
a digital Kanban to carry out this development process and both were used in
Decentralized Science.

\section{Technology}
\label{tech}

The proposed system relies upon two emerging distributed technologies. On the
one hand, the Blockchain~\cite{buterin2014ethereum} provides a public
decentralized ledger to record the system's interactions. On the other hand,
IPFS~\cite{benet_ipfs-content_2014} is a distributed file system to store all
the papers and reviews sent to the platform. This ensures that all the
information is persistent, free, accessible, and does not rely on a centralized
server.

% -------------------------------------------------------------------
\subsection{IPFS}
% -------------------------------------------------------------------
\label{tech:sec:ipfs}

IPFS stands for Interplanetary File System. It is a peer-to-peer file-sharing
protocol that uses a cryptographic hashes to store files in a distributed
network. IPFS works very similar to the HTTP protocol but in a BitTorrent way.
It can be seen as a giant git repository where everyone can store, share and
exchange files~\cite{benet2014ipfs}.

\subsubsection*{IPFS on Decentralized Science}

IPFS provides a robust, distributed and secure way to store files in a
decentralized network. These files are identified by a string of characters and
each identifier is unique. In order to achieve this IPFS forms the hash of the
data inside the file (a cheap computational operation) and outputs its
identifier. This hash behaves like a link within the IPFS network that allows
users to identify and recover the file.

This feature implies that two identical data files have the same hash, so they
have the same IPFS address, eliminating duplicates in an easy, secure and fast
way.

IPFS merges four main ideas: Distributed Hash Tables, BitTorrent, Git and
Self-Certified Systems.

\subsubsection*{Distributed Hash Tables}
\label{tech:sec:ipfs:dht}

A distributed hash table (\emph{DHT}) is a decentralized structure that works
very similar to a hash table. It consists on a table that behaves as a
collection of keys (hash strings) that identify items in a distributed database.
The table performs simple mathematical operations generating a random string
called \emph{hash}. The hash acts as a pointer that directs to the table's
information, allowing users to find it in a large database without performing an
exhaustive search~\cite{kaluszka2010distributed}.

In distributed hash tables, hash strings are keys that identifies a value in one
or more nodes. Any node can use a key to retrieve data. This system includes a
data structure called ``keyspace'' that is a set of all possible keys, which is
broadcasted across the nodes in the system. The mapping of the keys is made by a
function that calculates the \ii{keyspace} of each node and shares it with its
neighbors. These nodes also have and identifier and a set of identifiers
pointing to all its neighbors nodes. If a node is removed from the network, only
a small portion of the data must be recovered by other
nodes\cite{kaluszka2010distributed}.

This system makes \emph{DHTs} scalable, fast and robust. It is used by
frameworks such as Tapestry \cite{zhao2004tapestry}, Chord
\cite{stoica2001chord}, Kelips \cite{gupta2003kelips}, Kademlia
\cite{maymounkov2002kademlia} and IPFS \cite{benet2014ipfs}. These platforms are
similar in cost and performance if they are tested in a large enough network.
They behave very fast when it comes to searching for a key through massive
networks of nodes~\cite{li2004comparing}, that's why it is used by IPFS to
create its distributed file system.

\subsubsection*{BitTorrent - File sharing}
\label{tech:sec:ipfs:bt}
BitTorrent~\cite{cohen2003incentives} is a P2P file sharing system used
worldwide. In this system, files are divides into very small chucks of data, and
are shared in a peer-to-peer network. Each peer aims to maximize its download
rate by connecting to the best peers, meaning that peers with faster network
speed will be better than ones with slow connectivity. In BitTorrent's network,
peers with high upload rate will get higher download rate, so the key is
balancing the network bandwidth between downloading and uploading
files~\cite{pouwelse2005bittorrent}.

IPFS uses three main features from BitTorrent's protocol~\cite{benet2014ipfs}:
\begin{itemize}
\item BitTorrent's data exchange protocol rewards nodes who contribute to the
  network, and punishes the ones who don't.
\item BitTorrent tracks the availability of file chunks, sending the rarest
  first rather than sending the most common ones.
\item IPFS uses PropShare~\cite{levin2008bittorrent}, an alternative
  implementation of the original BitTorrent protocol designed to maximize
  network speed. This implementation improves the previous bandwidth allocation
  strategy for each peer, enhancing the download and upload speed of the
  network.
\end{itemize}

\subsubsection*{Git - Version control system}
\label{tech:sec:ipfs:git}

A Version Control System (\ii{VCS}) is a software to manage changes in a
document, computer programs or any information. Each change is called revision
and it is identified by a number, the person who did it and a timestamp. In a
\ii{VCS} revisions can be reverted to a previous version, making them useful for
software development.

Git is a distributed \emph{VCS}~\cite{torvalds2010git} that was born in 2005,
when the development process of the Linux kernel lost its version control
system. The Linux kernel is one of the biggest free software projects nowadays.
It has a great team of developers behind and the code usually changes very
frequently. In 2002 the team used BitKeeper\footnote{BitKeeper is an and
  distributed and scalable \ii{VCS} available at https://www.bitkeeper.org/} as
\ii{VCS} since they had a free license. But in 2005 when this license was over,
Linus Torvalds decided to develop his own \ii{VCS}~\cite{spinellis2005version}.

Git was designed to be scalable and distributed, and nowadays is widely used by
the open source community. The most important factors that IPFS inherits from
Git are~\cite{benet2014ipfs}:

\begin{itemize}
\item Git reflects changes in a file system in a distributed way using an
  acyclic graph, in which each revision is a node and each change is an arc.
\item Objects are identified by the cryptographic hash of their contents.
\item Version changes only update preferences and add objects. To broadcast
  version changes, git only needs to transfer the new objects and update the
  remote references.
\end{itemize}

\subsubsection*{Self-Certifying File Systems}
\label{tech:sec:ipfs:scfs}

\todo{explain more} A self-certifying file system (\emph{SCFS}) is a secure and
decentralized file system that uses public keys to map file names, separating
key management from file system security. Servers have a public key and clients
use the server public key to authenticate the server and establish a secure
communication channel. Once the client has verified the server a secure channel
is established and the actual file access takes place.~\cite{mazieres2000self}.

IPFS tries to connect these ideas into a cohesive, trustful and decentralized
file system. It is build on top of a peer-to-peer network, so no nodes are
privileged, and all of those store IPFS objects in local storage. These objects
represent files or other data structures.

In Decentralized Science, we use this technology to store all the files in a
robust and secure way, without relying on a centralized server. All these
characteristics provides a persistent platform, since IPFS works on a network of
thousands of nodes.

\subsection{Ethereum}
\label{tech:sec:ethereum}

As seen in the section~\ref{tb:eth}, Ethereum is a technology that allows its
users create fully decentralized and autonomous applications. These applications
(called DApps) are smart contracts that are uploaded to the Ethereum blockchain,
so it is not necessary to have a server to run or communicate with these
contracts.

Ethereum is used as \ii{backend} in Decentralized Science. All the internal
operation of the platform is programmed in a smart contract and is executed from
blockchain. This implies several important things:

\begin{itemize}
  \itbf{Open source:} All the source code uploaded to Ethereum is free, public
  and anyone can fork it in other projects.

  \ifbf{Auditable interactions:} All calls and interactions to the platform are
  registered in blockchain, making the entire process of scientific publication
  auditable.

  \itbf{Distributed platform:} All the source code is executed in the Ethereum's
  distributed network, meaning that Decentralized Science does not need a
  centralized server or a third party service to run the platform.

  \itbf{Free access to information:} Anyone can access to the platform's
  information (journals, authors, papers, reviews and reputation) without any
  cost and through a web page.
\end{itemize}

With Ethereum and IPFS, Decentralized Science is a 100 \% distributed platform,
open source and free\footnote{Interactions with the platform may have
  transaction costs within the Ethereum network (see section~\ref{arch:trans})
  \nopagebreak} to all users.


\subsection{Remix}
\label{sec:remix}

\figura{remix.png}{width=0.95\linewidth}{tech:sec:remix:sc}{Remix smart contract
  web compiler}Remix\footnote{https://remix.ethereum.org/} is an online smart
contract web compiler designed by the Ethereum community that allows users
create, compile and deploy contracts both in a test net or in the Ethereum's
blockchain.

This compiler offers the possibility to try the functionality of the contracts
through a virtual machine that simulates the blockchain's behavior called EVM
(\emph{Ethereum Virtual Machine})~\cite{hildenbrandt2017kevm}. When deploying a
contract with this platform, a series of HTML elements are generated, simulating
a real deployment and allowing developers to interact with the contract, without
the need of developing a frontend to try the contract behavior.

Remix also has a series of interesting features: a debugger to follow the
execution process of a transaction, a gas calculator that estimates the gas cost
of each transaction, a code analizer to detect possible vulnerabilities and an
ABI export system (see section \ref{jsmm}).

This tool facilitate the process of smart contracts developments, since the web
service allows to try different implementations of the same contract doing only
small changes, without the need of installing any software (see figure
~\ref{tech:sec:remix:sc}).

\subsection{Testrpc}
\label{testrpc}

\figura{testrpc.png}{width=0.7\linewidth}{testrpc:sc}{Transactions mined by
  testrpc in a test network}

Testrpc is a Node.js based Ethereum client for testing a development. It
simulates an Ethereum client behavior using the ethereum.js library. Once
installed throug \code{npm}, it can lauched through the linux terminal with the
command \code{testrpc} and offers the developers a series of interesting
features:

\begin{itemize}
  \itbf{Simulate accounts:}
  Testrpc\footnote{https://www.npmjs.com/package/ethereumjs-testrpc} default
  command launches 5 accounts, but it offers the possibility to add as many as
  needed with the option ``--accounts'', this allowed this project to simulate
  the interaction of 1000 accounts communicating with the contracts''. Example:
  \code{testrpc -a 1000}

  \itbf{View transactions:} Once testrpc is lauched, developers can see through
  the console all the transactions made with the blockchain in a user-frendly
  way. It also has a gas\footnote{Amount of money needed to execute a
    transaction (see section~\ref{ts:at})} estimator, a tool that calculate the
  price of each transaction (see figure \ref{testrpc:sc}).

  \itbf{Create test net:} Testrpc creates a test net through the port 8545. Any
  user in the internet using Metamask (see section \ref{jsmm}) can connect to
  this test net and interact with the contracts deployed.

  \itbf{Give ETH to an existing account:} If an user already has an Ethereum
  account and wants to connect a testrpc node, the administrator can assign ETH
  to that account to allow the user do transactions in the test net.
\end{itemize}

\subsection{JavaScript and Metamask}
\label{jsmm}

In order to interact with a smart contract in Ethereum, users normally use a web
application connected to the contract address (see section~\ref{ts:at}).
Nevertheless, the web application must be connected to an Ethereum node in order
to communicate with it. There are two different ways to achieve this. The first
one is to install an specialized web browser like
AlethZero\footnote{https://github.com/ethereum/alethzero},
Mist\footnote{https://github.com/ethereum/mist} or
Brave\footnote{https://brave.com/} that provide the libraries needed to interact
with smart contracts and connects to the blockchain automatically. The second
one is installing an extension like Metamask\footnote{https://metamask.io/} as
an extension of the web browser.


Metamask is a web browser extension (available for Chrome, Firefox, and Opera)
that allows its users to connect both to the Ethereum network or a custom test
net. It contains the mechanism needed to make transactions and communicate with
smart contracts in the blockchain. To use it, users only need to install the
extension from the Metamask homepage and configure it with an Ethereum account.
Metamask also allows the creation of new accounts, but in order to make
transactions with the main Ethereum blockchain, these accounts must have enough
funds.

Thus, Metamask not only acts as a bridge between a web browser and the Ethereum
network, it also works as a wallet and supports Ethereum's official tokens.

JavaScript is used to connect to a smart contract address and be able to call to
specific functions of it. To do so, the JavaScript code must have two important
data:

\begin{itemize}
\item \textbf{Ethereum's contrat address:} When the user loads the web page to
  connect to the contract, JavaScript must have the address in which it is
  stored, because all the interactions with the platform are made through
  transactions to this address.
\item \textbf{Contract's ABI:} An ABI (Application Binary Interface) is a data
  structure in which JavaScript can find all the methods a contract has. Each
  method defines its inputs, outputs, type, if its payable and a state.
  JavaScript uses this information to build the transactions to call these
  methods (see listing \ref{abi}).

\end{itemize}

These two technologies were used to test the platform proposed in this work,
using HTML and Testrpc to simulate a functional journal (see chapter~\ref{poc})


\begin{lstlisting}[frame=single,caption=ABI fragment example,label=abi]
 {
 "constant": false,
 "inputs": [ 
    { 
       "name": "_reviewerAddress",
       "type": "address"
    },
    { 
       "name": "_reputation", 
       "type": "bool" 
    } 
  ], 
  "name": "giveRep", 
  "outputs": [], 
  "payable": false, 
  "stateMutability": "nonpayable", 
  "type": "function" 
  } 
\end{lstlisting}


\subsection{Github}
\label{sec:github}

Github\footnote{https://github,com} is a web hosting service for \ii{VCS} using
Git (see section~\ref{tech:sec:ipfs:git}). If offers all Git features as well as
its own features like bug tracking, issues and task management, kanban boards,
wiki pages among others.

Decentralized Science is 100\% open software and uploaded to Github. All the
source code, documents, web page, and all the information is available at:

\begin{lstlisting}[frame=single]
  https://github.com/DecentralizedScience/
\end{lstlisting}
%%% Local Variables:
%%% mode: latex
%%% TeX-master: "../Tesis.tex"
%%% End:

\chapter{Platform Description}

\begin{FraseCelebre}
  \begin{Frase}
    The needs of the many outweigh the needs of the few
  \end{Frase}
  \begin{Fuente}
    Spock - The Wrath of Khan
  \end{Fuente}
\end{FraseCelebre}

We propose a blockchain-enabled decentralized publication system for open
science. It consists of three main components that decentralize and try to
improve three different aspects related to scientific publication:

1) Peer review governance communication is traditionally centralized and
controlled by editors and publishers. Our proposal opens and decentralizes these
communications making the process more transparent.

2) Peer reviewer quality and reliability information is difficult to
predict~\cite{callaham_relationship_2007}, and it is usually hold private by
publishers and journals. The system proposes to open this information through a
decentralized reputation network of peer reviewers over a blockchain.

3) Scientific papers are traditionally obtained or bought from a centralized
publisher. We propose a decentralized network to distribute academic works and
promote free access to science.

These ideas are further discussed in the following sections.

% \section{Decentralized publication system for open science}

% We propose a blockchain-enabled decentralized publication system for open
% science. It consist of three components that decentralizes three different
% aspects of science publication.

% 1) Peer review governance communication is traditionally centralized and
% controlled by editors and publisher. Subsection \ref{workflow} further explore
% the opening and decentralization of this communication with blockchain
% technology.

% 2) Peer reviewer quality and reliability information is difficult to
% predict~\cite{callaham_relationship_2007}, and is usually hold private by
% publishers and journals. The system propose to distribute and open this
% information though a decentralized reputation network of peer reviewers over
% blockchain. Subsection \ref{reputation} further explore this proposal.

% 3) Open access papers are traditionally obtained from a centralized publisher
% infrastructure. Subsection \ref{distributedOA} explores the decentralization
% of this infrastructure.

\section{Transparent Peer Review Governance}
\label{workflow}

The system provides a platform for the peer review process communication, from
paper submission to paper acceptance or rejection. It registers all the
interactions into a blockchain based distributed
ledger. % and provides open access to all relevant content, from the different versions of the paper to the submitted reviews.

The interaction diagram of the system (Figure \ref{InteractionDiagram})
describes the interactions of the supported peer review governance. Following,
this interactions and their implementation are described.

\figura{Uml.png}{width=0.9\linewidth}{umlseq}{Sequence diagram of platform
  interaction}

\begin{LaTeXdescription}
\item[Paper submission] A paper submission is registered by submitting the IPFS
  address of the paper to an Ethereum contract. Then, the Ethereum sender
  address is recorded as the corresponding author, and the submission is
  timestamped in the blockchain.

\item[Reviewer proposal] A journal editor may invite a peer reviewer to review a
  specific paper. The transaction will record the Ethereum address of the
  reviewer and optionally, a deadline to submit the review.

\item[Reviewer acceptance/rejection] An invited reviewer may accept or reject
  the review of a paper. The response will be recorded into the blockchain.

\item[Submit review] A reviewer should make a transaction to deliver the review.
  The transaction will record the acceptance/rejection and the IPFS address of
  the detailed review.

\item[Rate review] A novelty of the system further discussed in Section
  \ref{reputation} is the rating system for reviews. The transaction will record
  the sender address and the rating as well as the rated review and reviewer
  addresses.
\end{LaTeXdescription}

\section{The Peer Review Reputation network}
\label{reputation}

The system proposes the use of a peer review reputation network were the quality
of peer reviews is rated by the authors, editors and reviewers of the system.
The work extends traditional peer review governance with the possibility of
rating the reviews, building a reputation system for
reviewers~\cite{resnick2000reputation}. Reviewers get rewarded for worthy, fair,
an timely reviews, or penalized otherwise.

This network of peer reviewers would enable a better reviewer selection, a fair
recognition of reviewers work and a protection against unfair reviews for
authors. However, it could also rise privacy concerns for both reviewers and
raters~\cite{van1999effect,schaub2016trustless}. We consider these privacy
issues in section \ref{privacy}.

% With this network of rated peer reviews, different metrics of the peer
% reviewers can be provided by the system. Journals can benefit from this
% network by searching for well rated reviewers that respond on time, authors
% can expect shorter review time and forget about unaccountable bad reviews and
% reviewers can get their work publicly recognized. This approach can suppose a
% privacy problem for both reviewers and raters. We contemplate these problems
% and how to solve them in section \ref{PrivacyReviewRating}.

\section{Distributed Open Access infrastructure}
\label{distributedOA}
% \commant{Copied from PEERE Paper.TODO rewrite}
Open Access focuses in the free access to scientific knowledge. While publishers
provide free of charge their Open Access content, their control of the science
dissemination infrastructure allows them to impose certain rules, such as
charging authors unreasonable fees to offer their work as Open
Access~\cite{solomon2012study} (Gold Open Access) or the temporal embargo and
restrictions on the dissemination of the final version (Green Open
access)~\cite{bjork2014anatomy}, among others.

The system proposes a decentralized infrastructure for science publication.
Academic documents - from first drafts to final versions, including peer
reviews- are shared in IPFS, an open P2P network~\cite{benet_ipfs-content_2014}.
Thus, the system inherently grants Open Access by the design of its distributed
infrastructure and circumvents the publishers dominant role.

\section{Privacy Settings of Open Peer Review and Rating}
\label{privacy}
Anonymity of reviewers and authors in peer reviews is traditionally used to
improve the fairness of the process. Thanks to single blind reviews, anonymous
reviewers can honestly critic a paper without fearing the reactions of the
authors. Double blind reviews also allow to reduce the impact of personal
biases. Finally, open review models propose that both authors and reviewers know
each other. These different privacy settings are shown in the left part of
Figure \ref{PrivacyReviewRating}.

\figura{privacyReviewRating.jpg}{width=0.9\linewidth}{PrivacyReviewRating}{Review
  and Rating privacy models}

Note, however, that the anonymity of the reviewers can be also abused. Unfair
and low quality reviews are not discouraged by the system due to the lack of
consequences. In order to alleviate this problem, our system proposes the
construction of a reputation network of peer reviewers so that reviewers are
awarded or criticized according to their work. This reputation network can also
adopt different privacy settings, allowing both anonymous and signed ratings of
either signed or anonymous reviews as depicted on the right side of Figure
\ref{PrivacyReviewRating}.

The implementation of these different privacy settings in a blockchain requires
different approaches. The question of whether we can keep the benefits of blind
review while providing accountability and recognition to reviewers deserves
special consideration. Next, we discuss the different privacy models for review
and privacy settings.

% Anonymity of reviewers and authors in the peer review process is a simple tool
% traditionally used to improve the fairness of the process. Thanks to single
% blind reviews, anonymous reviewers could honestly critic a paper without
% fearing authors reactions. Double blind reviews allow to reduce the impact of
% biases in the review. Open reviews model however propose that both authors and
% reviewers are known. Left part of Figure \ref{PrivacyReviewRating} illustrate
% these different privacy settings of peer
% review. %\commant{Mañana por la mañana tenemos una versión profesional del diagrama}.

% As already discussed, the anonymity of the reviewers can also be abused.
% Unfair or low quality reviews were not discouraged by the system due to the
% lack of consequences.

% The system propose the construction of a reputation network of peer reviewers.
% This reputation network can also adopt different privacy settings, allowing
% both anonymous and signed ratings of either signed or anonymous reviews as
% depicted in the right side of Figure \ref{PrivacyReviewRating}.

% Each of the anonymity options of the system requires different solutions,
% which are discussed bellow. The question of whether we can keep the benefits
% of blind review while providing accountability and recognition to reviews
% deserves special consideration.

\subsection*{Blind peer review}
Blind review is the protection of the identity of reviewers in the peer review
process. In a blockchain, this protection could be easily achieved by using
single-use addresses previously agreed with the editor.

\subsection*{Double blinded peer review}
A double blinded review is a blind review that additionally protects the authors
identity to prevent social bias~\cite{lee2013bias}~\cite{budden2008double}.
Authors could protect their identities prior to publication by providing a
single-use public address on submission. Later they can reveal their real
identity since they are the only ones with access to that address.

\subsection*{Open peer review}
Open evaluation proposes the de-anonymization of all the parties involved in the
peer review process~\cite{ford2013defining}. While studies found effect on the
percentage of reviewers declining to review~\cite{van1999effect} other
implications remain open to debate~\cite{groves2010open}.

% \subsection*{Blind peer review}
% Blind review is the protection of the identity of reviewers in the peer review
% process. In a blockchain, this protection could be easily achieved by using
% single use addresses or passwords agreed with the editor.

% \subsection*{Double blinded peer review}
% A double blinded review is a blind review that additionally protects the
% authors identity to prevent social
% bias~\cite{lee2013bias}~\cite{budden2008double}. Authors could protect their
% identities prior to publication by providing a single use public key from
% which later they sign their real identity or signing the paper with the hash
% of their names followed by a random constant, revealing the constant after
% acceptance.

% \subsection*{Open peer review}
% Open evaluation proposes the opening and deanonymization of peer
% review~\cite{ford2013defining}. While studies found effect on the percentage
% of reviewers declining to review~\cite{van1999effect} other implications
% remain open to debate~\cite{groves2010open}.

% Signed reviews are easy to implement by maintaining a public identity for the
% reviewer.

\subsection*{Open Rating}

Similarly to open reviews, open ratings are easy to implement by maintaining a
public identity for the raters.

\subsection{Anonymous Rating}

Protecting the identity of raters is interesting in several reputation systems.
We can support this anonymity feature using \emph{blinded tokens}
~\cite{schaub2016trustless} that grant permission to rate without revealing the
identity of the rater. People authorized to rate a review, such as authors,
editors and other reviewers involved in the process, may each get one of these
tokens.

\subsection*{Rating anonymous reviews}

In a system that support voluntary signing of reviews, unsigned reviews would
not affect the reviewers reputation unless they acknowledge their authorship.
Thus, reviewers may only reveal their identities for well rated reviews,
reducing the desired accountability for poor quality, unfair or late reviews.

A system allowing anonymous, yet accountable, reputation system for peer
reviewing is therefore of great interest. Following, we discuss the feasibility
of adopting different anonymity approaches to realize this system.

\emph{Collateral models} are widely used in blockchain technology to ensure that
an actor assumes negative consequences of an interaction in order to avoid the
greater consequence of loosing the collateral. A similar strategy can be used
for the anonymous reputation network. If a reputation collateral is requested
from the reviewers, they would be encouraged to claim even negative ratings.
This model can be combined with anonymity measures to ensure accountable yet
anonymous, peer reviews.

\emph{Coin mixing} protocols are designed to obfuscate the relation between
senders and receivers of Bitcoin payments by mixing in a single transaction many
senders and receivers ~\cite{meiklejohn2015privacy}. We can not directly apply
this approach to rate reviews since the receiver identity is known. However it
can be used in collaboration with other techniques discussed below.

\emph{Reusable payment codes} enable the possibility of using a large amount of
addresses to receive a payment ~\cite{harrigan2016unreasonable,
  ranvierReusable}. Reviewers may share one of this addresses with each of the
actors with permission to rate and then collect the reputation probably using an
anonymity layer such as coin mixing. Using a collateral model would encourage
the acceptance of bad ratings.

\emph{ZK-Snarks} are a cryptographic tool enabling to prove a statement without
revealing anything else than the statement is in fact true (Zero-Knowledge Proof
of Knowledge) ~\cite{blum1988non,bitansky2013succinct}. They also provide this
property in a succinct and non-interactive fashion (i.e. using a relatively
small proof and not requiring further communication between prover and
verifier). Zcash uses this technology to build an anonymous
cryptocurrency~\cite{sasson2014zerocash}. A similar approach could be used to
manage anonymous ratings. A reviewer could also receive the rating of a review
she did without revealing from which review or which rating the reputation
comes. As before, a collateral model can be used to encourage the acceptance of
low ratings.

\section{Architecture}

The platform's architecture consists in two main parts as explained in section
\ref{tech}, a decentralized file system in which users can upload all the files
using IPFS (see section \ref{tech:sec:ipfs}), and a smart contact to register
all interactions of the users with the platform (see section
\ref{tech:sec:ethereum:sm}).

As a decentralized technology, anyone can run a node locally, connecting to the
IPFS network and to the Ethereum's blockchain to interact with the platform, but
this technology is not used commonly, and not all users have the knowledge to
install and run these programs. As a solution, a ``gateway server'' to test the
platform's implementation was created, using an web browser extension called
Metamask\footnote[1]{https://metamask.io/} to interact with the blockchain and
running an IPFS node to upload the files as explained in the diagram of the
figure \ref{plat:sec:architecture:diagram}.

\figura{architecture.jpg}{width=0.9\linewidth}{plat:sec:architecture:diagram}{Architecture
  diagram of a node with IPFS and an Ethereum light client}

\subsection{Smart Contract Architecture}

Smart contracts in Ethereum can interact with each other, creating an ecosystem
of programs that resemble object oriented programming. The contract structure
and source code is crucial because once a contract is in the blockchain, there
is no way to change it.

Uno de los retos más importantes para diseñar un contrato es reducir el coste de
las transacciones, ya que el coste de estas puede ser muy alto si el diseño del
contrato es ineficiente~\cite{croman2016scaling}. Insertar datos en la
blockchain es muy caro, por eso es recomendable utilizar las estructuras de
datos que nos ofrece Ethereum para reducir el coste de transacción. El problema
se plante a cuando se utilizan direcciones en IPFS, las cuales están en BASE58,
un tamaño que sólo es representable por el tipo ``string'' de Ethereum, el cual
es de los tipos de datos maś costosos para trabajar.

Como solucion, la plataforma realiza la funcion hash en base32 de la dirección
IPFS, convirtiendo el tipo de dato a uno mucho más barato para reducir en un
30\% el coste de transacción.

La estructura dentro de Ethereum está separada en cuatro contratos inteligentes:

\begin{itemize}
\item \textbf{Decentralized Journal:} Es el contrato desde el que se controlan
  todas las interacciones con los autores y los editores. Los autores
  interactuan con este contrato para enviar los papers, y los editores asignan a
  revisores para el proceso de revision por pares.
\item \textbf{Decentralized Library:} Este contrato es el que almacena las
  direcciones IPFS de los papers y controla si están aceptados por el journal o
  no. Es el que se encarga de convertir una dirección de base58 a bytes32 para
  que sea manejada por la plataforma.
\item \textbf{Reviewer Reputation Hub:} Se encarga de almacenar las direcciones,
  los campos de investigación, y la reputación de los revisores que actuan con
  la plataforma. Con este contrato interactuan los editores que quieran
  encontrar a un revisor para realizar el proceso de revisión por pares.
\item \textbf{Reviews Library:} Es el contrato en el que se almacenan las
  reviews que hacen los revisores de los papers. A través de este contrato, se
  pueden puntuar las revisiones, implicando la ganancia o perdida de la
  reputación del revisor que la ha realizado.
\end{itemize}

\figura{contracts.png}{width=0.8\linewidth}{plat:sec:scarchitecture:uml}{Diagrama
  UML del la estructura de contratos}

El diagrama \ref{plat:sec:scarchitecture:uml} intenta ilustrar la estructura de
los contratos a través de un UML, si asemejamos los contratos inteligentes de
ethereum como objetos.



%%%
%%% Local Variables:
%%% mode: latex
%%% TeX-master: "../Tesis.tex"
%%% End:

\chapter{Discussion}

\begin{FraseCelebre}
  \begin{Frase}
    The needs of the many outweigh the needs of the few
  \end{Frase}
  \begin{Fuente}
    Spock - The Wrath of Khan
  \end{Fuente}
\end{FraseCelebre}

Como resultado del desarrollo de esta plataforma, el escenario ideal sería que
algunos journals que ya hayan ido migrando al sistemas de publicación
alternativos como los explicados en la sección~\ref{soa:aps} se adapten a esta
plataforma. Al realizar un pequeño análisis del posible impacto de este trabajo
hay dos puntos importantes a destacar.

\section{Monetary Impact}
El impacto monetario sería uno de los más notables tras la implantación de este
sistema en sistemas de publicación científica de hoy en día. Según datos de
investigaciones al respecto, el coste de publicación de un artículo en una
revista de imacto varía de entre 1000\$ hasta los
5000\$~\cite{van2013true,russel2008business}, coste muchas veces inviable para
investigadores que quieran avanzar en la investigación científica.

El coste de la publicación y el acceso a la ciencia a través de el trabajo
propuesto sería únicamente variable en función del precio del ETH\footnote{La
  criptomoneda de Ethereum explicada en la sección \ref{tech:sec:ethereum}}.
Tras realizar un análisis con varias versiones de la plataforma desplegada se
determina que el coste de una transacción varía entre 100000 y 150000 de
gas\footnote{Gas es lo que pagas como comisión a la red de Ethereum por ejecutar
  una transacción}.

Teniendo en cuenta que para que se publique un paper han de realizarse como
mínimo 5 transacciones se puede determinar que el precio actual para publicar un
paper ronda entre los 4\$ y 6\$ segun datos de Ethereum Gas
Station\footnote{Precio de una transacción en Ethereum
  https://ethgasstation.info/}, más de 250 veces más barato que los sistemas de
publicación actual en el mejor de los casos.

\section{Review Time and Quality Impact}

Otro de los impactos importantes sería la reducción del tiempo y el aumento de
la calidad en el proceso de revision por pares.

Si se dispone de una masa crítica de usuarios de la plataforma propuesta, se
crea un ecosistema de usuarios que alimentan tanto la red de reputación de
revisores como los contratos de publicación científica (ver
section~\ref{contracts}). El proceso de Peer review se vería afectado de dos
maneras:

\begin{enumerate}
\item \textbf{El tiempo de revisión:} Los contratos inteligentes permiten
  establecer tiempos límites para la revisión de un artículo, suponiendo una
  penalización a los revisores que no cumplan estos plazos (ver sección
  \ref{reputation}). Si un Decentralized Journal tiene unos tiempos de revisión
  establecidos, y los revisores que asignan aceptan las revisiones,
  probablemente se experimente una mejoría en el tiempo de entrega de las
  revisiones y por lo tanto en el proceso de publicación, con respecto a los
  tiempos muchas veces excesivos de los sistemas de publicacion
  actual~\cite{huisman2017duration}.
\item \textbf{La calidad de las revisones:} Todas las revisiones son rateables
  por la comunidad y afectan directamente a la reputación del revisor, así que
  es altamente probable que la calidad de las revisiones en el sistema sufra una
  mejoría y desaparezcan muchos de los problemas respecto a la revisión por
  pares comentados en la sección \ref{intro}.
\end{enumerate}

\section{Science Distribution}

Toda interacción con la plataforma ha de realizarse mediante una cuenta Ethereum
(ver section \ref{tech:sec:ethereum}) y quedan grabadas en la blockchain de
este. Esto implicaría que a través de la dirección de un investigador científico
se puedan obtener dato de todos los papers que ha publicado y revisado. Todo la
comunidad científica podría sufrir una mejor gracias a este sistema. Además, los
nuevos investigadores que quieren empezar su carrera en el mundo académico
pueden obtener visibilidad si los revisores que revisan los papers que envían a
los Distributed Journals tienen alta reputación o no.

Si el sistema se implantara de manera exitosa, la comunidad científica empezaría
a cuestionarse la existencia de los publishers, y si estos desaparecieran, se
encontrarían nuevas formas de financiación de proyectos.

\section{Problems}
\figura{chart.png}{width=0.99\linewidth}{prob:txfee}{Ethereum transaction fees evolution}


Hay varios problemas para implantar este proyecto hoy en día, ya que las
tecnologías propuestas todavía tienen poda expansión y son poco conocidas por
el usuario medio.

La primera es el cambio de plataforma para los investigadores de la comunidad,
ya que la costumbre de utilizar las plataformas de hoy en día es dificil de
cambiar, por lo que proponer un cambio en los sistemas de comunicación y
revisión puede suponer un gran rechazo inicial, pudiendo llevar al proyecto a un
punto muerto, sobretodo si la metodología de conexión a la blockchain de
Ethereum es compleja actualmente.

La segunda es el precio de las transacciones. Ethereum es una moneda que fluctúa
bastante, y ultimamente se ha experimentado una gran crecida de todas los
precios de las criptomonedas con respecto a hace seis meses. Las subidas en el
precio provocan subidas en las transacciones, que a su vez provoca que la
interacción sea más cara para todos los usuarios que la utilizan.




%%%
%%% Local Variables:
%%% mode: latex
%%% TeX-master: "../Tesis.tex"
%%% End:

\chapter{Conclusions and future work}

\begin{FraseCelebre}
  \begin{Frase}
    The needs of the many outweigh the needs of the few
  \end{Frase}
  \begin{Fuente}
    Spock - The Wrath of Khan
  \end{Fuente}
\end{FraseCelebre}

% -------------------------------------------------------------------
\section{Cool Section}
% -------------------------------------------------------------------

Lorem ipsum dolor sit amet, consectetur adipiscing elit. Praesent fermentum orci
a justo sagittis, at tincidunt enim luctus. Curabitur imperdiet mauris sed
mattis semper. Aenean augue risus, viverra vel porta a, auctor ac enim. Sed quis
auctor tellus. Suspendisse potenti. Vestibulum nec lectus turpis. Morbi luctus
eros ante, eu consequat magna maximus ut. Donec nec dui sagittis, ornare lorem
a, condimentum odio. Vestibulum ante ipsum primis in faucibus orci luctus et
ultrices posuere cubilia Curae; Nunc quis ipsum eget tellus placerat facilisis.
Curabitur tortor nunc, elementum at imperdiet a, hendrerit id augue. Aenean
vitae lacus eget diam posuere aliquet vel in elit.

Aenean purus est, tempus eget tristique nec, fringilla a enim. Sed in volutpat
eros. Sed commodo congue metus ac aliquam. Quisque auctor dolor libero, vitae
mattis ante luctus sit amet. Aliquam iaculis urna nec lorem rhoncus, commodo
dignissim mi ultrices. Mauris sem augue, luctus ac tincidunt id, sollicitudin at
ipsum. Suspendisse mattis venenatis dolor. Pellentesque velit sem, pulvinar
vitae tortor tempor, blandit aliquet velit. Nunc mattis urna diam, ac maximus
libero molestie non.

Orci varius natoque penatibus et magnis dis parturient montes, nascetur
ridiculus mus. Quisque id egestas tellus. Nunc suscipit ex ac quam pretium, sit
amet vehicula risus hendrerit. Suspendisse faucibus ante metus, sed pulvinar
odio pulvinar condimentum. Donec vel arcu egestas, posuere metus quis, varius
mauris. Suspendisse ut ligula id justo eleifend vestibulum sit amet faucibus
neque. Praesent dignissim risus quis consectetur porta. Maecenas faucibus velit
non pretium ullamcorper. Praesent fringilla pharetra purus. Aenean ullamcorper
nisi gravida sagittis dapibus. Nunc commodo arcu nec cursus venenatis. Integer
vel turpis convallis, feugiat sapien id, euismod arcu. Aliquam sit amet iaculis
lacus. Vestibulum ut ex ac libero efficitur varius at in nisl. Morbi vel posuere
diam, a porttitor turpis.

Pellentesque non justo est. Nunc luctus ullamcorper tincidunt. Aliquam eu nisl a
orci aliquam cursus. Mauris id sollicitudin mauris. Suspendisse quis dolor id
magna porttitor mollis eu sit amet diam. Donec a elementum nibh. Sed egestas id
sapien nec aliquet. Vivamus porta dignissim bibendum. Donec at odio volutpat,
vestibulum elit at, fringilla ipsum. Donec vehicula lectus efficitur est
fermentum, ac tincidunt quam maximus. Proin lectus sem, sodales quis nunc ut,
maximus ullamcorper erat. Mauris lobortis justo quis malesuada viverra. Sed
suscipit, elit quis tincidunt condimentum, ipsum augue ornare velit, sit amet
fermentum leo orci ac risus. Vestibulum elementum viverra porta.

Sed gravida, risus nec scelerisque egestas, metus elit maximus leo, sit amet
lacinia erat purus at mauris. Aliquam eget libero velit. Proin luctus risus et
maximus efficitur. Praesent eget cursus ipsum. In hac habitasse platea dictumst.
Duis fringilla purus eu enim hendrerit, ut gravida tortor scelerisque. Donec
quis pellentesque ex, in auctor turpis. Ut non turpis purus.

Duis porttitor turpis purus, eget volutpat diam posuere eget. Morbi porttitor
risus quis tortor pellentesque varius. Nullam eu odio a augue tincidunt cursus.
Proin semper sapien augue, sit amet maximus turpis vestibulum ac. Quisque semper
justo nunc, sed efficitur augue ullamcorper ac. Praesent egestas eget neque quis
consectetur. Ut quis nulla rutrum, placerat leo nec, euismod quam. Phasellus
dapibus ligula vitae lacus lacinia blandit. Morbi vel ligula iaculis, aliquet ex
ut, scelerisque sem. Donec rutrum lacus quis odio vulputate ultrices. Quisque
ullamcorper rhoncus mauris, ac vestibulum magna blandit et. Suspendisse eu diam
rhoncus, sagittis quam vitae, maximus est. Donec augue diam, euismod a elit ac,
imperdiet accumsan nunc.

Aenean quis metus sed urna fringilla scelerisque. Proin tellus lectus, laoreet
tincidunt commodo eget, euismod sit amet nisi. Morbi eu massa eu arcu lobortis
malesuada. Orci varius natoque penatibus et magnis dis parturient montes,
nascetur ridiculus mus. Donec ut enim vel lacus eleifend suscipit. Curabitur id
ex vitae leo tempor elementum. In vestibulum mi eu ligula sagittis, eu efficitur
est auctor. Morbi ornare molestie rutrum. Aliquam sit amet fermentum enim. Fusce
sed tempus sem. In elementum dolor nec justo tempus hendrerit. Cras sit amet
cursus est, sit amet porta lacus. Nulla pretium at dolor quis volutpat.
Pellentesque condimentum ultricies urna, sed aliquam ipsum. Praesent consequat
faucibus massa id porta. Nullam nibh purus, maximus eu tristique ut, convallis
sit amet sapien.

Integer ac nisi leo. Quisque sollicitudin eros lorem, in suscipit orci congue
eget. Suspendisse gravida augue nec faucibus interdum. Integer blandit ligula
nec ex dapibus sodales. Aenean accumsan ante nibh, ac varius nisl pharetra in.
Duis mollis convallis neque, vitae finibus dui tincidunt ac. Aliquam a leo sed
elit porttitor dignissim blandit sit amet mauris. Proin viverra id ex quis
maximus. Etiam ultrices tristique faucibus. Donec egestas lorem enim, vel rutrum
erat interdum eu. Vestibulum turpis libero, aliquet non neque nec, placerat
tempus nulla.

Phasellus accumsan lacus ut enim rutrum posuere. In sit amet enim mi. Nulla et
fringilla justo. Etiam ut posuere velit, non fringilla risus. Praesent mattis
diam nec convallis finibus. Donec malesuada felis eu consectetur volutpat. Nulla
mattis neque eu arcu fermentum, ac accumsan quam placerat. Maecenas iaculis
blandit leo, ut semper orci vestibulum et. Quisque ut lorem pretium, condimentum
lorem vel, volutpat nisl. Maecenas ut enim sed elit maximus aliquam eget in
diam. Donec vulputate nunc sed diam accumsan tempus.

Phasellus mollis tempor lectus vitae pretium. Praesent ullamcorper suscipit est
eleifend aliquet. Nulla nec diam consectetur, venenatis arcu non, finibus odio.
Ut vel nisi condimentum, imperdiet enim id, ultrices felis. Duis congue dui
sapien, eu mattis ipsum mollis at. Donec a ex a orci tincidunt fermentum.
Aliquam pretium non orci nec pharetra. Donec tellus erat, maximus et
sollicitudin eget, elementum a mauris. Phasellus tempus mauris vitae tortor
elementum, eleifend viverra nisl luctus.

Aliquam in est sed velit elementum dictum. Phasellus auctor est sit amet egestas
vulputate. Nunc volutpat nibh lacus, eu dignissim libero gravida sit amet. Sed
accumsan dui sit amet leo pretium, id porttitor nisl fringilla. Mauris vehicula
pretium cursus. Integer bibendum tellus a ante eleifend tempor. Phasellus vitae
efficitur felis, a vulputate nulla. In finibus massa ut ante consequat auctor.
Phasellus elementum odio eget viverra lobortis. Lorem ipsum dolor sit amet,
consectetur adipiscing elit. Nam et quam et tellus ullamcorper consequat non non
mauris. Sed molestie ut turpis quis consectetur. Ut hendrerit justo eget orci
faucibus dignissim. Aenean ut viverra nulla. Proin vestibulum vehicula dapibus.

Cras posuere laoreet dui sit amet bibendum. Phasellus mollis euismod sapien eget
iaculis. Donec vel commodo ante. Sed vitae elit eget felis suscipit rhoncus. Nam
lacus leo, porttitor et vestibulum sodales, convallis eget sapien. Fusce quam
tellus, finibus eget feugiat at, ultrices sed risus. Cras ac semper odio.

Cras egestas rhoncus lorem ac cursus. Sed ut ipsum ut urna commodo tincidunt sit
amet id nulla. Integer purus tortor, iaculis vel nisi sit amet, commodo varius
diam. Etiam accumsan, diam sit amet posuere laoreet, felis mauris tincidunt
ante, vitae hendrerit nibh sapien in sem. Suspendisse rhoncus a nunc eget
ornare. Quisque posuere mauris nunc, ac convallis nisi condimentum vel. Integer
pulvinar neque at ligula interdum, ac blandit augue ultricies. Aliquam fermentum
enim in metus volutpat ullamcorper. Nam rutrum posuere mi, sit amet molestie
erat convallis eget. Vestibulum eu dictum lorem. Ut commodo cursus nisl, non
elementum quam ullamcorper eget.

Vivamus in tincidunt enim, quis tempus metus. Nam eu tristique nibh, sit amet
consectetur quam. Nulla nisi erat, tristique eget urna sit amet, viverra congue
metus. Cras eu nisl vel diam lobortis tempor non at tortor. Morbi maximus tellus
placerat elementum elementum. Praesent nec dignissim dolor. Sed eget purus
tortor. Nunc quis ullamcorper lacus.

Aenean quis lectus congue, vehicula leo porta, viverra velit. Suspendisse
potenti. Etiam fermentum tempus enim. Integer ac pharetra odio, eget facilisis
velit. Praesent auctor risus nec mauris imperdiet maximus. Mauris venenatis nisi
vitae aliquam fermentum. Ut volutpat bibendum tincidunt. Nam quis risus at orci
porttitor blandit ac a nulla. Ut ipsum quam, dapibus eu ligula id, molestie
accumsan elit. Nam rhoncus sem sit amet elit finibus, posuere porttitor augue
facilisis.

%%%
%%% Local Variables:
%%% mode: latex
%%% TeX-master: "../Tesis.tex"
%%% End:

%\include{...}
%\include{...}
%\include{...}
%\include{...}

% Apéndices
%\appendix
%%---------------------------------------------------------------------
%
%                          Parte 3
%
%---------------------------------------------------------------------
%
% Parte3.tex
% Copyright 2009 Marco Antonio Gomez-Martin, Pedro Pablo Gomez-Martin
%
% This file belongs to the TeXiS manual, a LaTeX template for writting
% Thesis and other documents. The complete last TeXiS package can
% be obtained from http://gaia.fdi.ucm.es/projects/texis/
%
% Although the TeXiS template itself is distributed under the
% conditions of the LaTeX Project Public License
% (http://www.latex-project.org/lppl.txt), the manual content
% uses the CC-BY-SA license that stays that you are free:
%
%    - to share & to copy, distribute and transmit the work
%    - to remix and to adapt the work
%
% under the following conditions:
%
%    - Attribution: you must attribute the work in the manner
%      specified by the author or licensor (but not in any way that
%      suggests that they endorse you or your use of the work).
%    - Share Alike: if you alter, transform, or build upon this
%      work, you may distribute the resulting work only under the
%      same, similar or a compatible license.
%
% The complete license is available in
% http://creativecommons.org/licenses/by-sa/3.0/legalcode
%
%---------------------------------------------------------------------


\partTitle{Apéndices}

\makepart

%%---------------------------------------------------------------------
%
%                          Apéndice 1
%
%---------------------------------------------------------------------


\chapter{Anexo 1: Reuniones del equipo}
\label{ap1:Reuniones}

\begin{FraseCelebre}
\begin{Frase}
...
\end{Frase}
\begin{Fuente}
...
\end{Fuente}
\end{FraseCelebre}

\begin{resumen}
...
\end{resumen}

\section{Reunión del 07 de Septiembre de 2017}

En la primera reunión del equipo se hicieron las presentaciones de los integrantes, y se discutieron las posibles ideas que se podrían implementar como proyecto en la \textit{hackathon}.
El tema principal sobre el que se discutía era el impacto social del proyecto, y que las métricas de la \textit{hackathon} así lo exigían.

Realizamos una tormenta de ideas en la que surgieron las siguientes:

\begin{itemize}
  \item \textbf{Plataforma de publicación de artículos académicos distribuida}: Implementar un sistema de publicación de artículos para compartir a través de la comunidad científica utilizando IPFS. Esta plataforma pretende eliminar los costes para el acceso a los artículos que imponen las empresas que se encargan de publicarlos y se benefician por ello. Implementar una plataforma totalmente descentralizada para compartir los artículos de divulgación científica conseguirá que el conocimiento de la investigación académica sea público y accesible por todos.

  \item \textbf{Wikipedia distribuida con modelos de gobernanza}: La idea de este proyecto inicialmente era descentralizar la plataforma de Wikipedia a través de IFPS y añadir algún modelo de gobernanza y de reputación para las revisiones de los artículos. EL problema es que es un proyecto muy complejo para implementarlo en sólo un mes, y haría falta un equipo bastante grande y la colaboración de la propia Wikipedia para llevarlo a cabo.

  \item \textbf{Aplicación de contactos para homosexuales en países donde son colectivos reprimidos}: En países como Rusia, los colectivos LGBT son reprimidos hasta el punto de que expresar su sexualidad puede ser un peligro para su seguridad personal. Esta idea trataba de poner en contacto de la manera más anónima y discreta posible a esas personas sin exponerse a los riesgos que ello conlleva.

  \item \textbf{ONG distribuida}: Esta plataforma pretendía ofrecer una bolsa de dinero en la que las personas iban realizando donaciones. Cada semana los donantes votaban dónde se iban a invertir el dinero mediante un sistema de votos.

  \item \textbf{Plataforma de intercambio de conocimientos de programación distribuida}: Stack Exchange es una de las web más importantes en la comunidad informática. Esta solución propone una altenativa totalmente distribuida mediante blockchain.

  \item \textbf{Plataforma de toma de decisiones distribuida}: La toma de decisiones en comunidades reprimidas es bastante dificil. Mediante una aplicación de toma de decisiones en blockchain (como la que tiene Loomio), se pueden ofrecer una herramienta para que estas personas en riesgo de exclusión se hagan oir.

  \item \textbf{Plataforma de crowdfunding para \textit{wistleblowers}}: El problema de las plataformas de crowdfunding es que una vez que se financia el proyecto, el usuario sólo puede ver el final del producto esperando que lo que ha financiado sea como promenten los desarrolladores. Esta plataforma propondría una alternativa con varios entregables en función del dinero que se vaya consiguiendo.


\end{itemize}

...

% Variable local para emacs, para  que encuentre el fichero maestro de
% compilación y funcionen mejor algunas teclas rápidas de AucTeX
%%%
%%% Local Variables:
%%% mode: latex
%%% TeX-master: "../Tesis.tex"
%%% End:

\section{Reunión del 08 de Septiembre de 2017}

Una vez que el equipo ha decidido el proyecto que vamos a afrontar, nos reunimos para ir decidiendo poco a poco las funcionalidades que habría de tener nuestra plataforma. Algunas de ellas son:

%\include{...}
%\include{...}
%\include{...}

\backmatter

%
% Bibliograf_a
%

%---------------------------------------------------------------------
%
%                      configBibliografia.tex
%
%---------------------------------------------------------------------
%
% bibliografia.tex
% Copyright 2009 Marco Antonio Gomez-Martin, Pedro Pablo Gomez-Martin
%
% This file belongs to the TeXiS manual, a LaTeX template for writting
% Thesis and other documents. The complete last TeXiS package can
% be obtained from http://gaia.fdi.ucm.es/projects/texis/
%
% Although the TeXiS template itself is distributed under the
% conditions of the LaTeX Project Public License
% (http://www.latex-project.org/lppl.txt), the manual content
% uses the CC-BY-SA license that stays that you are free:
%
%    - to share & to copy, distribute and transmit the work
%    - to remix and to adapt the work
%
% under the following conditions:
%
%    - Attribution: you must attribute the work in the manner
%      specified by the author or licensor (but not in any way that
%      suggests that they endorse you or your use of the work).
%    - Share Alike: if you alter, transform, or build upon this
%      work, you may distribute the resulting work only under the
%      same, similar or a compatible license.
%
% The complete license is available in
% http://creativecommons.org/licenses/by-sa/3.0/legalcode
%
%---------------------------------------------------------------------
%
% Fichero  que  configura  los  par_metros  de  la  generaci_n  de  la
% bibliograf_a.  Existen dos  par_metros configurables:  los ficheros
% .bib que se utilizan y la frase c_lebre que aparece justo antes de la
% primera referencia.
%
%---------------------------------------------------------------------


%%%%%%%%%%%%%%%%%%%%%%%%%%%%%%%%%%%%%%%%%%%%%%%%%%%%%%%%%%%%%%%%%%%%%%
% Definici_n de los ficheros .bib utilizados:
% \setBibFiles{<lista ficheros sin extension, separados por comas>}
% Nota:
% Es IMPORTANTE que los ficheros est_n en la misma l_nea que
% el comando \setBibFiles. Si se desea utilizar varias l_neas,
% terminarlas con una apertura de comentario.
%%%%%%%%%%%%%%%%%%%%%%%%%%%%%%%%%%%%%%%%%%%%%%%%%%%%%%%%%%%%%%%%%%%%%%
\setBibFiles{%
latex,otros,nuestros}

%%%%%%%%%%%%%%%%%%%%%%%%%%%%%%%%%%%%%%%%%%%%%%%%%%%%%%%%%%%%%%%%%%%%%%
% Definici_n de la frase c_lebre para el cap_tulo de la
% bibliograf_a. Dentro normalmente se querr_ hacer uso del entorno
% \begin{FraseCelebre}, que contendr_ a su vez otros dos entornos,
% un \begin{Frase} y un \begin{Fuente}.
%
% Nota:
% Si no se quiere cita, se puede eliminar su definici_n (en la
% macro setCitaBibliografia{} ).
%%%%%%%%%%%%%%%%%%%%%%%%%%%%%%%%%%%%%%%%%%%%%%%%%%%%%%%%%%%%%%%%%%%%%%
\setCitaBibliografia{
\begin{FraseCelebre}
  \begin{Frase}
    You act, and you know why you act, but you don't know why you know that you know what you do?
\end{Frase}
\begin{Fuente}
  The name of the rose - Umberto eco
\end{Fuente}
\end{FraseCelebre}
}

%%
%% Creamos la bibliografia
%%
\makeBib

% Variable local para emacs, para  que encuentre el fichero maestro de
% compilaci_n y funcionen mejor algunas teclas r_pidas de AucTeX

%%%
%%% Local Variables:
%%% mode: latex
%%% TeX-master: "../Tesis.tex"
%%% End:


%
% _ndice de palabras
%

% S_lo  la   generamos  si  est_   declarada  \generaindice.  Consulta
% TeXiS.sty para m_s informaci_n.

% En realidad, el soporte para la generaci_n de _ndices de palabras
% en TeXiS no est_ documentada en el manual, porque no ha sido usada
% "en producci_n". Por tanto, el fichero que genera el _ndice
% *no* se incluye aqu_ (est_ comentado). Consulta la documentaci_n
% en TeXiS_pream.tex para m_s informaci_n.
\ifx\generaindice\undefined
\else
%%---------------------------------------------------------------------
%
%                        TeXiS_indice.tex
%
%---------------------------------------------------------------------
%
% TeXiS_indice.tex
% Copyright 2009 Marco Antonio Gomez-Martin, Pedro Pablo Gomez-Martin
%
% This file belongs to TeXiS, a LaTeX template for writting
% Thesis and other documents. The complete last TeXiS package can
% be obtained from http://gaia.fdi.ucm.es/projects/texis/
%
% This work may be distributed and/or modified under the
% conditions of the LaTeX Project Public License, either version 1.3
% of this license or (at your option) any later version.
% The latest version of this license is in
%   http://www.latex-project.org/lppl.txt
% and version 1.3 or later is part of all distributions of LaTeX
% version 2005/12/01 or later.
%
% This work has the LPPL maintenance status `maintained'.
% 
% The Current Maintainers of this work are Marco Antonio Gomez-Martin
% and Pedro Pablo Gomez-Martin
%
%---------------------------------------------------------------------
%
% Contiene  los  comandos  para  generar  el �ndice  de  palabras  del
% documento.
%
%---------------------------------------------------------------------
%
% NOTA IMPORTANTE: el  soporte en TeXiS para el  �ndice de palabras es
% embrionario, y  de hecho  ni siquiera se  describe en el  manual. Se
% proporciona  una infraestructura  b�sica (sin  terminar)  para ello,
% pero  no ha  sido usada  "en producci�n".  De hecho,  a pesar  de la
% existencia de  este fichero, *no* se incluye  en Tesis.tex. Consulta
% la documentaci�n en TeXiS_pream.tex para m�s informaci�n.
%
%---------------------------------------------------------------------


% Si se  va a generar  la tabla de  contenidos (el �ndice  habitual) y
% tambi�n vamos a  generar el �ndice de palabras  (ambas decisiones se
% toman en  funci�n de  la definici�n  o no de  un par  de constantes,
% puedes consultar modo.tex para m�s informaci�n), entonces metemos en
% la tabla de contenidos una  entrada para marcar la p�gina donde est�
% el �ndice de palabras.

\ifx\generatoc\undefined
\else
   \addcontentsline{toc}{chapter}{\indexname}
\fi

% Generamos el �ndice
\printindex

% Variable local para emacs, para  que encuentre el fichero maestro de
% compilaci�n y funcionen mejor algunas teclas r�pidas de AucTeX

%%%
%%% Local Variables:
%%% mode: latex
%%% TeX-master: "./Tesis.tex"
%%% End:

\fi

%
% Lista de acr_nimos
%

% S_lo  lo  generamos  si  est_ declarada  \generaacronimos.  Consulta
% TeXiS.sty para m_s informaci_n.


\ifx\generaacronimos\undefined
\else
%---------------------------------------------------------------------
%
%                        TeXiS_acron.tex
%
%---------------------------------------------------------------------
%
% TeXiS_acron.tex
% Copyright 2009 Marco Antonio Gomez-Martin, Pedro Pablo Gomez-Martin
%
% This file belongs to TeXiS, a LaTeX template for writting
% Thesis and other documents. The complete last TeXiS package can
% be obtained from http://gaia.fdi.ucm.es/projects/texis/
%
% This work may be distributed and/or modified under the
% conditions of the LaTeX Project Public License, either version 1.3
% of this license or (at your option) any later version.
% The latest version of this license is in
%   http://www.latex-project.org/lppl.txt
% and version 1.3 or later is part of all distributions of LaTeX
% version 2005/12/01 or later.
%
% This work has the LPPL maintenance status `maintained'.
%
% The Current Maintainers of this work are Marco Antonio Gomez-Martin
% and Pedro Pablo Gomez-Martin
%
%---------------------------------------------------------------------
%
% Contiene  los  comandos  para  generar  el listado de acr_nimos
% documento.
%
%---------------------------------------------------------------------
%
% NOTA IMPORTANTE:  para que la  generaci_n de acr_nimos  funcione, al
% menos  debe  existir  un  acr_nimo   en  el  documento.  Si  no,  la
% compilaci_n  del   fichero  LaTeX  falla  con   un  error  "extra_o"
% (indicando  que  quiz_  falte  un \item).   Consulta  el  comentario
% referente al paquete glosstex en TeXiS_pream.tex.
%
%---------------------------------------------------------------------


% Redefinimos a espa_ol  el t_tulo de la lista  de acr_nimos (Babel no
% lo hace por nosotros esta vez)

\def\listacronymname{Lista de acr_nimos}

% Para el glosario:
% \def\glosarryname{Glosario}

% Si se  va a generar  la tabla de  contenidos (el _ndice  habitual) y
% tambi_n vamos a  generar la lista de acr_nimos  (ambas decisiones se
% toman en  funci_n de  la definici_n  o no de  un par  de constantes,
% puedes consultar config.tex  para m_s informaci_n), entonces metemos
% en la  tabla de contenidos una  entrada para marcar  la p_gina donde
% est_ el _ndice de palabras.

\ifx\generatoc\undefined
\else
   \addcontentsline{toc}{chapter}{\listacronymname}
\fi


% Generamos la lista de acr_nimos (en realidad el _ndice asociado a la
% lista "acr" de GlossTeX)

\printglosstex(acr)

% Variable local para emacs, para  que encuentre el fichero maestro de
% compilaci_n y funcionen mejor algunas teclas r_pidas de AucTeX

%%%
%%% Local Variables:
%%% mode: latex
%%% TeX-master: "../Tesis.tex"
%%% End:

\fi

%
% Final
%
%---------------------------------------------------------------------
%
%                      fin.tex
%
%---------------------------------------------------------------------
%
% fin.tex
% Copyright 2009 Marco Antonio Gomez-Martin, Pedro Pablo Gomez-Martin
%
% This file belongs to the TeXiS manual, a LaTeX template for writting
% Thesis and other documents. The complete last TeXiS package can
% be obtained from http://gaia.fdi.ucm.es/projects/texis/
%
% Although the TeXiS template itself is distributed under the
% conditions of the LaTeX Project Public License
% (http://www.latex-project.org/lppl.txt), the manual content
% uses the CC-BY-SA license that stays that you are free:
%
%    - to share & to copy, distribute and transmit the work
%    - to remix and to adapt the work
%
% under the following conditions:
%
%    - Attribution: you must attribute the work in the manner
%      specified by the author or licensor (but not in any way that
%      suggests that they endorse you or your use of the work).
%    - Share Alike: if you alter, transform, or build upon this
%      work, you may distribute the resulting work only under the
%      same, similar or a compatible license.
%
% The complete license is available in
% http://creativecommons.org/licenses/by-sa/3.0/legalcode
%
%---------------------------------------------------------------------
%
% Contiene la última página
%
%---------------------------------------------------------------------


% Ponemos el marcador en el PDF al nivel adecuado, dependiendo
% de su hubo partes en el documento o no (si las hay, queremos
% que aparezca "al mismo nivel" que las partes.
\ifpdf
\ifx\tienePartesTeXiS\undefined
   \pdfbookmark[0]{Fin}{fin}
\else
   \pdfbookmark[-1]{Fin}{fin}
\fi
\fi

\thispagestyle{empty}\mbox{}

\vspace*{4cm}

\small

\hfill \emph{--¿Qué te parece desto, Sancho? -- Dijo Don Quijote --}

\hfill \emph{Bien podrán los encantadores quitarme la ventura,}

\hfill \emph{pero el esfuerzo y el ánimo, será imposible.}

\hfill

\hfill \emph{Segunda parte del Ingenioso Caballero}

\hfill \emph{Don Quijote de la Mancha}

\hfill \emph{Miguel de Cervantes}

\vfill%space*{4cm}

\hfill \emph{--Buena está -- dijo Sancho --; fírmela vuestra merced.}

\hfill \emph{--No es menester firmarla -- dijo Don Quijote--,}

\hfill \emph{sino solamente poner mi rúbrica.}

\hfill

\hfill \emph{Primera parte del Ingenioso Caballero}

\hfill \emph{Don Quijote de la Mancha}

\hfill \emph{Miguel de Cervantes}


\newpage
\thispagestyle{empty}\mbox{}

\newpage

% Variable local para emacs, para  que encuentre el fichero maestro de
% compilación y funcionen mejor algunas teclas rápidas de AucTeX

%%%
%%% Local Variables:
%%% mode: latex
%%% TeX-master: "../Tesis.tex"
%%% End:


\end{document}
