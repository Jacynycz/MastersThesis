
\def\tituloPortadaVal{\titulo}
\newcommand{\tituloPortada}[1]{%
\def\tituloPortadaVal{#1}
}

% Autor
\def\autorPortadaVal{\autor}
\newcommand{\autorPortada}[1]{%
\def\autorPortadaVal{#1}
}

% Tutor
\def\directorPortadaVal{Tutor no definido. Usa
  \texttt{$\backslash$tutorPortada}}
\newcommand{\directorPortada}[1]{%
\def\directorPortadaVal{#1}
}

% Fecha de publicacion
\def\fechaPublicacionVal{}
\newcommand{\fechaPublicacion}[1]{%
\def\fechaPublicacionVal{#1}
}

% Imagen de la portada (escudo)
\def\imagenPortadaVal{Imagenes/Vectorial/Todo}
\newcommand{\imagenPortada}[1]{%
\def\imagenPortadaVal{#1}
}
\def\escalaImagenPortadaVal{1.0}
\newcommand{\escalaImagenPortada}[1]{%
\def\escalaImagenPortadaVal{#1}
}
% Tipo de documento (TESIS, MANUAL, ...)
\def\tipoDocumentoVal{TESIS DOCTORAL}
\newcommand{\tipoDocumento}[1]{%
\def\tipoDocumentoVal{#1}
}

% Instituci_n
\def\institucionVal{}
\newcommand{\institucion}[1]{%
\def\institucionVal{#1}
}

% Primer subt_tulo de la segunda portada
\def\textoPrimerSubtituloPortadaVal{%
\textit{Memoria que presenta para optar al master en ingeniería informática} \leavevmode \\[0.3em]%
\textbf{\autorPortadaVal}%
}
\newcommand{\textoPrimerSubtituloPortada}[1]{%
\def\textoPrimerSubtituloPortadaVal{#1}
}

% Segundo subt_tulo de la segunda portada
\def\textoSegundoSubtituloPortadaVal{%
\textit{Directed by} \leavevmode \\[0.3em]
\textbf{\directorPortadaVal}
}
\newcommand{\textoSegundoSubtituloPortada}[1]{%
\def\textoSegundoSubtituloPortadaVal{#1}
}

% ISBN
\newcommand{\isbn}[1]{%
\def\isbnVal{#1}
}

% Copyright
\newcommand{\copyrightInfo}[1]{%
\def\copyrightInfoVal{#1}
}

% Cr_ditos a TeXiS
\newcommand{\noTeXiSCredits}{%
\def\noTeXiSCreditsVal{}
}

% Explicaci_n sobre impresi_n a doble cara
\newcommand{\explicacionDobleCara}{%
\def\explicacionDobleCaraVal{}
}
%%%
% Configuraci_n terminada
%%%


%%%
%% COMANDO PARA CREAR LAS PORTADAS.
%% CONTIENE TODO EL C_DIGO LaTeX
%%%
\newcommand{\makeCover}{%


% Ponemos el marcador en el PDF
\ifpdf
\pdfbookmark{Página de Título}{titulo}
\fi

%
%  M_RGENES
%
% La maquetaci_n de las p_ginas en LaTeX es bastante complicada en lo
% que se refiere a los m_rgenes. Por ejemplo, _siempre_ hay un
% desplazamiento de una pulgada hacia la derecha y hacia abajo, porque
% por razones de maquetaci_n se cree que es necesario ese espacio. Si
% quieres (en una hoja impar, que son de las _nicas que me he
% preocupado) escribir por encima de la primera pulgada, o hacia la
% izquierda, tienes que usar "m_rgenes" negativos.
%
% Para poder seguir esto un poco, lo mejor es que ejecutes \layout con
% el paquete layout cargado, para ver una imagen...
%
% Los valores importantes en lo que se refiere al margen (horizontal,
% que es del _nico que me he preocupado) son:
%
%   - \hoffset : desplazamiento horizontal del "eje" de coordenadas. A
%   este valor hay que sumarle, irremediablemente, una pulgada. El
%   valor por defecto es 0 pt
%   - \oddsidemargin : "margen" izquierdo (en las p_ginas impares). El
%   texto principal de los p_rrafos comenzar_ en esa posici_n (es
%   decir, 1 pulgada + \hoffset + \oddsidemargin). El valor por
%   defecto es 22 pt
%   - \textwidth : longitud del texto (de los p_rrafos). El valor por
%   defecto son 360 pt.
%   - \marginparsep : separador de la parte derecha del texto
%   principal y el espacio a anotaciones en el margen (notas
%   marginales). El valor por defecto es 7 pt.
%   - \marginparwidth : ancho de la secci_n de notas marginales. El
%   valor por defecto es 106 pt.
%
% Fijate que las notas marginales NO necesariamente llegar_n hasta el
% final del folio. La separaci_n entre el extremo derecho de las notas
% al margen y el final del folio en realidad depender_ del tama_o de% _ste; no se especifica de ninguna manera. Esto significa que si
% quieres ajustar exactamente el margen derecho tienes que echar
% cuentas con respecto al tama_o del papel, que se mantiene en
% \paperwidth.
%
% En realidad, no s_ de qu_ manera pero todos los valores est_n
% relacionados de alg_n modo, y un cambio en uno afecta a los dem_s de
% maneras bastante inveros_miles. Adem_s, _s_lo_ pueden cambiarse en
% el pre_mbulo (bueno... al menos \textwidth, \oddsidemargin se puede
% cambiar en otros sitios, pero no se pueden hacer muchas cosas s_lo
% con aquellos que se pueden cambiar...).
%
% En general se recomienda que NO cambies los m_rgenes. Est_n elegidos
% por especialistas que saben de maquetaci_n y que han estudiado en
% profundidad las mejores organizaciones. Por ejemplo, los cambios que
% hagas pueden alargar demasiado las l_neas, o dejarlas demasiado
% cortas. Pero bueno, si aun as_ quieres cambiar los m_rgenes _a nivel
% global_ es preferible que uses el paquete geometry, cuya inclusi_n
% recibe los cent_metros de m_rgen que quieres en cada lado, y _l se
% encarga de hacer las cuentas para que queden as_, porque tocar tanto
% valor es un infierno.
%
% Si quieres cambiar los m_rgenes moment_neamente, entonces lo mejor
% es hacer uso de un entorno tipo "lista" que permite tocar algunos
% contadores para ajustar las posiciones de los p_rrafos. Eso es
% precisamente lo que hace el entorno cambiamargen definido en
% TeXiS.sty
%
% El problema es que esos cambios son _relativos_ a los m_rgenes
% oficiales. Si quieres hacer un cambio dr_stico (como el que
% necesitamos en la portada, para que quede centrada), entonces es
% necesario echar cuentas con los contadores anteriores para realizar
% el desplazamiento adecuado en cada lado para conseguirlo.
%
% Para resumir, los valores usados en los m_rgenes eran:
% 1 pulgada + hoffset + oddsidemargin +
%          + textwidth +
% + marginparsep + marginparwidth + AJUSTE
%                                          = paperwidth
%
% Lo que necesitamos en crear un entorno cambiamargen pasando los
% valores adecuados para que quede centrado. Para eso, hay que hacer
% cuentas, y eso es un tanto infierno en LaTeX a no ser que se incluya
% el paquete calc que permite notaci_n infija de operadores. Por
% tanto, para que esto funcione hay que incluirlo.
%
% Para aclararnos, vamos a crear un par de longitudes para hacer las
% cuentas poco a poco. Adem_s, lo primero es asumir que queremos
% _eliminar todos los m_rgenes_ y que los p_rrafos lleguen totalmente
% de lado a lado. Para eso, en la izquierda tendremos que restar la
% suma de los tres primeros valores (1 pulgada, \hoffset y
% \oddsidemargin).

\newlength{\cambioIzquierdo}
\setlength{\cambioIzquierdo}{1in + \hoffset + \oddsidemargin}
% Si te falla aqu_, incluye el paquete calc en el pre_mbulo.
% 1in = 1 pulgada = 72.27 pt

% En la parte derecha hay que restar el "margen" visible total, que es
% el tama_o de la hoja restando el espacio hasta la izquierda del
% p_rrafo y su ancho. Aprovechamos que el primer espacio lo tenemos en
% \cambioIzquierdo.
\newlength{\cambioDerecho}
\setlength{\cambioDerecho}{\cambioIzquierdo + \textwidth}
\setlength{\cambioDerecho}{\paperwidth - \cambioDerecho}

% Ya casi est_. Si hicieramos
%
% \begin{cambiamargen}{-\cambioIzquierdo}{-\cambioDerecho}
%    ...
% \end{cambiamargen}
%
% tendriamos p_rrafos que van de extremo a extremo de la hoja. Como
% eso es una exageraci_n, restamos a cada longitud el m_rgen real que
% queremos dejar.
\newlength{\margenPortada}
\setlength{\margenPortada}{3.5cm}

\setlength{\cambioIzquierdo}{\cambioIzquierdo - \margenPortada}
\setlength{\cambioDerecho}{\cambioDerecho - \margenPortada}


%%%
% Portada
%%%


% P_gina sin cabeceras
\thispagestyle{empty}

\begin{cambiamargen}{-\cambioIzquierdo}{-\cambioDerecho}


% En la primera hoja no se entiende de p_ginas pares e impares
\newlength{\evensidemarginOriginal}
\setlength{\evensidemarginOriginal}{\evensidemargin}

\newlength{\oddsidemarginOriginal}
\setlength{\oddsidemarginOriginal}{\oddsidemargin}

\setlength{\evensidemargin}{0cm}
\setlength{\oddsidemargin}{0cm}

% Comienza el tex...
\begin{huge}
\mbox{ }
\vfill %

% Cuidado: Ajustarlo si el t_tulo cambia...
\newlength{\longTitulo}
\settowidth{\longTitulo}{%
\textbf{Arquitectura y metodología para el}}
\begin{center}
\rule{\longTitulo}{.5mm}\leavevmode \\\relax
\vskip 1cm
\textbf{\tituloPortadaVal}
\vskip 0.7cm
\rule{\longTitulo}{.5mm}\leavevmode \\\relax
\end{center}
\end{huge}

\vfill %

\begin{center}
  \includegraphics[scale=\escalaImagenPortadaVal]{\imagenPortadaVal}
\end{center}

\vfill %

\begin{center}
  {\Large \textbf{\tipoDocumentoVal}}
\end{center}

\vfill

\begin{large}

\begin{center}
  \textbf{\autorPortadaVal}%\leavevmode \\[0.3cm]


\leavevmode \\\relax \mbox{ }% \leavevmode \\% \mbox{ } \leavevmode \\
\textoSegundoSubtituloPortadaVal \leavevmode \\[0.3]
\leavevmode \\\relax \mbox{ }% \leavevmode \\ \mbox{ } \leavevmode \\

\textbf{\institucionVal}\leavevmode \\[1em]
\textbf{\fechaPublicacionVal}

\end{center}

\end{large}

\vfill %

\end{cambiamargen}

\newpage

% P_gina que en la cara de detr_s del folio de la portada.
% Ponemos que el documento esta maquetado con TeXiS y la
% aclaraci_n de que debe imprimirse a doble cara.
\thispagestyle{empty}
\mbox{ }
\vfill%space*{4cm}
\begin{small}
\begin{center}
\ifx\noTeXiSCreditsVal\undefined
  Document made with \texis\ v.\texisVer. modified by Viktor Jacycynycz García
\else
\mbox{ }
\fi
\end{center}
\end{small}
\vspace*{2cm}
\begin{small}
\begin{center}
\ifx\explicacionDobleCaraVal\undefined
\mbox{ }
\else
\noindent This document is prepared for duble-side printing.
\fi
\end{center}
\end{small}

%%%
% Segunda portada
%%%

\newpage

\thispagestyle{empty}

\mbox{ }

\begin{Huge}
\begin{center}
\tituloPortadaVal
\end{center}
\end{Huge}

\vfill %

\begin{large}
  \begin{center}
    \textoPrimerSubtituloPortadaVal\
\leavevmode \\\relax \mbox{ } \leavevmode \\\relax \mbox{ } \leavevmode \\\relax
\textoSegundoSubtituloPortadaVal \leavevmode \\[0.3em]
\end{center}
\end{large}

\vfill %

\begin{large}
\begin{center}
\textbf{\institucionVal}\leavevmode \\[0.2em]
    \mbox{ }  \leavevmode \\\relax
\textbf{\fechaPublicacionVal}
\end{center}
\end{large}


\newpage
\thispagestyle{empty}
\mbox{ }

% Informacion del ISBN y copyright
\vskip 19cm
Copyright \textcopyright Viktor Jacynycz García

\newpage
\thispagestyle{empty}
\mbox{ }
\vskip 5cm
El/la abajo firmante, matriculado/a en el Máster en Ingeniería Informática de la
Facultad de Informática, autoriza a la Universidad Complutense de Madrid
(UCM) a difundir y utilizar con fines académicos, no comerciales y
mencionando expresamente a su autor el presente Trabajo Fin de Máster:
“TÍTULO”, realizado durante el curso académico 2017-2018 bajo la dirección
de Samer Hassan Collado y Antonio Sanchez Ruiz-Granados en el
Departamento de ISIA, y a la Biblioteca de la UCM a depositarlo en el
Archivo Institucional E-Prints Complutense con el objeto de incrementar la
difusión, uso e impacto del trabajo en internet y garantizar su preservación y
acceso a largo plazo.



}% \newcommand{\makeCover}

% Variable local para emacs, para que encuentre el fichero
% maestro de compilaci_n
%%%
%%% Local Variables:
%%% mode: latex
%%% TeX-master: "../Tesis.tex"
%%% End:
