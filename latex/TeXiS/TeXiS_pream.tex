
\usepackage{verbatim}


%\usepackage[spanish]{layout}


\ifx\generaacronimos\undefined
  % No queremos acr_nimos. Como no incluiremos el paquete glosstex
  % redefinimos sus comandos para que no cruja si se han usado.
  \newcommand{\glosstex}[1]{}
  \newcommand{\acronym}[1]{}
  \newcommand{\ac}[1]{#1}
  \newcommand{\acs}[1]{#1}
  \newcommand{\acl}[1]{#1}
  \newcommand{\acf}[1]{#1}
\else
  \usepackage{glosstex}
\fi


\usepackage{shortvrb}

\usepackage[utf8]{inputenc}


\usepackage[round]{natbib}

% Activa el  uso de  espa_ol. Babel es  un paquete bastante  gordo que
% afecta a  otros paquetes de  LaTeX que antes ten_an  cableadas cosas
% para el ingl_s... por ejemplo, hace  que la '_' vaya despu_s de la N
% en el orden alfab_tico, y cosas varias sobre espa_ol.
% En alg_n sitio le_ que "activeacute" nos permite poner _ en lugar de
% tener que escribir  \'a ... aunque eso en teor_a  lo hace el paquete
% inputenc :-m

\usepackage[english]{babel}

% El paquete url sirve para poder  poner direcciones Web y no morir en
% el intento,  sobre todo  en la bibliograf_a.   Es _til porque  si la
% direcci_n  no entra  en una  l_nea,  la divide  de forma  coherente.
% Adem_s en  la bibliograf_a tambi_n  cambia el formato de  la fuente.
% Se utiliza poniendo la  direcci_n con \url{http://...}  Tambi_n vale
% para direcciones de correo, etc.
% No conviene  quitarlo, porque con  casi toda seguridad, al  menos la
% bibliograf_a utilizar_ el comando.

\usepackage{url}

% El documento est_  pensado para ser compilado o  bien con pdflatex o
% bien con  latex. Algunas  cosas s_lo se  soportan si se  compila con
% pdflatex (por  ejemplo, a_adir en  el "_ndice del documento"  PDF un
% t_tulo para una secci_n para luego verlo cuando se ve en el visor de
% PDF el  _ndice). Para  meter c_modamente las  l_neas que  s_lo deben
% "ejecutarse" si  se compila el  documento con pdflatex  definimos un
% nuevo "tipo de if".
% En  realidad,  en vez  de  utilizar  el  paquete, se  puede  definir
% directamente, utilizando:
% \newif\ifpdf\ifx\pdfoutput\undefined\pdffalse\else\pdfoutput=1\pdftrue\fi
%
% La idea es luego poder tener:
% \ifpdf ...
% \else
% ...
% \fi

\usepackage{ifpdf}

% Juego (solo make latex) :-p
%\usepackage[mirror,invert]{crop}

% Si estamos usando  pdflatex, queremos pedir que se  creen enlaces en
% las  referencias  a figuras,  cap_tulos  y  bibliograf_a para  poder
% "navegar" al visualizar el PDF. Tambi_n queremos que esos enlaces NO
% aparezcan rodeados de un marco que queda horrible.
%
% La opci_n "pdfpagelabels"  sirve para que en el  cuadrito del n_mero
% de p_gina del Adobe Acrobat  salga el n_mero de "p_gina l_gico", por
% ejemplo  "v (5  de  180)"  (con n_mero  romano  porque al  principio
% empieza a numerarse as_).

\ifpdf
   \RequirePackage[pdfborder=0,colorlinks,hyperindex,pdfpagelabels]{hyperref}
   \def\pdfBorderAttrs{/Border [0 0 0] } % Ning_n borde en los enlaces
\fi

% Cuando se  genera el  _ndice, se necesita  el paquete  hyperref, que
% permite referencias  (o enlaces) en el propio  documento. El paquete
% aparece como \RequirePackage cuando se est_ usando pdflatex, gracias
% al comando anterior. Si no se est_ en pdf, se incluye directamente
\ifpdf
\else
\usepackage{hyperref}
\fi

\hypersetup{colorlinks=false}


% Durante  la infancia  de TeXiS  (es decir,  cuando est_bamos  en las
% primeras etapas de la escritura de nuestras tesis), ten_amos la idea
% de tener _ndice de palabras, es decir una lista de palabras usadas a
% lo   largo  del   documento,  junto   con  las   p_ginas   donde  se
% mencionaban. Esto aparece en algunos libros, permitiendo la b_squeda
% r_pida de los lugares donde se habla de ciertas cosas.
%
% El problema  de hacer  dicho _ndice no  es s_lo "tecnol_gico"  en lo
% referente al  uso de  LaTeX, sino  tambi_n a la  hora de  marcar qu_
% palabras (y en qu_ lugares) se quieren a_adir a dicho _ndice.
%
% En nuestro  caso, despu_s de haber superado  la barrera tecnol_gica,
% decidimos que no  merec_a la pena el esfuerzo  de recorrer las tesis
% tras  ser  escritas  para  anotar  las palabras  importantes  y  que
% aparecieran en el _ndice. Debido  a eso, ni siquiera el soporte est_
% terminado  (a  nivel, por  ejemplo,  del  Makefile).  Por lo  tanto,
% podemos  decir que  TeXiS  tiene un  soporte  embrionario para  esas
% listas de palabras,  pero que necesitar_a algo m_s  de esfuerzo para
% hacerlo amigable.  Para no crear  falsas expectativas, el  manual de
% TeXiS, por  tanto, no menciona  nada de este asunto,  aunque dejamos
% aqu_ esa configuraci_n original del soporte b_sico.
%
% En realidad, para a_adir un _ndice de palabras bastar_a con:
%
%    - Modificar la parte final de Tesis.tex, e incluir siempre
%      TeXiS/TeXiS_indice (ahora su inclusi_n est_ comentada).
%    - Modificar el Makefile y descomentar las lineas
%      -makeindex $(NOMBRE_LATEX)
%      pare que se genere el _ndice de palabras.
%
% No se ha  hecho "oficial" este soporte (ni se  explica en el manual)
% porque no se ha probado en "producci_n". No obstante, a continuaci_n
% se describe su uso, tanto a nivel de configuraci_n (mucho de lo cual
% TeXiS ya hace) como a nivel del usuario:


% Permite crear  _ndices (de palabras)  en LaTeX (es decir,  el _ndice
% que suele  haber al final de  los libros diciendo en  qu_ p_ginas se
% habla de qu_ t_rminos). La  generaci_n no es una cosa trivial...  Lo
% primero que hay que hacer  es poner '\makeindex' en el pre_mbulo del
% documento (antes del  \begin{document}), luego poner \index{entrada}
% a lo largo  de todo el documento para  indicar qu_ palabras queremos
% que  aparezcan en  el _ndice  tem_tico.  Por _ltimo,  hay que  poner
% \printindex en  el lugar donde queramos  que se genere  y muestre el
% _ndice.

% Pero  eso no  es todo...  en realidad  la generaci_n  del  _ndice se
% realiza con  una herramienta externa a LaTeX  llamada makeindex (que
% funciona  con  la  misma  "filosof_a"  que bibtex,  como  una  etapa
% posterior).  Total, que  hay que ejecutar LaTeX una  primera vez, lo
% que  genera un  fichero  auxiliar  con extensi_n  .idx  con toda  la
% informaci_n sobre las  palabras que deben salir en  el _ndice. Luego
% se ejecuta  MakeIndex, que recoge  ese fichero y utiliza  un fichero
% con extensi_n  .ist con informaci_n  sobre estilo para  generar otro
% fichero,   esta   vez   con   extensi_n  .ind   (que   supongo   que
% conceptualmente es  semejante al "bbl"  de BibTeX).  Por  _ltimo, se
% vuelve a  ejecutar LaTeX que  recompila el fichero y,  utilizando el
% .ind, mete por f_n el _ndice bueno.

% Aconsejan NO  poner los \index  seg_n vas escribiendo  el documento,
% porque el _ndice que te queda seguramente sea un poco malo. En lugar
% de eso  dicen que utilices  "delatex" para sacar todas  las palabras
% que utilizas en tu  documento, luego las "limpies" decidiendo cuales
% quieres que aparezcan, y luego  metas todos los \index. En cualquier
% caso,  el  programa "delatex"  no  se  instala  con la  distribuci_n
% habitual de latex.

% Puedes anular la generaci_n del _ndice si quitas el \makeindex en el
% pre_mbulo (supongo que no se generar_ el .idx).

% http://www.mpi.nl/world/persons/private/robstu/latex/how_to/indexes.html
% http://ipagwww.med.yale.edu/latex/makeindex.pdf

\usepackage{makeidx}


% Paquete de  depuraci_n. Se si utiliza,  en el margen  de cada p_gina
% aparecen  las palabras  que se  a_adir_n  al _ndice  de esa  p_gina.
% Pero... parece que no funciona con el pdflatex de Linux :'( En lugar
% de sacar las palabras en la esquina de la p_gina la saca en el sitio
% donde est_n insertadas, en mitad del p_rrafo... y claro, es un tanto
% ca_tico :'(
% Por  eso, aunque  estemos en  modo "depuraci_n",  si  estamos usando
% pdflatex no se incluye.

\ifpdf
\else
   \ifx\release\undefined
      \usepackage{showidx}
   \fi
\fi

%\citeindextrue Activa la aparici_n de las referencias en el _ndice
%\citeindexfalse La desactiva


% Para pedir que se genere  la informaci_n para el _ndice de palabras.
% Para  no  ralentizar la  compilaci_n,  s_lo  se  genera cuando  est_
% definido el s_mbolo correspondiente.
\ifx\generaindice\undefined
\else
\makeindex
\fi

% Pruebas con _ndices
%\usepackage{index}
%\newindex{default}{idx}{ind}{_ndice}
%\newindex{cite}{cdx}{cnd}{_ndice de citas}
%\renewcommand{\citeindextype}{cite}
%\citeindextrue


% Otro paquete  de depuraci_n, para mostrar las  etiquetas del \label,
% etc. Se  puede desactivar  el mostrado de  algunas de ellas  con las
% opciones [notref]  y [notcite].  Lo tengo  desactivado, porque queda
% demasiado  cargado.   Adem_s,  provoca  errores de  compilaci_n  con
% algunas figuras  convertidas desde eps  a pdf cuando se  compila con
% pdflatex.

\ifx\release\undefined
%\usepackage{showkeys}
\fi


% Para poder utilizar el entorno multicol, _til en los sitios donde se
% quieren poner dos figuras una al lado de la otra.
\usepackage{multicol}

%
% Para  poder poner  "subfiguras". El  paquete permite  incluir varias
% figuras dentro  de un entorno  flotante (t_picamente una  figura), y
% puede  etiquetar  cada  una  con  una letra,  y  asociar  diferentes
% descripciones a cada una.
%
% La idea  es crear  un entorno figura  tradicional, pero no  poner en
% ella  directamente  el  \includegraphics  (o  cualquier  otra  cosa,
% vamos), sino subdividir ese entorno  figura en varias partes. A cada
% una  de   ellas  se  le   da  una  etiqueta  diferente   para  poder
% referenciarlas, una descripci_n, etc_tera.  Un ejemplo:
%
% \begin{figure}[t]
%   \centering
%   %
%   \subfloat[<ParaElIndice1>][<Caption1>]{
%      % Contenido para este "subelemento" (podr_ ser una
%      % figura, una tabla, o cualquier otra cosa).
%      \includegraphics[width=5cm]{ficheroSinExtension}
%      \label{fig:etiqueta1}
%   }
%   \subfloat[<ParaElIndice2>][<Caption2>]{
%      % Contenido para este "subelemento" (podr_ ser una
%      % figura, una tabla, o cualquier otra cosa).
%      \includegraphics[width=5cm]{ficheroSinExtension}
%      \label{fig:etiqueta2}
%   }
%  \caption{Descripci_n global para la figura}
%  \label{Etiqueta para toda la figura}
% \end{figure}
%
% El sistema  autom_ticamente decide cuando poner  la siguiente figura
% al lado, o  en otra linea. Es  posible forzar a que se  ponga en una
% linea nueva si se deja una linea en blanco en el .tex. Esto tiene la
% repercusi_n  de que  no deber_as  dejar lineas  en blanco  en ning_n
% momento dentro  del entorno flotante,  para evitar que se  "salte de
% linea"  en las  figuras.  Si quieres  por  legilibidad dejar  alguna
% linea, pon un comentario vacio.
%
% La separaci_n  entre dos figuras/tablas  que se colocan en  la misma
% fila puede ser demasiado peque_a. Para seprarlas un poco m_s, puedes
% poner \qquad entre el cierre llaves de un \subfloat y el siguiente:
%
% [...]
%    \subfloat[..][..]{ ... }
%    \qquad
%    \subfloat[..][..]{ ... }
% [...]
%
% Por otro lado,  el \subfloat tiene dos "par_metros",  que se colocan
% entre   corchetes    justo   despu_s.   En    realidad   ambos   son
% opcionales. Podr_amos poner directamente:
%
% \subfloat{ <comandos para el subelemento> }
%
% pero en ese caso no se etiquetar_ con una letra.
%
% El texto que se pone entre los primeros corchetes se utiliza para el
% _ndice de figuras. En teor_a, se mostrar_ en dicho _ndice primero la
% descripci_n global de la figura,  y luego la de cada subelemento. Si
% no  quieres que  ocurra,  deja  en blanco  el  contenido del  primer
% corchete. En la  pr_ctica... yo he puesto una cadena  y no ha salido
% en el _ndice.
% El  segundo  corchete  recibe   el  texto  con  la  descripci_n  del
% subelemento,   es  decir  lo   que  aparecer_   junto  a   la  letra
% identificativa.  Si lo  dejas vac_o  (pero poniendo  los corchetes),
% saldr_ la letra, sin texto.  Si ni siquieras pones los corchetes, no
% saldr_ tampoco la letra.
%
% Para referenciar  uno de los subelementos  basta con \ref{etiqueta},
% siendo  la   etiqueta  una  definida  mediante   \label  DENTRO  del
% \subfloat. La  forma de salir  ser_, por ejemplo, 2.3a  Si prefieres
% que salga  2.3 (a) (o quieres poner  "en la imagen (a)  de la figura
% 2.3...")  puedes  utilizar  \subref{etiquitaDeSubelemento}.  \subref
% escribir_ el (a).  Combin_ndolo con la etiqueta global  de la figura
% (del  entorno flotante contenedor,  vamos) que  mostrar_ el  2.3 del
% ejemplo, puedes hacer cosas como la anterior con facilidad.
%
% Se pueden hacer muchas m_s pijadas, pero esto deber_a ser suficiente
% la mayor parte de las veces. Mira el manual oficial para m_s
% informaci_n (ftp://tug.ctan.org/pub/tex-archive/macros/latex/contrib/subfig/subfig.pdf)
%
% Por cierto,  este paquete es la  evoluci_n de subfigure,  pero no es
% compatible con _l (por eso decidieron cambiarle el nombre).
\usepackage{subfig}

% El  paquete anterior  tiene un  problema  cuando se  utiliza en  los
% subfloat entornos verbatim (eso incluye a los lstlistings), debido a
% que  no  se  pueden  anidar  entornos verbatim  (bueno,  esa  es  la
% explicaci_n que  he encontrado,  aunque no ten_a  ni idea de  que se
% estuvieran anidando entornos verbatim :-) ).

% Para  arreglarlo, en la  documentaci_n de  subfig dan  una soluci_n,
% creando un  nuevo entorno,  SubFloat, que est_  en una caja  LaTeX o
% algo  as_, y que  consigue poder  meterlos. El  entorno se  define a
% continuaci_n  (cortar  y  pegar  de  la  documentaci_n  del  paquete
% subfig).
%
% Los dos par_metros que ten_amos  antes para el \subfloat (caption, y
% caption para el _ndice) ahora se pasan de forma distinta:
%
% \begin{SubFloat}[texto indice]{Texto figura\label{<etiqueta>}
%    <contenido>
% \end{SubFloat}
%
% Es decir, hay que poner la etiqueta directamente en el par_metro del
% entorno.
%
% Para que el SubFloat funcione,  hay que hacer algo m_s, no obstante,
% ya que el entorno verbatim  (y lstlisting) NO definen en su creaci_n
% lo que ocupan, y por lo tanto el \subfloat no es capaz de determinar
% c_mo colocar las  subfiguras, ya que no conoce  su tama_o. Para eso,
% el entorno verbatim (o lstlisting)  hay que enmarcarlo dentro de una
% minipage en la que s_ definimos (a mano) el tama_o. Un ejemplo puede
% ser:
%
% \begin{SubFloat}{\label{fig:ejercicios1a3:2}Ejercicio 2}%
% \begin{minipage}{0.4\linewidth}%
% \begin{lstlisting}[frame=tb]
% public void f(int b) {
%     int a;
%     a = 3 + b;
% }
% \end{lstlisting}%
% \end{minipage}
% \end{SubFloat}
%
% De esta  forma, se  hace que  el c_digo en  el lstlisting  ocupe dos
% quintas  partes del ancho  de la  p_gina.  Se  podr_a pensar  que es
% posible  utilizar  el  "par_metro"  linewidth  del  lstlisting,  que
% permite     tambi_n     indicar      el     ancho     del     c_digo
% (\begin{lstlisting}[frame=tb,linediwth=0.4\linewidth]; sin embargo,
% eso  no es  posible ya  que ese  ancho es  utilizado por  el entorno
% lstlisting para  dibujar el marco  y, si tiene activado  el "romper"
% las l_neas  autom_ticamente, para romperlas; si todas  las l_neas de
% c_digo  son m_s  cortas que  ese  espacio, el  espacio efectivo  del
% c_digo ser_ el ancho de la l_nea m_s ancha.
%
% Otra desagradable sorpresa surge  con los subfloat: existe un margen
% a los  lados del _rea reservada  a la subfigura, para  que no queden
% todas  pegadas.   Esto  est_  bien,  porque as_  no  tienes  t_  que
% encargarte de  definir el espacio; simplemente  decides cu_nto ocupa
% de ancho cada  subfigura, y _l se encarga de  colocarlo todo. Lo que
% ocurre es que parece no tener  en cuenta que, cuando a la derecha de
% un subfloat NO hay otro (porque es el _ltimo de la "l_nea"), _sigue_
% poniendo  el espacio  en blanco  para  separar con  el siguiente  (e
% inexistente) subfloat, lo  que hace que, en caso  de estar centrando
% los elementos, no quede perfectamente centrado :(

\makeatletter
\newbox\sf@box
\newenvironment{SubFloat}[2][]%
{\def\sf@one{#1}%
\def\sf@two{#2}%
\setbox\sf@box\hbox
\bgroup}%
{ \egroup
\ifx\@empty\sf@two\@empty\relax
\def\sf@two{\@empty}
\fi
\ifx\@empty\sf@one\@empty\relax
\subfloat[\sf@two]{\box\sf@box}%
\else
\subfloat[\sf@one][\sf@two]{\box\sf@box}%
\fi}
\makeatother


% Para poder  incluir c_digo  fuente resaltado f_cilmente.   El tama_o
% b_sico de la letra, peque_o.
\usepackage{listings}
\lstset{basicstyle=\small}
\lstset{showstringspaces=false}
\lstset{tabsize=3}

% Para poder utilizar el entorno tabularx, que sirve para hacer tablas
% con  p_rrafos, sin tener  que indicar  el ancho  de cada  columna de
% p_rrafo particular.
\usepackage{tabularx}

% Definimos  los  nombres para  las  tablas  y  el _ndice  de  tablas,
% sobreescribiendo los de babel, que los llama 'cuadros'.
\addto\captionsspanish{%
  \def\tablename{Table}%
  \def\listtablename{Index of tables}%
  \def\listfigurename{Index of figures}%
  \def\contentsname{Index}%
  \def\chaptername{Chapter}%
}

% Supuestamente,  codificaci_n   que  tiene  'nativas'   las  palabras
% acentuadas,  para poder  tener guionado  de palabras  con acentos...
% aunque no estoy yo demasiado seguro...
\usepackage[T1]{fontenc}

% Paquete para poder utilizar figuras EPS
\usepackage{epsfig}
%\usepackage[dvips]{graphicx}
%http://www.cidse.itcr.ac.cr/revistamate/HERRAmInternet/Latex/wmlatexrevista/node19.html

% Hacemos que los  p_rrafos se separen con algo m_s  de espacio que lo
% habitual.
\setlength{\parskip}{0.2ex}

% El paquete  psboxit permite poner  una imagen Post Script  dentro de
% una  caja de  TeX. La  imagen se  parametriza con  la posici_n  y el
% tama_o de la caja TeX.

%\usepackage{psboxit}

% Permite hacer c_lculos sencillos dentro  de un documento de LaTeX Lo
% necesitamos en  la portada, para  poder ajustar los m_rgenes  con un
% poco de independencia  del tama_o de la p_gina.  Consulta el fichero
% portada.tex para m_s informaci_n.
\usepackage{calc}

% Define el entorno  longtable que sirve para crear  tablas que ocupan
% m_s de una p_gina.

%\usepackage{longtable}

% Cuando en una tabla tenemos lineas dobles de separaci_n entre filas
% y columnas, LaTeX las pinta de manera que d_ la sensaci_n de que
% cada celda es un rect_ngulo:
%          | |
%  [celda] | | [celda]
%          | |
% ---------+ +---------
%                       <- hueco entre lineas
% ---------+ +---------
%          | |
%  [celda] | | [celda]
%          | |
%
% Si queremos que las l_neas se corten:
%          | |
%  [celda] | | [celda]
%          | |
% ---------+-+---------
%          | |          <- hueco entre lineas
% ---------+-+---------
%          | |
%  [celda] | | [celda]
%          | |
%
% podemos utilizar el paquete hhline, que permite concretar en gran
% medida c_mo queremos que queden los cruces
% (http://www.cs.wright.edu/~jslater/hhline.pdf)
%
% Otro uso interesante de hhline es como sustitutivo de \cline. \cline
% (nativo de LaTeX) permite poner lineas de separaci_n entre filas que
% no englobe  a todas las  columnas. El problema  es que \cline  no se
% lleva bien con celdas con  fondo de color, porque queda _debajo_ del
% relleno y no se ve. \hhline sin embargo queda por encima, por lo que
% puede utilizarse para pintar lineas sencillas sin preocuparnos de la
% interacci_n  con las l_neas  horizontales (quiz_  tambi_n sencillas)
% pero aprovechar  que s_ van a  verse.  En concreto, en  una tabla de
% cuatro columnas  \cline{1-3} es  similar a \hhline{----~}  (el gui_n
% indica  que  queremos  l_nea para  esa  columna,  y  el ~  que  no).
% \usepackage{hhline}


% \usepackage[verbose]{geometry}
% \geometry{a4paper}
% \geometry{twosideshift=0pt}
% \geometry{top=4cm,bottom=4cm,left=3cm,right=3cm,headsep=1cm}

%
% Paquete para  poder ampliar las  opciones de las tablas,  para poder
% tener p_rrafos  dentro de una celda  y que nos los  ajuste el propio
% LaTeX.
%
%\usepackage{tabulary}
% Otras opciones anteriores
%\usepackage{tabularx}  % Este est_ inclu_do por arriba
%\usepackage{array}

%\usepackage{slashbox}
% Para poder poner l_neas diagonales en una tabla:
%
% \ b|
%  \ |  ...
% a \|
% ---+----------
%    |
%    |
%
% En la casilla a dividir, \backslashbox{a}{b} & ...
% Vamos... no queda muy bien... en alg_n sitio vi que para mejorar un
% poco el aspecto se pod_a inclu_r pict2e tambi_n, aunque en cualquier
% caso el resultado pod_a seguir siendo discutible.

%
% Paquete para poder poner celdas de colores en una tabla
%
%\usepackage{colortbl}
%\usepackage{color}


% Paquete  que permite  cambiar  el formato  en  el que  se ponen  los
% t_tulos   de   cap_tulos,  secciones,   etc.    Tambi_n  define   el
% comando \chaptertitlename, que  equivale a \chaptername ("Cap_tulo")
% cuando se est_  dentro de un cap_tulo o  a \apendixname ("Ap_ndice")
% cuando se  est_ en un ap_ndice.  Eso se utiliza en  la definici_n de
% las cabeceras.
\usepackage{titlesec}


% Cambio del t_tulo de los cap_tulos (sacado de
% http://aristarco.dnsalias.org/book/export/html/13)

% \newcommand{\bigrule}{\titlerule[0.5mm]}

% \titleformat{\chapter}[display] % cambiamos el formato de los cap_tulos
% {\bfseries\Huge} % por defecto se usar_n caracteres de tama_o \Huge en negrita
% {% contenido de la etiqueta
%  \titlerule % l_nea horizontal
%  \filleft % texto alineado a la derecha
%  \Large\chaptertitlename\ % "Cap_tulo" o "Ap_ndice" en tama_o \Large en lugar de \Huge
%  \Large\thechapter} % n_mero de cap_tulo en tama_o \Large
% {0mm} % espacio m_nimo entre etiqueta y cuerpo
% {\filleft} % texto del cuerpo alineado a la derecha
% [\vspace{0.5mm} \bigrule \vspace{1cm}] % despu_s del cuerpo, dejar espacio vertical y trazar l_nea horizontal gruesa


% Variable local para emacs, para  que encuentre el fichero maestro de
% compilaci_n y funcionen mejor algunas teclas r_pidas de AucTeX

\usepackage[colorinlistoftodos, textwidth=4cm, shadow]{todonotes}
%%%
%%% Local Variables:
%%% mode: latex
%%% TeX-master: "../Tesis.tex"
%%% End:
