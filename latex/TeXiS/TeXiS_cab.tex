%---------------------------------------------------------------------
%
%                          TeXiS_cab.tex
%
%---------------------------------------------------------------------
%
% TeXiS_cab.tex
% Copyright 2009 Marco Antonio Gomez-Martin, Pedro Pablo Gomez-Martin
%
% This file belongs to TeXiS, a LaTeX template for writting
% Thesis and other documents. The complete last TeXiS package can
% be obtained from http://gaia.fdi.ucm.es/projects/texis/
%
% This work may be distributed and/or modified under the
% conditions of the LaTeX Project Public License, either version 1.3
% of this license or (at your option) any later version.
% The latest version of this license is in
%   http://www.latex-project.org/lppl.txt
% and version 1.3 or later is part of all distributions of LaTeX
% version 2005/12/01 or later.
%
% This work has the LPPL maintenance status `maintained'.
%
% The Current Maintainers of this work are Marco Antonio Gomez-Martin
% and Pedro Pablo Gomez-Martin
%
%---------------------------------------------------------------------
%
% Contiene  los comandos  para poder  tener cabeceras  en  las p_ginas
% distintas de las que se configuran por defecto.
%
% Define  adem_s  algunos  comandos   para  utilizar  en  cap_tulos  y
% secciones "especiales"  que no siguen  la cabecera normal,  como los
% agradecimientos, resumen, tablas de _ndices y secciones de resumen y
% notas bibliogr_ficas.
%
%---------------------------------------------------------------------


% EXPLICACI_N DEL FANCYHDR (paquete utilizado para las cabeceras)

% Para definir las cabeceras que  se quieren, se utilizan los comandos
% \fancyhead y  \fancyfoot. Ambos reciben  como par_metro entre  [] el
% sitio donde se  quiere establecer. Por ejemplo si  quiero cambiar lo
% que aparece en las p_ginas pares (_E_ven), en la parte de la derecha
% (_R_ight) de la cabecera  (_head_) se utiliza \fancyhead[ER]{Esto va
%   a la  derecha en  las pares}. En  realidad, esos  par_metros entre
% corchetes se pueden agrupar. Por  ejemplo, si queremos que el n_mero
% de p_gina (\thepage)  aparezca en las pares a la  izquierda y en las
% impares a la derecha, se usa \fancyhead[LE,RO]{\thepage}.
%
% Por otro  lado, en la  pr_ctica se utiliza  un texto en  las p_ginas
% izquierdas (es decir, las pares),  y otra en las derechas (es decir,
% impares). Por eso, existen dos comandos preestablecidos, \righmark y
% \leftmark,  que  contienen el  texto  que  aparecer_.  Lo normal  es
% definir         despu_s:         \fancyhead[LO]{\rightmark}        y
% \fancyhead[RE]{\leftmark}, para  que a  la izquierda de  las p_ginas
% impares aparezca la  marca destinada a las p_ginas  de la derecha, y
% en  la  derecha  de las  pares  el  texto  para  las p_ginas  de  la
% izquierda.
%
% Por  lo tanto,  si  queremos  que en  la  izquierda (p_ginas  pares)
% aparezca el  nombre del cap_tulo,  habr_ que "definir"  el \leftmark
% con el nombre.  Si queremos que en la  derecha (impares) aparezca el
% nombre de la secci_n, habr_ que "definir" as_ el \rightmark.
%
% La realidad  es que no se  definen esos dos comandos  (de hecho creo
% que no son  comandos como tal redefinibles...}, sino  que se utiliza
% el comando \markboth. Seg_n la documentaci_n del paquete fancyhdr:
%
% The \leftmark  contains the Left  argument of the Last  \markboth on
% the  page, the  \rihtmark contains  de Right  argument of  the first
% \markboth or the only argument  of the first \markright on the page.
% If no  marks are  present on  a page they  are "inherited"  from the
% previous page.
%
% Es decir, el \markboth debe ser  alg_n tipo de comando que tiene dos
% par_metros, y  que se puede poner  en cualquier lugar  de la p_gina,
% para cambiar el t_tulo. Por  ejemplo, se puede poner al principio de
% un cap_tulo con \markboth{Titulo del capitulo}{}.
%
% En  realidad, ese \markboth  intuyo que  est_ ya  por defecto  en el
% comando \chapter, igual  que hay otro similar en  las secciones.  Si
% se quiere poner otros, basta con redefinir el comando \chaptermark y
% \sectionmark, que intuyo que ser_n  como hooks dentro del \chapter y
% de \section. Y eso es precisamente lo que he utilizado al definir la
% cabecera.

% Paquete  para mejorar  el estilo  de  las cabeceras.  El paquete  es
% fancyhdr.   En  versiones   antiguas  de   LaTeX,  el   paquete  era
% fancyheader, que es el utilizado por ejemplo en el tutorial de LaTeX
% que sirvi_ de inspiraci_n para  esta plantilla. Adem_s de incluir el
% paquete,  definimos   el  estilo  de  cabeceras   que  queremos.  En
% particular, utilizaremos el  estilo "predefinido" fancy, con algunas
% modificaciones. Adem_s, cambio el estilo de las p_ginas de principio
% del cap_tulo (t_cnicamente estilo "plain"), para que el n_mero de la
% p_gina aparezca en la esquina de la derecha (abajo), en vez de en el
% centro. Para los curiosos, explicaci_n al final del fichero.
\usepackage{fancyhdr}

\pagestyle{fancy}

% Definici_n del estilo en las p_ginas normales:

% Alargar la cabecera ser_a:
%\addtolength{\headwidth}{\marginparwidth}

% Hacemos un poco m_s alta la cabecera, ya que el valor por defecto
% (12) no es suficiente para el tipo de letra y contenido de cabecera
% elegido, lo que provoca warnings cont_nuos en la compilaci_n
\addtolength{\headheight}{2pt}

% Establecemos el resto de par_metros de apariencia
\newcommand{\restauraCabecera}{
\fancyhead[LO]{\rightmark}
\fancyhead[RE]{\leftmark}
}

\renewcommand{\headrulewidth}{0.2pt}
\renewcommand{\chaptermark}[1]{
\markboth{\textsc{\chaptertitlename\ \thechapter.}\ #1}{}%
}
\renewcommand{\sectionmark}[1]{\markright{\thesection.\ #1}}
\fancyhf{}
\restauraCabecera
\fancyhead[RO,LE]{\thepage}

% Para los cap_tulos que no tienen numeraci_n en el _ndice ni tienen
% secciones, se debe cambiar la cabecera, para que no aparezca algo
% como "Cap_tulo 0. Agradecimientos". En esos cap_tulos, la cabecera
% es "especial", porque en las p_ginas impares, tampoco aparece el
% nombre de la secci_n, sino tambi_n el nombre del "cap_tulo". Para
% esos casos, se puede utilizar el comando siguiente.

% IMPORTANTE: este comando _sobreescribe_ el funcionamiento de la
% cabecera, hasta que se llame al comando \restauraCabecera, por lo
% despu_s del "cap_tulo especial", debe invocarse a
% \restauraCabecera. Y decir _despu_s_ significa _despu_s_, es decir
% cuando el cap_tulo YA HA TERMINADO, y se ha empezado el siguiente, o
% forzando final de p_gina con \newpage (\newpage\restauracabecera).
\newcommand{\cabeceraEspecial}[1]{
\fancyhead[LO]{\textsc{#1}}
\fancyhead[RE]{\textsc{#1}}
}

% La forma  "*" de  chapters y  sections NO llaman  a los  comandos de
% marca (chaptermark y sectionmark) que son los que finalmente alteran
% la cabecera. En  la tesis se utilizan secciones  sin numeraci_n para
% el resumen y notas bibliograficas de cada cap_tulo, y queda bastante
% mal que  se mantenga en la  cabecera el nombre de  la _ltima secci_n
% con numeraci_n.
%
% La soluci_n es  modificar a mano la cabecera  (en concreto, la parte
% izquierda  de las  p_ginas  impares, es  decir  el \markright).  Por
% ejemplo, para las conclusiones habr_a que hacer:
%
% \section*{Resumen\markright{Resumen}}
%
% Para evitar tener  que hacerlo todo el tiempo,  creamos los comandos
% \Resumen y  \NotasBibliograficas que lo  haga por nosotros.   El uso
% ser_:
%
% \section*{
%                   \Resumen
% }
%
% Por otro lado, al ser secciones  sin numerar no se meten en la tabla
% de contenidos ni  como marcador para que aparezca  en el "contenido"
% del PDF mostrado  por el lector. Para que lo haga,  hay que a_adir a
% mano la entrada en el toc. Por ejemplo, para las conclusiones:
%
% \addcontentsline{toc}{section}{Resumen}
%
%
% que dice "quiero meter en el TOC (Table Of Content) una entrada tipo
% section, con el nombre 'Resumen  ".  Esto tiene el efecto lateral de
% a_adir el marcador en el PDF.
%
% Esto  ya   no  lo   podemos  meter  en   los  comandos   \Resumen  y
% \NotasBibliograficas, porque como lo estamos usando _dentro_ del
% \section*, LaTeX a_n  no ha definido la posici_n en  el PDF y aunque
% el  TOC sale  bien, el  enlace que  se  pone en  dicho TOC  y en  el
% contenido del PDF  mostrado por el lector no sale  bien. En lugar de
% eso,  hace referencia al  _ltimo marcador  (por ejemplo,  la secci_n
% anterior, o la _ltima figura).
%
%
% Hay que ejecutar el comando \addcontentsline por tanto _despu_s_ del
% comando \section. Una alternativa ser_a definir \Resumen de modo que
% _l   mismo  hiciera   el  \section*{...}    completo,  y   luego  el
% \addcontentsline... y as_ no habr_a problemas. Pero esta soluci_n no
% es  compatible  con  la   capacidad  de  los  editores  de  resaltar
% secciones. Por tanto, cada vez que pongas la secci_n de conclusiones
% o  de en  el pr_ximo  cap_tulo  tienes que,  MANUALMENTE, a_adir  el
% \addcontentsline. Hazlo  _despu_s_ del  \section*. En otro  caso, se
% utilizar_ tambi_n en  el enlace del PDF la  _ltima secci_n o figura.
% Haci_ndolo despu_s  en realidad  no queda del  todo bien,  porque el
% enlace queda a la primera letra de la secci_n, en lugar de al t_tulo
% como ocurre con  las secciones numeradas, pero no  he encontrado una
% forma de hacerlo mejor y tampoco quiero perder m_s tiempo.
%
% Para    eso,    hay    otros    dos    comandos,    \TocResumen    y
% \TocNotasBibliograficas

% En resumen. Modo de uso:
%
%%---------------------------------------------------------------------
%\section*{
%                               \Resumen
%}
%%---------------------------------------------------------------------
%\TocResumen

\newcommand{\Resumen}{Resumen\markright{Resumen}}
\newcommand{\TocResumen}{\addcontentsline{toc}{section}{Resumen}}

\newcommand{\NotasBibliograficas}{Notas bibliogr_ficas\markright{Notas
    bibliográficas}}
\newcommand{\TocNotasBibliograficas}{\addcontentsline{toc}{section}{Notas
    bibliográficas}}

\newcommand{\ProximoCapitulo}{En el próximo
  capítulo\ldots\markright{En el próximo capítulo\ldots}}
\newcommand{\TocProximoCapitulo}{\addcontentsline{toc}{section}{En
    el próximo capítulo}}

\newcommand{\Conclusiones}{Conclusiones\markright{Conclusiones\ldots}}
\newcommand{\TocConclusiones}{\addcontentsline{toc}{section}{Conclusiones}}

% Para el ap_ndice, hay que indicar que queremos que ponga "Ap_ndice
% X", y no "Cap_tulo X" como hace normalmente.

% Si estamos en modo Debug, ponemos en el pie de p_gina una indicaci_n.

\ifx\release\undefined
\fancyfoot[LE,RO]{\vspace*{1cm}\small \sc Borrador -- \today}
\else
\fi

% Definici_n del estilo en la  p_gina de inicio de cap_tulo: N_mero de
% la p_gina abajo a la derecha, y sin l_nea en la zona superior.
\fancypagestyle{plain}{%
  \fancyhf{}
  \fancyfoot[R]{\thepage}
  \renewcommand{\headrulewidth}{0pt}
  \renewcommand{\footrulewidth}{0pt}
}

% Cuando   se   cambia   de    cap_tulo,   se   ejecuta   el   comando
% \cleardoublepage.  Si queremos que  la posible  p_gina que  se queda
% completamente en blanco NO tenga cabeceras, tenemos que redefinir el
% comando. (Cogido de la documentaci_n del fancyhdr)
\makeatletter
\def\cleardoublepage{\clearpage\if@twoside \ifodd\c@page\else
  \hbox{}
  \thispagestyle{empty}
  \newpage
  \if@twocolumn\hbox{}\newpage\fi\fi\fi}
\makeatother



% Variable local para emacs, para  que encuentre el fichero maestro de
% compilaci_n y funcionen mejor algunas teclas r_pidas de AucTeX

%%%
%%% Local Variables:
%%% mode: latex
%%% TeX-master: "../Tesis.tex"
%%% End:
