%---------------------------------------------------------------------
%
%                          Apéndice 1
%
%---------------------------------------------------------------------


\chapter{Anexo 1: Reuniones del equipo}
\label{ap1:Reuniones}

\begin{FraseCelebre}
\begin{Frase}
...
\end{Frase}
\begin{Fuente}
...
\end{Fuente}
\end{FraseCelebre}

\begin{resumen}
...
\end{resumen}

\section{Reunión del 07 de Septiembre de 2017}

En la primera reunión del equipo se hicieron las presentaciones de los integrantes, y se discutieron las posibles ideas que se podrían implementar como proyecto en la \textit{hackathon}.
El tema principal sobre el que se discutía era el impacto social del proyecto, y que las métricas de la \textit{hackathon} así lo exigían.

Realizamos una tormenta de ideas en la que surgieron las siguientes:
asdfads
\begin{itemize}
  \item \textbf{Plataforma de publicación de artículos académicos distribuida}: Implementar un sistema de publicación de artículos para compartir a través de la comunidad científica utilizando IPFS. Esta plataforma pretende eliminar los costes para el acceso a los artículos que imponen las empresas que se encargan de publicarlos y se benefician por ello. Implementar una plataforma totalmente descentralizada para compartir los artículos de divulgación científica conseguirá que el conocimiento de la investigación académica sea público y accesible por todos.

  \item \textbf{Wikipedia distribuida con modelos de gobernanza}: La idea de este proyecto inicialmente era descentralizar la plataforma de Wikipedia a través de IFPS y añadir algún modelo de gobernanza y de reputación para las revisiones de los artículos. EL problema es que es un proyecto muy complejo para implementarlo en sólo un mes, y haría falta un equipo bastante grande y la colaboración de la propia Wikipedia para llevarlo a cabo.

  \item \textbf{Aplicación de contactos para homosexuales en países donde son colectivos reprimidos}: En países como Rusia, los colectivos LGBT son reprimidos hasta el punto de que expresar su sexualidad puede ser un peligro para su seguridad personal. Esta idea trataba de poner en contacto de la manera más anónima y discreta posible a esas personas sin exponerse a los riesgos que ello conlleva.

  \item \textbf{ONG distribuida}: Esta plataforma pretendía ofrecer una bolsa de dinero en la que las personas iban realizando donaciones. Cada semana los donantes votaban dónde se iban a invertir el dinero mediante un sistema de votos.

  \item \textbf{Plataforma de intercambio de conocimientos de programación distribuida}: Stack Exchange es una de las web más importantes en la comunidad informática. Esta solución propone una altenativa totalmente distribuida mediante blockchain.

  \item \textbf{Plataforma de toma de decisiones distribuida}: La toma de decisiones en comunidades reprimidas es bastante dificil. Mediante una aplicación de toma de decisiones en blockchain (como la que tiene Loomio), se pueden ofrecer una herramienta para que estas personas en riesgo de exclusión se hagan oir.

  \item \textbf{Plataforma de crowdfunding para \textit{wistleblowers}}: El problema de las plataformas de crowdfunding es que una vez que se financia el proyecto, el usuario sólo puede ver el final del producto esperando que lo que ha financiado sea como promenten los desarrolladores. Esta plataforma propondría una alternativa con varios entregables en función del dinero que se vaya consiguiendo.


\end{itemize}

...

% Variable local para emacs, para  que encuentre el fichero maestro de
% compilación y funcionen mejor algunas teclas rápidas de AucTeX
%%%
%%% Local Variables:
%%% mode: latex
%%% TeX-master: "../Tesis.tex"
%%% End:

\section{Reunión del 08 de Septiembre de 2017}

Una vez que el equipo ha decidido el proyecto que vamos a afrontar, nos reunimos para ir decidiendo poco a poco las funcionalidades que habría de tener nuestra plataforma. Algunas de ellas son:
